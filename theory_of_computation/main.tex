\documentclass[nobib, oneside, openany]{tufte-book}
% nobib     =   do not use package natbib, default in tufte-book.
%               it conflicts with biblatex.
% oneside   =   no blank page between chap and page do not move
%               left or right based on the page number.
% openany   =

% ------------------------------------
% |            GEOMETRY              |
% ------------------------------------
\setcounter{tocdepth}{2}
\setcounter{secnumdepth}{2}

\makeatletter
% Paragraph indentation and separation for normal text
\renewcommand{\@tufte@reset@par}{
    \setlength{\RaggedRightParindent}{0pt}
    \setlength{\JustifyingParindent}{0pt}
    \setlength{\parindent}{0pt}
    \setlength{\parskip}{4pt}
}
\@tufte@reset@par

% Paragraph indentation and separation for marginal text
\renewcommand{\@tufte@margin@par}{
    \setlength{\RaggedRightParindent}{0pt}
    \setlength{\JustifyingParindent}{0pt}
    \setlength{\parindent}{0pt}
    \setlength{\parskip}{4pt}
}


\makeatletter
\renewcommand{\maketitlepage}{
    \begingroup
        \setlength{\parindent}{0pt}
            {\fontsize{24}{24}\selectfont\textit{\@author}\par}
        \vspace{1.75in}
            {\fontsize{36}{54}\selectfont\@title\par}
        \vspace{0.5in}
            {\fontsize{14}{14}\selectfont\textsf{\smallcaps{\@date}}\par}
        \vfill{\fontsize{14}{14}\selectfont\textit{\@publisher}\par}
        \thispagestyle{empty}
    \endgroup
}
\makeatother

\geometry{
    %showframe,             % DEBUG ONLY -- displays the margins
	left=13mm,              % left margin
	textwidth=135mm,        % width of main text
	marginparsep=8mm,       % gutter between main text block and margin notes
	marginparwidth=50mm     % width of margin notes
}
\fontsize{10}{20}\selectfont

% chapter format
\titleformat{\chapter}
    {\huge\rmfamily\itshape\color{red}}             % format applied to label+text
    {\llap{\colorbox{red}{\parbox{1.5cm}{\hfill\itshape\huge\color{white}\thechapter}}}}
    {2pt}   % horizontal separation between label and title body
    {}      % before the title body
    []      % after the title body

% section format
\titleformat{\section}
    {\normalfont\Large\itshape\color{orange}}       % format applied to label+text
    {\llap{\colorbox{orange}{\parbox{1.5cm}{\hfill\color{white}\thesection}}}}
    {1em}   % horizontal separation between label and title body
    {}      % before the title body
    []      % after the title body

% subsection format
\titleformat{\subsection}
    {\normalfont\large\itshape\color{blue}}% format applied to label+text
    {\llap{\colorbox{blue}{\parbox{1.5cm}{\hfill\color{white}\thesubsection}}}}
    {1em}   % horizontal separation between label and title body
    {}      % before the title body
    []      % after the title body

% ------------------------------------
% |        GENERIC PACKAGES          |
% ------------------------------------
% Math
\usepackage{amsmath,amsthm,amssymb,amsfonts}
\usepackage{relsize, xfrac}

% Pseudocode
\usepackage{clrscode3e}         % Pseudocode as in "Introduction to algorithms"
\usepackage{minted}             % Code highlighting

% Hyperlinks in PDFs
\usepackage{hyperref}

% Lorem Ipsum text
\usepackage{lipsum}

% Images
\usepackage{graphicx}
\setkeys{Gin}{width=\linewidth,totalheight=\textheight,keepaspectratio}
\graphicspath{/images}

% Verb environment
\usepackage{fancyvrb}           % Customization of verb environments.
\fvset{fontsize=\normalsize}    % Use a smaller font for the verb environment.

% Language
\usepackage[italian]{babel}

% Bibliography
\usepackage{csquotes}
\usepackage[
    backend=biber,
    style=alphabetic,
    sorting=ynt
]{biblatex}
\addbibresource{bib/bibliography.bib}

% Filter out some warning
\usepackage{silence}
  \WarningFilter*{latex}
    {Marginpar on page \thepage\space moved}
  \WarningFilter{biblatex}          % ONLY IF BIBLATEX DOES NOT USE IBID STYLE REF.
    {Patching footnote}

% IDK
\usepackage{booktabs}               % Makes prettier tables.
\usepackage{multicol}               % Small sections of multiple columns in
                                    % documents and tables.

\usepackage{bookmark}
\usepackage[
    activate={true,nocompatibility},
    final,
    tracking=true,
    nopatch=footnote,
    kerning=true,
    spacing=false,
    factor=1100,
    stretch=10,
    shrink=10
]{microtype}

\usepackage{pgf,tikz,tikz-cd}

\usepackage{fancyhdr}               % Custom headers and footers.
\pagestyle{fancyplain}              % Makes all pages in the document conform
                                    % to the custom headers and footers.
\fancyhead{}                        % No page header - if you want one, create
                                    % it in the same way as the footers below.
\fancyfoot[L]{}                     % Empty left footer.
\fancyfoot[C]{}                     % Empty center footer.
\fancyfoot[R]{\thepage}             % Page numbering for right footer.
\renewcommand{\headrulewidth}{0pt}  % Remove header underlines.
\renewcommand{\footrulewidth}{0pt}  % Remove footer underlines.
\setlength{\headheight}{13.6pt}     % Customize the height of the header.
\allowdisplaybreaks

\usepackage{tabularx,environ} % For problem definition


% ------------------------------------
% |      THEOREM ENVIRONMENTS        |
% ------------------------------------
\newtheorem{thm}{Theorem}
\newtheorem{cor}[thm]{Corollary}
\newtheorem{prop}[thm]{Proposition}
\newtheorem{lem}[thm]{Lemma}
\newtheorem{conj}[thm]{Conjecture}
\newtheorem{quest}[thm]{Question}
\newtheorem{claim}{Claim}
\newtheorem{property}{Proprietà}

\theoremstyle{definition}
\newtheorem{defn}[thm]{Definition}
\newtheorem{defns}[thm]{Definitions}
\newtheorem{con}[thm]{Construction}
\newtheorem{exmp}[thm]{Example}
\newtheorem{exmps}[thm]{Examples}
\newtheorem{notn}[thm]{Notation}
\newtheorem{notns}[thm]{Notations}
\newtheorem{addm}[thm]{Addendum}
\newtheorem{exer}[thm]{Exercise}

\theoremstyle{remark}
\newtheorem{rem}[thm]{Osservazione}
\newtheorem{ans}[thm]{Answer}
\newtheorem{rems}[thm]{Remarks}
\newtheorem{warn}[thm]{Warning}
\newtheorem{sch}[thm]{Scholium}

% ------------------------------------
% |       COMMAND DEFINITIONS        |
% ------------------------------------
% Operators
\newcommand{\Mod}[1]{\ (\text{mod}\ #1)}
\newcommand{\N}{\mathbb{N}}
\newcommand{\Z}{\mathbb{Z}}
\newcommand{\Q}{\mathbb{Q}}
\newcommand{\R}{\mathbb{R}}
\newcommand{\C}{\mathbb{C}}
\newcommand{\OR}{\,|\,}
\newcommand{\AND}{\,\&\,}
\newcommand{\NOT}{!}

% Parenthesis
\newcommand{\bra}[1]{\left(#1\right)}
\newcommand{\sbra}[1]{\left[#1\right]}
\newcommand{\cbra}[1]{\left\{#1\right\}}
\newcommand{\norm}[1]{\left\lVert#1\right\rVert}
\newcommand{\ang}[1]{\left\langle#1\right\rangle}

% Shortened versions
\newcommand{\tn}[1]{\textnormal{#1}}

% Other
\DeclareMathOperator*{\argmin}{argmin}
\DeclareMathOperator*{\argmax}{argmax}
\newcommand{\upperAccE}{È \,}

% ------------------------------------
% |    DOCUMENT-SPECIFIC PACKAGES    |
% ------------------------------------

% ------------------------------------
% |    DOCUMENT-SPECIFIC COMMANDS    |
% ------------------------------------

% Abbreviations for pattern matching
\newcommand{\occ}{\tn{Occ}}
\newcommand{\bwt}{\tn{BWT}}
\newcommand{\lf}{\tn{LF}}

% Cost function for approx algorithms
\newcommand{\opt}{\tn{opt}}
\newcommand{\appr}{\tn{A}}

% ------------------------------------
% |  DOCUMENT-SPECIFIC ENVIRONMENTS  |
% ------------------------------------
\makeatletter           % https://tex.stackexchange.com/a/199244/26355
\newcolumntype{\expand}{}
\long\@namedef{NC@rewrite@\string\expand}{\expandafter\NC@find}
\NewEnviron{problem}[2][]{%
    \def\problem@arg{#1}%
    \def\problem@framed{framed}%
    \def\problem@lined{lined}%
    \def\problem@doublelined{doublelined}%
    \ifx\problem@arg\@empty%
        \def\problem@hline{}%
    \else%
        \ifx\problem@arg\problem@doublelined%
        \def\problem@hline{\hline\hline}%
        \else%
        \def\problem@hline{\hline}%
        \fi%
    \fi%
    \ifx\problem@arg\problem@framed%
        \def\problem@tablelayout{|>{\bfseries}lX|c}%
        \def\problem@title{\multicolumn{2}{|%
        >{\raisebox{-\fboxsep}}%
        p{\dimexpr\textwidth-4\fboxsep-2\arrayrulewidth\relax}%
        |}{%
            \textsc{\Large #2}%
        }}%
    \else
        \def\problem@tablelayout{>{\bfseries}lXc}%
        \def\problem@title{\multicolumn{2}{>%
        {\raisebox{-\fboxsep}}%
        p{\dimexpr\textwidth-4\fboxsep\relax}%
        }{%
            \textsc{\Large #2}%
        }}%
    \fi%
    \bigskip\par\noindent%
    \renewcommand{\arraystretch}{1.2}%
    \begin{tabularx}{\textwidth}{\expand\problem@tablelayout}%
        \problem@hline%
        \problem@title\\[2\fboxsep]%
        \BODY\\\problem@hline%
    \end{tabularx}%
    \medskip\par%
}
\makeatother


% ------------------------------------
% |         Main section             |
% ------------------------------------
\title{Teoria della\\Computazione}
\author{Mattia Bolognini}
\date{Ultima modifica: \today}

\begin{document}

\maketitle

\tableofcontents

\chapter{Teoria della computabilità}
La teoria della computabilità studia quali problemi sono risolvibili utilizzando una procedura di calcolo meccanica/automatica.
Ragionando in termini più matematici, una funzione $f: I \rightarrow O$(che altro non è che un problema) è calcolabile sse esiste un algoritmo che la può calcolare, ovvero che tale che l'algoritmo calcola correttamente, per ogni $x \in I$, il relativo $f(I)$.

Come si vedrà, non tutti i problemi sono computabili, come l'\textit{halting problem}.

\section{Macchine di Turing}
\subsection*{La macchina di Turing}
\subsubsection*{Cenni storici}
La nozione di algoritmo è stata sempre conosciuta e di facile comprensione: un algoritmo può essere infatti definito come una sequenza finita e ordinata di passi che porta, in un tempo finito, al risultato desiderato.
Solo nell'ultimo secolo, però, sono stati definiti dei modelli che effettivamente descrivono formalmente la nozione di algoritmo. Queste nuove descrizioni formali sono state sviluppate principalmente per rispondere al problema della decisione (Entscheidungsproblem), introdotto da Hilbert nella sua famosa lista di problemi irrisolti al suo tempo nel campo dei fondamenti della matematica. Il problema chiede se è possibile definire una procedura automatica che, data una qualsiasi WFF della logica del primo ordine, stabilisce se la formula è deducibile all'interno della logica del primo ordine, ovvero è un teorema. Church e Turing hanno risposto negativamente alla domanda, introducendo rispettivamente il lambda calcolo e le macchine di Turing.

\subsubsection*{Definizione formale}
Una macchina di Turing, formalmente, è una sestupla:
\begin{center}
    $\langle \Sigma, \textnormal{B}, \textnormal{Q}, q_0, \textnormal{F}, \delta \rangle$
\end{center}
dove:
\begin{itemize}
    \item $\Sigma$ è l'\textit{alfabeto di nastro}, ovvero l'insieme dei simboli che possono essere scritti su nastro. L'alfabeto di nastro è un insieme \textit{finito};
    \item B è il \textit{simbolo di blank}, un simbolo speciale usato per specificare che una cella del nastro è vuota. Il simbolo di blank appartiene all'alfabeto di nastro, ovvero $\textnormal{B} \in \Sigma$;
    \item Q è l'\textit{insieme degli stati} che compongono la FSM. L'insieme degli stati è un insieme \textit{finito};
    \item $q_0$ è lo \textit{stato iniziale} della FSM. $q_0$ è uno dei possibili stati della FSM, quindi $q_0 \in \textnormal{Q}$;
    \item F è l'\textit{insieme degli stati finali/di accettazione} della FSM. Vale $\textnormal{F} \subseteq \textnormal{Q}$;
    \item $\delta$ è la \textit{funzione di transizione}, che definisce come avvengono i passaggi di stato della FSM.
\end{itemize}

La parte più “importante” di una macchina di Turing, che di fatto definisce la sua semantica, è la funzione di transizione.
La funzione di transizione è definita come:
\begin{center}
    $\delta: \textnormal{Q} \times \Sigma \rightarrow \textnormal{Q} \times \Sigma \times \left\{ \leftarrow, \textnormal{\textendash}, \rightarrow \right\}$
\end{center}
ovvero, a partire dallo stato corrente e leggendo il simbolo di nastro puntato dalla testina, aggiorna lo stato della FSM, scrive nella cella puntata dalla testina un nuovo simbolo e, infine, muove la testina.

Essendo Q e $\Sigma$ due insiemi finiti, il loro prodotto cartesiano Q$\times \Sigma$ è un insieme finito, e di conseguenza è possibile elencare esplicitamente tutti gli elementi del dominio, anche se farlo effettivamente può richiedere molto tempo e spazio.

Tutte le macchine di Turing sono algoritmi e, viceversa, tutti gli algoritmi sono macchine di Turing.
Si può così formalizzare la nozione di algoritmo considerandolo, di fatto, una macchina di Turing.

Si può dire anche che una macchina di Turing è una funzione
\begin{center}
    $f:  \Sigma^* \rightarrow \Sigma^*$
\end{center}
che mappa stringhe su stringhe.

Un sottoinsieme di queste funzioni sono le \textit{funzioni di decisione}:
\begin{center}
    $f: \Sigma^* \rightarrow \{ 0, 1 \}$
\end{center}
Non si perde di generalità considerando solo le funzioni di decisione: ogni problema generico ha infatti un problema di decisione corrispondente.

Una funzione di decisione  può anche essere rappresentata come il linguaggio di tutte le stringhe per cui la funzione restituisce 1.
\begin{center}
    $L_D = \left\{ x \in \Sigma^* \; | \; f_D (x) = 1 \right\}$
\end{center}
Si può quindi affermare che si ha un'equivalenza tra i problemi (di decisione) e i linguaggi.
\subsubsection*{Rappresentazione informale}
Una macchina di Turing può essere vista come una \textit{macchina a stati finiti} deterministica operante su un \textit{nastro}, diviso in celle, di lunghezza infinita (e quindi memoria infinita); la macchina possiede, inoltre, una \textit{testina}, che punta ad una singola cella del nastro.
\begin{figure}[h]
    \centering
    \includegraphics[width=0.5\linewidth]{images/turing_machine.png}
    \caption{Rappresentazione informale della macchina di Turing}
    \label{fig:turing_machine}
\end{figure}

Per convenzione, d'ora in poi si considereranno solo macchine di Turing con nastro semi-infinito. Non si perde però capacità computazionale: qualsiasi macchina a nastro infinito può essere simulata con una macchina a nastro semi-infinito.

\subsubsection*{Configurazione di una macchina di Turing e complessità}
Lo stato in cui si trova una macchina di Turing a un certo istante di tempo può
essere descritto con la sua \textit{configurazione} (CID), ovvero la quadrupla:
\begin{center}
    $(q, \sigma, \textnormal{sx}, \textnormal{dx})$
\end{center}
dove:
\begin{itemize}
    \item $q$ è lo stato corrente della FSM;
    \item $\sigma$ è il simbolo di nastro presente nella cella puntata dalla testina.
    \item sx è il contenuto del nastro a sinistra della testina, quindi dal simbolo, diverso dal simbolo di blank, presente più a sinistra alla testina.
    \item dx è il contenuto del nastro a destra della testina, quindi dalla testina al simbolo, diverso dal simbolo di blank, più a destra.
\end{itemize}

Una \textit{computazione} della macchina di Turing può quindi essere espressa come una sequenza di CID.
Il passaggio da una CID alla successiva viene indicato con il simbolo di derivazione logica $\vdash$.

La computazione di una macchina di Turing può terminare, e quindi essere rappresentata con una sequenza finita di CID, o non terminare.
Sia $M$ una macchina di Turing e $x$ un input, se la computazione di $M$ su $x$ non termina, allora:
\begin{center}
    $M(x) = \bot$
\end{center}
$\bot$ non è l'output della computazione, è solo un simbolo per indicare la non terminazione della macchina.

Se, invece, la computazione sull'input $x$ termina, allora possono essere fatte considerazioni sulla sequenza di configurazioni attraversate per arrivare alla terminazione.

Dato un algoritmo e un input $x$, il numero di configurazioni che la macchina di Turing che lo modella necessita per terminare viene indicato con $t_M (x)$, dove $M$ è la macchina di Turing.
Detta $n$ la dimensione dell'input, la \textit{complessità dell'algoritmo} per input di dimensione $n$ è, invece, il massimo numero di configurazioni che la macchina può attraversare prima di terminare, dato un input di dimensione $n$, ed è indicata con $T_M (n)$.
\begin{center}
    $T_M (x) = \textnormal{max} \left\{ t_M (x) \; | \; x \in \textnormal{I} \land |x| = n \right\}$
\end{center}
Spesso, però, si è interessati alla sola crescita asintotica della complessità in funzione della dimensione dell'input, ovvero come varia $T_M (n)$ al variare di $n$.

L'output di una macchina di Turing che termina varia a seconda della convenzione adottata: può essere il contenuto del nastro al termine della computazione (usato per i problemi di ricerca) o lo stato della FSM al termine della computazione, in particolare YES se ci trova in uno stato di accettazione, NO altrimenti (usato per i problemi di decisione).

\subsection*{Macchina di Turing multinastro}
La macchina di Turing è stata fin'ora definita su un singolo nastro (semi-infinito).
È però possibile definire una \textit{macchina di Turing multinastro}.
In questo tipo di macchina, la transizione non dipende più, oltre che dallo stato corrente, da un solo simbolo di nastro: detto $k$ il numero di nastri, la funzione dipenderà dagli $k$ simboli puntati dalle $k$ testine, una per nastro. Le testine, inoltre, si possono muovere in modo completamente indipendente l'una dall'altra.
\begin{figure}[h]
    \centering
    \includegraphics[width=0.5\linewidth]{images/multitape_turing_machine.png}
    \caption{Macchina di Turing multinastro}
    \label{fig:multitape-turing-machine}
\end{figure}
Supponendo di aver assegnato ai $k$ nastri, e di conseguenza alle relative testine, un indice $1 \le i \le k$, la funzione di transizione per una macchina multinastro è definita come:
\begin{center}
    $\delta: \textnormal{Q} \times \Sigma^k \rightarrow \textnormal{Q} \times \Sigma^k \times \left\{ \leftarrow, \textnormal{\textendash}, \rightarrow \right\}^k$
\end{center}
Una CID è invece definita come:
\begin{center}
    $(q, \sigma^1, \textnormal{sx}^1, \textnormal{dx}^1, \ldots, \sigma^k, \textnormal{sx}^k, \textnormal{dx}^k)$
\end{center}

\subsection*{Equivalenza tra macchine a singolo nastro e macchine multinastro}
\begin{thm}
    Sia $M$ una macchina di Turing con $k$ nastri, $k>1$, allora esiste $M'$ macchina di Turing a singolo nastro equivalente, tale che $\forall x \in \Sigma*, \; M(x) = M'(x)$.
\end{thm}

\begin{proof}
    La dimostrazione del teorema è per costruzione. Si vuole dimostrare che:
    \begin{itemize}
        \item data una macchina a singolo nastro, è possibile costruire una macchina multinastro che la simula.
        \item data una macchina multinastro, è possibile costruire una macchina a singolo nastro che la simula.
    \end{itemize}
    Dimostrare il primo punto è molto semplice: una macchina a singolo nastro altro non è che una macchina multinastro con $k = 1$ nastri. Di conseguenza la dimostrazione è già stata fatta.\\
    Il punto critico è poter simulare una macchina multinastro usando una macchina a singolo nastro.\\
    Per farlo, basta semplicemente considerare le stringhe che rappresentano lo stato corrente di ognuno dei nastri, e concatenarle usando un nuovo carattere separatore, non ancora presente nell'alfabeto di nastro.\\
        INSERIRE IMMAGINE.\\
    Bisogna poter simulare le $k$ testine della macchina multinastro con una sola testina. Per farlo, per ogni simbolo $c$ dell'alfabeto di nastro della macchina multinastro viene definito il simbolo $\dot{c}$, che indica che la cella in cui è contenuto $c$ è puntata dalla testina del relativo nastro (simulato).
    Di conseguenza, se $\Sigma$ è l'alfabeto della macchina multinastro e $\Sigma'$ l'alfabeto della macchina a singolo nastro:
    \begin{center}
        METTERE RELAZIONE TRA CARDINALITà INSIEMI.
    \end{center}
    Per simulare le transizioni, invece, la macchina a singolo nastro parte dall'inizio del nastro (che abbiamo assunto semi-infinito), e scorre il nastro fino ad aver letto i $k$ simboli puntati. Durante lo scorrimento, però, la macchina deve poter memorizzare i simboli che legge, e lo fa usando gli stati della FSM: alla lettura di un particolare simbolo, la macchina cambia di stato, memorizzandolo. Si capisce però che, per rappresentare con la FSM tutte le possibili combinazioni degli $n$ simboli dell'alfabeto della multinastro che è possibile leggere, è necessario u numero di stati pari a:
    \begin{center}
        $\sum_{i=1}^{k} n^k = O(n^k)$ SBAGLIATO
    \end{center}
    Una volta letti tutti i $k$ simboli puntati, allora la macchina a singolo nastro compie la transizione corretta.
\end{proof}

\begin{thm}
    Sia $t_M (x)$ il tempo di calcolo di $M$, macchina di Turing a $k$ nastri, $k>1$, e sia $M'$ una macchina a singolo nastro che simula $M$, allora 
    \begin{center}
        $t_{M'} (x) = O \left( t_M (x) \right) \cdot t_M (x) = O \left( ({t_M (x)}^2 \right)$
    \end{center}
\end{thm}

\begin{proof}
    Sia $s_M (x)$ lo spazio esplorato dalla macchina durante l'esecuzione, allora sicuramente $t_M (x) \ge s_M (x)$ (non è infatti possibile visitare un numero di celle in un numero di passi inferiore al numero di celle stesso).
    Il numero di passi necessari alla macchina a singolo nastro per simulare una singola transizione è pari a $O(t_M (x))$: bisogna infatti scorrere tutto il nastro simulante fino a leggere tutti i $k$ simboli puntati dalle testine simulate, tornare a inizio nastro, scorrere ancora tutto il nastro simulato per aggiornare le celle puntate e spostare le testine simulate, e infine tornare a inizio nastro. Ogni passata richiede $O(t_M (x))$.
    Ma il numero totale di transizioni che la macchina multinastro effettua, e quindi il numero di transizioni da simulare, è pari a $t_M (x)$. Di conseguenza, il tempo impiegato dalla macchina a singolo nastro per simulare il funzionamento della macchina multinastro è pari a:
    \begin{center}
        $t_{M'} (x) = O \left( {t_M (x)}^2 \right)$ 
    \end{center}
\end{proof}

\begin{rem}
    I problemi trattabili con una macchina multinastro rimangono trattabili anche con una macchina a singolo nastro: sia infatti $t_M (x)$ polinomiale in funzione della dimensione dell'input (e quindi trattabile sulla macchina multinastro), allora $t_{M'} (x) = O \left( {t_M (x)}^2 \right)$ è un quadrato di un polinomio, che a sua volta è un polinomio. 
\end{rem}

\subsection*{Macchina di Turing con alfabeto binario}
\begin{thm}
    Sia $M$ una macchina di Turing su alfabeto $\Sigma$, allora esiste $M'$ con alfabeto $\Sigma' = \left\{ \triangleright, B, 0, 1 \right\}$ equivalente.
\end{thm}
\begin{proof}
    La costruzione della macchina $M'$ ad alfabeto binario prevede di effettuare la codifica a blocchi dei simboli dell'alfabeto $\Sigma$.
    Ovviamente, per rappresentare un singolo simbolo sarà necessario un numero di bit pari a $n = \lceil \log_2 |\Sigma \setminus \left\{ \triangleright, B \right\}| \, \rceil$.
\end{proof}
\begin{rem}
    Il tempo richiesto dalla macchina a codifica binaria $M'$ per simulare la macchina $M$ su un input $x$ è $t_{M'}(x) = t_M(x) \cdot \lceil \log_2 |\Sigma \setminus \left\{ \triangleright, B \right\}| \, \rceil$. Ma $\lceil \log_2 |\Sigma \setminus \left\{ \triangleright, B \right\}| \, \rceil$ è costante, e quindi la complessità è:
    \begin{center}
        $t_{M'}(x) = O(t_M(x))$
    \end{center}
\end{rem}

\subsection*{Macchina di Turing universale}
La \textit{macchina di Turing universale} è una macchina di Turing che accetta in input un'altra macchina, codificata come stringa, e un input per la macchina ricevuta in input.
I computer moderni sono macchine universali, in quanto ricevono in input un programma (che è la codifica di una macchina di Turing) e un input per il programma.\\
Per poter descrivere una macchina universale è, però, necessario specificare come una macchina di Turing può essere codificata tramite una stringa. Per farlo, basta

FINIRE CODIFICA MACCHINA DI TURING

Tutte le macchine di Turing possono essere codificate sottoforma di stringa e, viceversa, ogni stringa codifica una macchina di Turing.
Non è però detto che tutte le stringhe portino a una macchina di Turing valida: sarà compito dell'altra macchina, che la riceverà in input, di accertarsi di ciò.

\begin{thm}
    Esiste una macchina di Turing $U$, detta macchina di Turing universale, tale che, $\forall \alpha, x \in \left( \Sigma - \{ \triangleright , B \} \right)^*$, si ha $U(\alpha, x) = M_{\alpha} (x)$.
\end{thm}
dove $\alpha$ è la codifica della macchina di Turing a singolo nastro da simulare, mentre $x$ è l'input per $\alpha$.\\
Per convenzione, se $\alpha$ è la codifica di una macchina di Turing non valida, allora $U$ si arresta immediatamente.

\begin{proof}
    La dimostrazione viene fatta per costruzione.
    La macchina universale costruita è una multinastro, con $k = 3$, dove:
    \begin{itemize}
        \item il primo nastro coincide con il nastro di $M_{\alpha}$, e all'inizio contiene l'input $x$. La testina da simulare coincide con la testina di questo nastro;
        \item il secondo nastro contiene lo stato corrente della macchina simulata $M_{\alpha}$, più precisamente la codifica dello stato corrente. All'inizio conterrà lo stato iniziale di $M_{\alpha}$;
        \item il terzo nastro contiene la codifica $\alpha$.
    \end{itemize}
    SE POSSIBILE METTERE RAPPRESENTAZIONE DELLA MACCHINA CON NASTRI\\
    Una transizione della macchina $M_{\alpha}$ viene simulata nel seguente modo:
    \begin{enumerate}
        \item scorro il terzo nastro (quello della codifica) fino a trovare una regola di transizione definita sullo stato corrente $q$, contenuto nel secondo nastro.
        \item se ho trovato una regola di transizione sullo stato corrente, verifico se essa è definita sullo stesso simbolo di nastro $\sigma$ puntato dalla testina del primo nastro.\\
        Se ciò è vero, allora è stata trovata la regola di transizione corretta, ed è quindi possibile effettuarla: viene quindi copiato il nuovo simbolo nella cella puntata dalla testina del primo nastro, spostata quest'ultima e, infine, aggiornato lo stato presente sul secondo nastro.\\
        Se, invece, il simbolo dettato dalla regola non coincide con quello puntato dalla testina del primo nastro, allora si torna al punto 1.
        \item se la macchina universale non riesce a trovare alcuna regola di transizione valida per la configurazione corrente della macchina simulata, allora la macchina universale termina la sua esecuzione.
    \end{enumerate}
\end{proof}

\begin{thm}
    Sia $t_{M_{\alpha}} (x)$ il tempo di calcolo di $M_{\alpha}$ sull'input $x$, e sia $U$ la macchina di Turing universale, che riceve in input la codifica $\alpha$ e l'input $x$, allora:
    \begin{center}
        $t_U (\alpha, x) = O \left( |\alpha| \cdot t_{M_{\alpha}} (x) \right) = O \left( t_{M_{\alpha}} (x) \right)$
    \end{center}
\end{thm}

\begin{proof}
    Il tempo di calcolo necessario per simulare una transizione di $M_{\alpha}$ è pari a $O(1)$: è necessario infatti, per trovare la regola di transizione corretta, scorrere solo la codifica $\alpha$, contenuta nel terzo nastro della macchina universale, e, di conseguenza, sono necessari un numero di passi pari a $O(|\alpha|)$. Ma $|\alpha|$ è costante, e quindi $O(|\alpha|) = O(1)$.\\
    Il numero di transizioni da simulare è, invece, pari a $t_{M_{\alpha}} (x)$.
    Di conseguenza, $t_U (\alpha, x) = O(1) \cdot t_{M_{\alpha}} (x) = O \left( t_{M_{\alpha}} (x) \right)$.
\end{proof}

\begin{rem}
    Se $M_{\alpha}$ è una macchina a singolo nastro, allora la crescita asintotica della complessità della macchina universale che la simula, in funzione della dimensione $n$ dell'input $x$, è pari a:
    \begin{center}
        $T_U (n) = O \left( T_{M_{\alpha}} (n) \right)$
    \end{center}
    I problemi trattabili con una macchina di Turing $M_{\alpha}$ rimangono quindi trattabili se $M_{\alpha}$ viene simulata dalla macchina universale $U$.
\end{rem}

\subsection*{Decidibilità}
Se una macchina di Turing $M$ accetta tutte e sole le stringhe appartenenti a un linguaggio $L$, allora si dice che $M$ decide $L$.
Non tutti i linguaggi possono essere decisi da una macchina di Turing, e quindi non tutti i problemi possono essere risolti usando una macchina di Turing (e quindi un algoritmo): un esempio è l'halting problem.

I problema di decisione (o i rispettivi linguaggi) possono essere classificati in base alla loro decidibilità:
\begin{itemize}
    \item problemi decidibili (o linguaggi ricorsivi): la relativa macchina di Turing termina sempre e fornisce la risposta corretta;
    \item problemi semi-decidibili (o linguaggi ricorsivamente enumerabili): la relativa macchina di Turing termina sole se l'input $x$ appartiene al linguaggio, altrimenti non è garantita la sua terminazione. Un esempio di linguaggio semi-decidibile è quello relativo all'halting problem.
    \item problemi non decidibili (o linguaggi non ricorsivamente enumerabili): non si può garantire che la relativa macchina di Turing termini sull'input $x$, sia nel caso $x \in L$ che $x \notin L$. Un esempio di linguaggio non ricorsivamente enumerabile è il linguaggio di diagonalizzazione.
\end{itemize}
Anche i linguaggi semi-decidibili sono considerati indecidibili da una macchina di Turing.

Sui linguaggi valgono le seguenti relazioni:
\begin{itemize}
    \item sia $L$ un linguaggio ricorsivo, allora $L$ è ricorsivamente enumerabile.
    \item sia $L$ un linguaggio ricorsivo, allora $\overline{L} = \left\{ x \in \Sigma^* \; | \; x \notin L \right\}$ è ricorsivo.
    \item $L$ è ricorsivo $\leftrightarrow$ $L$ è ricorsivamente enumerabile e $\overline{L}$ è ricorsivamente enumerabile.
\end{itemize}

\subsection*{Macchina di Turing non deterministica}
RECUPERARE

Una macchina di Turing non deterministica accetta l'input $x$ sse esiste un ramo della computazione che accetta $x$.\\
Una macchina di Turing non deterministica rifiuta l'input $x$ sse non esiste un ramo della computazione che accetta $x$, ovvero tutti i rami rifiutano $x$.\\
In tutti gli altri casi, la macchina di Turing non deterministica non termina.

Formalmente, una macchina di Turing accetta quindi un input $x$ sse almeno uno dei suoi rami termina e accetta $x$, non richiedendo che anche tutti gli altri rami terminino. Per convenzione, però, verrà assunto che una macchina di Turing non deterministica accetta $x$ sse almeno un ramo accetta $x$ e tutti gli altri rami terminano.

\begin{thm}
    Sia $N$ una macchina di Turing non deterministica che accetta il linguaggio $L$, allora esiste $D$ macchina di Turing deterministica che accetta $L$.
\end{thm}
\begin{rem}
    Una macchina di Turing non deterministica non ha una capacità computazionale maggiore rispetto alla macchina di Turing deterministica.
\end{rem}
\begin{proof}
    Per dimostrare il teorema, bisogna mostrare che una macchina di Turing non deterministica può simulare una macchina di Turing deterministica e, viceversa, quella deterministica può simulare quella non deterministica.

    Il primo caso è molto semplice: una macchina deterministica può essere infatti simulata da una macchina non deterministica usando un solo ramo di computazione e nello stesso tempo di calcolo.
    Per quanto riguarda il secondo caso, invece, bisogna ricordare che la computazione di una macchina non deterministica definisce un albero delle computazioni. Questo albero può essere esplorato usando una visita BFS, e questo è quello che farà effettivamente la macchina deterministica per compiere la simulazione: utilizzerà quindi una coda, contenente le configurazioni correnti della macchina non deterministica, e adotterà una politica FIFO per la loro analisi e computazione.

    Non può essere usato l'algoritmo DFS perchè, potendo un ramo della macchina non deterministica non terminare, l'algoritmo si potrebbe bloccare su rami che non terminano, quando in realtà in rami successivi potrebbe essere presente un'accettazione dell'input. In questi casi, non si potrebbe quindi garantire l'accettazione di un input valido.
\end{proof}

\begin{thm}
    Se la macchina di Turing non deterministica $N$, che decide il linguaggio $L$, ha tempo computazionale pari a $T_N (n)$, con $n$ dimensione dell'input, allora la macchina deterministica $D$ che simula $M$ decide $L$ in tempo: 
    \begin{center}
        $O\left( {T_N (x)}^2 \cdot r^{2T_N (n)} \right)$
    \end{center}
    con $r$ grado uscente massimo di $N$.
\end{thm}
\begin{rem}
    La complessità computazionale della macchina $D$ è pari a un prodotto tra un polinomio e una funzione esponenziale: di conseguenza, la simulazione effettuata da $D$ risulta inefficiente.
\end{rem}
Il grado uscente di una macchina non deterministica è il numero massimo di configurazioni diverse che possono essere prodotte nel livello $i+1$ a partire da una singola configurazione nel livello $i$.
\begin{center}
    $r \le |\Sigma_N| \cdot |Q_N| \cdot 3$
\end{center}
\begin{proof}
    Il numero $l$ di foglie dell'albero definito dalla computazione di una macchina non deterministica $N$ è sicuramente $l \le O(r^{T_N(n)})$, dove $O(T_N(n))$ è l'altezza dell'albero e $r$ è il grado uscente della macchina non deterministica. 
    Lo spazio richiesto dalla macchina di Turing deterministica $D$ per simulare una singola foglia è pari alla dimensione del nastro usato dal ramo di computazione della macchina non deterministica $N$ associato alla foglia.
    Ma, per definizione, il tempo di computazione di un singolo ramo di una macchina di Turing non deterministica è polinomiale: di conseguenza, anche lo spazio richiesto deve essere polinomiale nella dimensione dell'input, ovvero $O(T_N(n))$. Di conseguenza, tutte le configurazioni poste nella coda dell'algoritmo BFS avranno dimensione $O(T_N(n))$.
    Lo spazio richiesto per simulare la macchina non deterministica $N$ usando una macchina deterministica $D$ è quindi pari a $O(r^{T_N(n)} \cdot T_N(n))$.
    
    Per quanto riguarda il tempo di esecuzione della macchina deterministica $D$, è stato affermato che la simulazione avviene tramite l'esecuzione dell'algoritmo BFS, che esplora l'albero definito dalla computazione della macchina non deterministica $N$.
    L'esecuzione dell'algoritmo BFS prevede l'utilizzo di una coda: deve infatti essere estratto il primo elemento, rappresentante la configurazione che si sta considerando, per poi accodare in fondo alla lista tutte le configurazioni raggiungibili dalla configurazione estratta. Ma questo implica, per ogni esplorazione di un singolo nodo, di dover scorrere la lista due volte: la prima per posizionarsi in fondo alla lista, la seconda per tornare alla posizione di partenza.
    La dimensione della coda è $O(r^{T_N(n)} \cdot T_N(n))$.
    Il numero di configurazioni presenti nella coda, e per ognuna della quali si deve simulare una transizione, è pari a $O(r^{T_N(n)})$. Ma il tempo richiesto per simulare tutte le transizioni da una singola configurazione estratta dalla coda è pari a $O(T_N(n))$.
    Di conseguenza, il tempo di esecuzione dell'intero algoritmo BFS è pari a $O({T_N(n)}^2 \cdot r^{2T_N(n)})$.
\end{proof}
\begin{rem}
    La dimostrazione per costruzione dimostra che esiste un algoritmo per simulare una macchina non deterministica $N$ usando una macchina deterministica $D$. Non viene, però, fatta alcuna affermazione sulla "qualità" dell'algoritmo, ovvero non viene affermato che esso sia effettivamente il migliore. Di conseguenza, il tempo di simulazione fornito è solo un upper-bound al possibile tempo di simulazione, da parte di una macchina deterministica, di una macchina non deterministica.
    Se si trovasse un algoritmo di simulazione polinomiale, allora P = NP.
\end{rem}

\subsection*{Esempi}

\subsubsection{Funzione di transizione della macchina che scrive "HELLO"}
FINIRE
La macchina di Turing che deve essere descritta si occuperà di scrivere sul suo unico nastro la stringa "HELLO".
Il suo alfabeto di nastro è $\Sigma = \{ \triangleright , B, H, E, L, L, O \}$, mentre il suo nastro è semi-limitato a destra.\\
Il suo funzionamento è molto semplice: scrive infatti H, per poi spostare la testina a destra e scrivere E, e così via.

$\delta = \left\{ 
(),
(),
(),
(),
(),
(),
(),
(),
\right\}$

\subsubsection{Decidere il linguaggio $\mathbf{a^nb^nc^n}$ con una macchina di Turing a singolo nastro}
La decisione del linguaggio $a^nb^nc^n$ con una macchina di Turing a singolo nastro avviene in due fasi: nella prima viene controllato che la stringa di input abbia struttura $a^*b^*c^*$, mentre la seconda controlla effettivamente che essa sia nella forma $a^nb^nc^n$.
L'utilizzo della sola seconda fase permetterebbe di accettare stringhe come $abcabc$, ovviamente non appartenente al linguaggio che si vuole riconoscere.

Fase 1:
\begin{enumerate}
    \item se leggi $a$, continua a leggere le successive $a$
    fino a quando non incontri un simbolo diverso da $a$.\\
    Se questo simbolo è $b$, allora passa al punto 2.\\
    Se questo simbolo è $c$, allora passa al punto 3.\\
    Se questo simbolo è il simbolo di blank, allora passa alla fase 2.
    \item se leggi $b$, continua a leggere le successive $b$ fino a quando non incontri un simbolo diverso da $b$.\\
    Se questo simbolo è $a$, allora fallimento.\\
    Se questo simbolo è $c$, allora passa al punto 3.\\
    Se questo simbolo è il simbolo di blank, allora passa alla fase 2.
    \item se leggi $c$, continua a leggere le successive $c$ fino a quando non incontri un simbolo diverso da $c$.\\
    Se questo simbolo è il simbolo di blank, allora passa alla fase 2.\\
    In tutti gli altri casi, fallimento.
\end{enumerate}
Fase 2:
\begin{enumerate}
    \item leggi una $a$ e sostituiscila con il simbolo di eliminazione.
    \item cerca a destra una $b$.
    \begin{itemize}
        \item se incontri il simbolo di eliminazione, ignoralo.
        \item se incontri una $c$ o il simbolo di blank, allora fallimento.
        \item se incontri una $b$, sostituiscila con il simbolo di eliminazione e passa al punto successivo.
    \end{itemize}
    \item cerca a destra una $c$.
    \begin{itemize}
        \item se incontri il simbolo di eliminazione, ignoralo.
        \item se incontri il simbolo di blank, allora fallimento.
        \item se incontri una $c$, sostituiscila con un simbolo di eliminazione e passa al punto successivo.
    \end{itemize}
    \item riposiziona la testina a inizio nastro.
    \item se sono presenti $a$ non lette, torna al punto 1.
    Altrimenti, scorri il nastro fino ad incontrare il primo simbolo di blank.
    \begin{itemize}
        \item se incontri il simbolo di eliminazione, ignoralo.
        \item se incontri $b$ o $c$, allora fallimento.
        \item se incontri il simbolo di blank, allora successo.
    \end{itemize}
\end{enumerate}


\subsubsection{Decidere il linguaggio $\mathbf{a^nb^nc^n}$ con una macchina di Turing multinastro SISTEMARE ITALIANO}
Un esempio di risoluzione di un problema tramite una macchina deterministica multinastro riguarda il linguaggio $a^nb^nc^n$.
Si vuole verificare che la stringa di input scritta sul nastro abbia effettivamente questa struttura, e per farlo si usa una macchina multinastro con 3 nastri.
Un algoritmo potrebbe essere il seguente:
\begin{enumerate}
    \item la stringa di input viene posta sul primo nastro. Secondo e terzo nastro sono vuoti.
    \item controlla che l'input sia della forma $a^*b^*c^*$. Per farlo, la macchina scorre semplicemente il primo nastro.
    Se l'input non è valido, allora si ha un fallimento.
    \item si posizionano tutte le testine dei nastri a inizio nastro.
    \item si legge il simbolo puntato dalla testina del primo nastro:
    \begin{itemize}
        \item se $a$, allora lo si ignora.
        \item se $b$, allora lo si sposta sul secondo nastro.
        \item se $c$, allora lo si sposta sul terzo nastro.
    \end{itemize}
    Dopo aver letto un simbolo, la testina del primo nastro viene spostata a destra, ed eventualmente anche la testina del nastro su cui si è spostato il simbolo viene spostata a destra.
    \item esegue il punto precedente fino a quando tutta la stringa di input (primo nastro) è stata letta.
    \item riposiziona tutte le testine agli inizi dei rispettivi nastri.
    \item legge i simboli puntati dalle 3 testine:
    \begin{itemize}
        \item se la prima testina punta a una $a$, la seconda a $b$ e la terza a $c$, allora sposta tutte le testine a destra e ripete questo punto.
        \item se i simboli puntati sono tutti il simbolo di blank, allora la stringa di input è della forma $a^nb^nc^n$. La macchina quindi termina con successo.
        \item in tutti gli altri casi, la stringa di input non è della forma $a^nb^nc^n$. La macchina quindi termina con fallimento.
    \end{itemize}
\end{enumerate}

\begin{rem}
    Il tempo richiesto dalla macchina di Turing a singolo nastro per verificare se una stringa appartiene al linguaggio $a^nb^nc^n$ è pari a $O(m^2)$, con $m$ lunghezza della stringa.
\end{rem}
\chapter{Teoria della complessità computazionale}
\section{Problemi di decisione}
\subsection*{Classi di complessità}
I problemi di decisione, o più precisamente i linguaggi, possono essere
classificati in base alla loro complessità computazionale. Un problema viene
classificato in base all'algoritmo di complessità più bassa tra tutti gli
algoritmi di risoluzione conosciuti per il problema.
La classificazione non indica, però, che il problema non può avere un algoritmo
migliore: la complessità migliore fornisce quindi solo un upper-bound alla
complessità del miglior algoritmo possibile.

\begin{thm}
    La classe $\textnormal{PTIME} \left( f(n) \right)$ è la classe che contiene
    tutti i problemi di decisione risolvibili in tempo $O(f(n))$ da una macchina
    di Turing deterministica.
    \begin{center}
        $\textnormal{DTIME} \left( f(n) \right) =
        \bra{ L \subseteq \Sigma^* \; | \; L \textnormal{ è deciso da una DTM in tempo } O \bra{ f(n) }}$
    \end{center}
\end{thm}

\begin{rem}
    La classe dei problemi decidibili in tempo polinomiale su una macchina di Turing deterministica è:
    \begin{center}
        $\textnormal{P} = \left\{ \bigcup\limits_{c \ge 0} \textnormal{DTIME}(n^c) \right\}$
    \end{center}
\end{rem}

\begin{thm}
    La classe $\textnormal{NTIME} \left( f(n) \right)$ è la classe che contiene tutti i problemi di decisione risolvibili in tempo $O(f(n))$ da una macchina di Turing non deterministica.
    \begin{center}
        $\textnormal{NTIME} \left( f(n) \right) = \left\{ L \subseteq \Sigma^* \; | \; L \textnormal{ è deciso da una NDTM in tempo } O \left( f(n) \right) \right\}$
    \end{center}
\end{thm}

\begin{rem}
    La classe dei problemi decidibili in tempo polinomiale su una macchina di Turing non deterministica è:
    \begin{center}
        $\textnormal{NP} = \left\{ \bigcup\limits_{c \ge 0} \textnormal{NTIME}(n^c) \right\}$
    \end{center}
\end{rem}

\begin{rem}
    Vale la relazione $\textnormal{P} \subseteq \textnormal{NP}$, in quanto ogni macchina deterministica può essere simulata in tempo polinomiale da una macchina non deterministica usando un solo ramo di computazione.
\end{rem}

È possibile definire due ulteriori classi, coP e coNP, che contengono i problemi complementari dei problemi di decisione contenuti rispettivamente in P e NP. Con problema complementare $\overline{L}$ si intende quel problema definito allo stesso modo del problema $L$, ma in cui le risposte fornite sono invertite: $w \in L \rightarrow w \notin \overline{L}$ e $w \in \overline{L} \rightarrow w \notin L$.
\begin{center}
    coP = $\left\{ L \subseteq \Sigma^* \; | \; \overline{L} \in P \right\}$\\
    coNP = $\left\{ L \subseteq \Sigma^* \; | \; \overline{L} \in NP \right\}$
\end{center}

È importante notare che coP $\neq \overline{P}$: $\overline{P}$ contiene infatti tutti i problemi che non sono in P (quindi anche quelli non decidibili), mentre coP contiene tutti i problemi complementari ai problemi in P. Il concetto è quello di applicare l'operatore di complemento ai singoli problemi (linguaggi) in P, non a tutto l'insieme P.
Lo stesso ragionamento può essere applicato a coNP $\neq$ NP.

\begin{thm}
    \textnormal{P = coP}
\end{thm}
\begin{proof}
    Sia un linguaggio $\textnormal{L} \in \textnormal{P}$, allora esiste, ovviamente, una macchina di Turing deterministica $M$ che decide $L$ in tempo $O(n^k)$.
    Si può costruire un'altra macchina di Turing deterministica $M'$ che, dato un input $x$, restituisce YES sse $M(x) = \textnormal{NO}$ e restituisce NO sse $M(x) = \textnormal{YES}$. Per farlo, serve semplicemente invertire gli stati di accettazione e di non accettazione che compongono la FSM di $M$.

    Siccome $M$ ha complessità $O(n^k)$, anche $M'$ avrà complessità $O(n^k)$.
    Di conseguenza, $\overline{L} \in \textnormal{P}$ per qualunque linguaggio $L$, e quindi P = coP.
\end{proof}

\begin{rem}
    Non è possibile applicare lo stesso ragionamento usato per dimostrare P = coP anche per NP = coNP.
\end{rem}
\begin{proof}
    La computazione di una macchina di Turing non deterministica può essere rappresentata come un albero, in cui ognuno dei rami porta a YES o NO (si assume quindi che tutti i rami terminino).
    INSERIRE IMMAGINE
    Invertire gli stati di accettazione della macchina, come fatto per P = coP, significa, di fatto, invertire i risultati forniti dai rami di esecuzione.
    INSERIRE IMMAGINE
    Si può notare, però, che una stringa appartenente al linguaggio deciso dalla macchina non deterministica continuerà ad appartenere al linguaggio anche invertendo gli stati di accettazione (e quindi i risultati forniti dai diversi rami): di conseguenza, non si è ancora riusciti a costruire una macchina di Turing non deterministica che accetti il linguaggio complementare di un linguaggio in NP.

    Se si dimostrasse che coNP = NP, allora si sarebbe dimostrato P = NP.
\end{proof}


\subsection*{Equivalenza tra decidibilità non det. in tempo $O(n^k)$ e verificabilità det. in tempo $O(n^k)$}
Per dimostrare che un linguaggio appartiene a NP, si potrebbe pensare di dimostrare che esso può essere deciso in tempo polinomiale da una macchina di Turing non deterministica. Può però risultare complicato costruire una macchina di questo tipo, e per questo, spesso, si preferisce dimostrare l'appartenenza a NP dimostrando che una stringa sull'alfabeto può essere verificata (ovvero si può affermare se appartiene o meno al linguaggio) da una macchina di Turing deterministica, detta \textit{verificatore}, in tempo polinomiale, usando una stringa aggiuntiva detta \textit{certificato}.

\begin{thm}
    Sia $L$ un linguaggio,
    \begin{center}
        $\textnormal{L} \in \textnormal{NP} \leftrightarrow \exists a \in \mathbb{N} \land V \textnormal{ DTM}$ tali che $\forall x \in \Sigma^*, \exists w \in \Sigma^{|x|^a}\; : \; V(x, w) = Y$
    \end{center}
    dove $w$ è detta certificato e ha lunghezza polinomiale in funzione della dimensione dell'input $|x|$.
\end{thm}
Un certificato indica, di fatto, quale dei possibili rami di esecuzione della macchina di Turing non deterministica, che riconosce $L$, deve essere valutato dal verificatore.

\begin{proof}
    In una macchina non deterministica che lavora in tempo polinomiale, un singolo ramo di esecuzione richiede ovviamente tempo polinomiale.
    Un certificato specifica quale ramo di esecuzione considerare, specificando quale sottoalbero analizzare quando ci si trova ad un certo nodo dell'albero.
    Essendo però il tempo di esecuzione di un ramo di esecuzione della macchina non deterministica limitato polinomialmente nella dimensione dell'input, allora anche il numero di nodi dell'albero attraversati dal ramo sarà limitato polinomialmente nella dimensione dell'input (non è possibile infatti che lo spazio cresca più velocemente del tempo, altrimenti non si riuscirebbe ad esplorare tutto lo spazio occupato).
    Di conseguenza, il certificato, che descrive un singolo ramo di esecuzione, avrà dimensione polinomiale nella dimensione dell'input.
    Verificare un certificato significa verificare il risultato a cui porta il ramo di esecuzione della macchina non deterministica da lui descritto.
    Ma, essendo il certificato di dimensione polinomiale, il ramo può essere simulato, e quindi verificato, da una macchina di Turing deterministica in tempo polinomiale, che prende il nome di verificatore.

    Viceversa, se esiste un verificatore, ovvero una macchina di Turing deterministica che simula, in tempo polinomiale, un certo ramo di esecuzione descritto da un certificato, allora è possibile costruire la macchina di Turing non deterministica che decide, in tempo polinomiale, il linguaggio $L$ considerato.
    Basta infatti generare tutti i possibili certificati $w \in \Sigma^*$. Dato un input $x$, quindi, la sua appartenenza al linguaggio può essere verificata generando tutti i possibili certificati, e applicando il verificatore a ogni possibile coppia $x$-certificato: se il verificatore restituisce un risultato positivo per almeno un certificato, allora $x \in L$; se invece nessun certificato porta a YES, allora $x \notin L$.
    Ma queste operazioni di verifica di tutti i possibili certificati possono essere effettuate parallelamente da una macchina di Turing non deterministica, qualsiasi sia l'input $x$.
    Di conseguenza, la macchina non deterministica che decide, in tempo polinomiale, il linguaggio $L$ è quella che esegue, per un certo ramo di esecuzione, il verificatore applicato a uno dei certificati.
\end{proof}

\subsection*{Riduzioni}
\begin{defn}
    Un linguaggio $L_A$ è riducibile polinomialmente a un linguaggio $L_B$, indicato con $L_A \le_{P} L_B$, sse esiste una funzione $f: \Sigma^* \rightarrow \Sigma^*$ calcolabile in tempo polinomiale da una macchina di Turing deterministica tale che $\forall w \in \Sigma^*, w \in L_A \leftrightarrow f(w) \in L_B$. Questo tipo di riduzioni prende il nome di riduzioni di Karp.
\end{defn}
Le riduzioni di Karp non sono l'unico tipo di riduzioni esistenti:
\begin{itemize}
    \item riduzioni di Turing: un problema DA FINIRE
    \item riduzione di Cook: riduzione di Turing rappresentata da una funzione calcolabile in tempo polinomiale.
\end{itemize}
La riducibilità da un linguaggio $L_A$ a un linguaggio $L_B$ viene indicata $L_A \le_{P} L_B$ perchè le riduzioni definiscono una \textit{relazione d'ordine parziale} sui problemi (e quindi sui linguaggi): il problema $L_A$ è sicuramente \textit{non più difficile} da risolvere del problema $L_B$, in quanto un algoritmo che risolve il problema definito da $L_B$ può essere usato per risolvere il problema definito da $L_A$.

Per le riduzioni vale anche la \textit{transitività}:
\begin{center}
    $L_A \le_P L_B \land L_B \le_P L_C \rightarrow L_A \le_P L_C$
\end{center}
Siano infatti $f_1$ la funzione che riduce $L_A$ in $L_B$ e $f_2$ la funzione che riduce $L_B$ in $L_C$, allora la funzione che riduce $L_A$ in $L_C$ è semplicemente la composizione di $f_1$ e $f_2$, ovvero $f_2 \circ f_1$. Inoltre, la composizione di due funzioni calcolabili in tempo polinomiale è a sua volta una funzione calcolabile in tempo polinomiale.

\begin{rem}
    L'esistenza di una riduzione $L_A \le_P L_B$ non implica l'esistenza della riduzione $L_B \le_P L_A$.
\end{rem}

\subsection*{NP-completezza}
\begin{defn}
    Un linguaggio $L$ è NP-hard sse $\forall L' \in \textnormal{NP}, L' \le_P L$, ovvero ogni linguaggio in NP può essere ridotto a $L$.
\end{defn}

Un problema NP-hard non è necessariamente un problema in NP: potrebbe infatti appartenere anche a classi di complessità maggiore, come EXPTIME. Teoricamente, un problema NP-hard potrebbe anche appartenere a P: se si riuscisse a dimostrarlo, allora P = NP.

\begin{defn}
    Un linguaggio $L$ è NP-completo sse $L \in \textnormal{NP} \land L \in \textnormal{NP-hard}$.
\end{defn}

I problemi NP-completi si possono quindi considerare come i problemi più difficili della classe NP.
Se si trovasse un algoritmo efficiente (almeno polinomiale) per un problema NP-completo, allora esso potrebbe essere applicato a tutti i problemi in NP, e di conseguenza P = NP.

\begin{thm}[di Cook-Levin]
    Il problema SAT è NP-completo.
\end{thm}
\begin{proof}
    Per dimostrare che un problema è NP-completo bisogna, per definizione, dimostrare che è in NP ed è NP-hard.
    SAT è in NP perchè esiste una macchina di Turing deterministica che può verificare, in tempo polinomiale, un suo certificato, anch'esso di lunghezza polinomiale.
    SAT è invece NP-hard perchè, come dimostrato da Cook e Levin, ogni ramo di computazione (o meglio la sequenza di configurazioni che lo definisce) di una macchina di Turing non deterministica può essere descritto usando una CNF-WFF.
    Essendo i problemi in NP risolvibili da una macchina non deterministica in tempo polinomiale, ed essendo la macchina non deterministica descrivibile tramite CNF-WFF, allora la computazione di ogni problema in NP può essere descritta tramite CNF-WFF.
    Di conseguenza, ogni problema in NP può essere ridotto al problema SAT.
\end{proof}

\begin{rem}
    Un problema $P_1 \in NP$ può essere dimostrato NP-completo sse un altro problema NP-completo $P_2$ può essere ridotto in tempo polinomiale a $P_1$.
\end{rem}

\subsection*{Relazioni sulle classi di complessità}
Alcune relazioni dimostrate sulle classi di complessità descritte nelle precedenti sezioni sono:
\begin{itemize}
    \item P = coP;
    \item $\textnormal{P} \subseteq \textnormal{NP}$;
    \item $\textnormal{NP-hard} \cap \textnormal{NP} = \textnormal{NP-complete} \neq \emptyset$;
\end{itemize}


\subsection{Esempi}
\subsection*{Il problema dell'arco minimo è in P}
Il problema dell'arco minimo è un problema di ricerca che, dato un grafo $G$, chiede di ritornare l'arco minimo di $G$, ovvero l'arco di peso minore.
Lo si può, però, trattare anche come un problema di decisione: in questo caso, il problema dell'arco minimo, dati in input un grafo $G$ e un arco $e$, chiede di stabilire se $e$ è l'arco minimo di $G$.
In realtà, il problema di decisione dell'arco minimo è solitamente formulato diversamente: dati un grafo $G$ e un valore $k$, esiste un arco $e$ con $w(e) \le k$?.
Per trovare il peso minimo all'interno del grafo basta risolvere più volte il problema, decrementando man mano il valore $k$. Se il problema afferma che ci sono archi di peso minore o uguale a $k$, ma non sono presenti archi di peso minore o uguale a $k-1$, allora $k$ è il peso minimo.
Una volta trovato il peso minimo, per trovare l'arco minimo basta risolvere più volte il problema modificando il peso di un singolo arco (rendendolo quello di peso maggiore): se la modifica varia la soluzione del problema sul peso minimo $k$, allora l'arco modificato è quello di peso minimo; altrimenti, l'arco modificato non è quello di peso minimo e bisogna quindi procedere con un'ulteriore modifica.

Se il grafo viene rappresentato con lista di adiacenza, allora la complessità dell'algoritmo che analizza tutti gli archi è $O(|E|)$. Se invece il grafo è rappresentato con matrice di adiacenza, allora l'algoritmo avrà complessità $O(|V|^2)$.
Di conseguenza, il problema appartiene a P.

Questo approccio può essere utilizzato anche per risolvere qualsiasi problema di ottimizzazione, ovvero di ricerca di un minimo o di un massimo.


\subsection*{Il problema di connettività di un grafo è in P}
Il problema di connettività di un grafo, dati un grafo $G$ e due vertici $s, t \in V$, chiede di stabilire se esiste un cammino da $s \leadsto t$.
È quindi un problema di decisione.
Il problema è risolvibile usando l'algoritmo DFS a partire da $s$.
Al termine dell'esecuzione dell'algoritmo, se $t$ appartiene alla stessa componente connessa di cui $s$ è radice, allora
$t$ è raggiungibile da $s$ ed esiste quindi $s \leadsto t$.
La complessità dell'algoritmo DFS è $O(|V| + |E|)$.
Di conseguenza, il problema è in P.

\subsection*{Il problema di primalità è in P}
Il problema di primalità è un problema di decisione che, dato un intero $k$, chiede di stabilire se $k$ è un numero primo.

L'approccio più ingenuo prevede di verificare che $d \nmid k, \forall d \in \mathbb{N}, 2 \le d \le \sqrt{k}$, per un totale di $\sqrt{k} - 1$ volte.
Si potrebbe quindi pensare che la sua complessità sia $O(k^{\frac{1}{2}})$. Ma la dimensione dell'input non è costante, ma $n = \lceil \log_2 k \rceil$.
Di conseguenza, $k = 2^n$ e la complessità sarà $O({(2^n)}^\frac{1}{2}) = O(2^n)$, e quindi esponenziale.

Un algoritmo migliore è AKS, che ha complessità polinomiale $O(n^12)$, e di conseguenza il problema di primalità è trattabile.
Negli ultimi anni si è riusciti ad abbassare ulteriormente la complessità, arrivando a $O(n^6)$.

\subsection*{UNSAT è in coNP}
UNSAT è il problema di decisione complementare rispetto a SAT.
Data una formula della logica proposizionale, UNSAT chiede di stabilire se la formula è insoddisfacibile, ovvero non esistono assegnamenti per cui risulta vera.

Si può sicuramente affermare che $\textnormal{UNSAT} \in \textnormal{coNP}$, in quanto il problema complementare di UNSAT, ovvero SAT, può essere risolto in tempo polinomiale da una macchina non deterministica.

Più difficile è invece affermare che $\textnormal{UNSAT} \in \textnormal{NP}$. Non si è, infatti, ancora riusciti a costruire un certificato, polinomiale nel numero di variabili $n$ della formula, che possa essere verificato in tempo polinomiale da una macchina deterministica: l'unico certificato al momento conosciuto per UNSAT è quello che contiene tutti i possibili assegnamenti da verificare, che è esponenziale in $n$ (i possibili assegnamenti su $n$ variabili sono $2^n$).

\subsection*{Traveling Salesman Problem è in NP}
La dimostrazione avviene sfruttando l'equivalenza tra decidibilità non deterministica in tempo polinomiale e verificabilità deterministica in tempo polinomiale.
TSP richiede di verificare, dato il grafo $G$ e un intero $k$, se esiste un ciclo hamiltoniano di peso minore di $k$.
Il relativo certificato altro non è che la sequenza $\langle v_1, \ldots, v_n \rangle$ dei nodi attraversati in un cammino del grafo $G$.
Di conseguenza, la dimensione del certificato è $O(|G|)$.

Bisogna poi dimostrare che il verificatore di certificati termini la sua esecuzione in tempo polinomiale.
In questo caso particolare, il verificatore deve:
\begin{enumerate}
    \item verificare che il certificato ricevuto rappresenti un ciclo hamiltoniano, ovvero tutti i vertici compaiano al suo interno una e una sola volta.
    \item verificare che la somma dei pesi degli archi che compongono il cammino sia minore di $k$.
\end{enumerate}
Entrambi i controlli possono essere effettuati in tempo polinomiale.
Di conseguenza, il verificatore ha tempo di esecuzione polinomiale in $|G|$.

Uno dei migliori algoritmi conosciuti per la risoluzione di TSP è basato sulla programmazione dinamica, ed ha complessità $O(2^{|V|} \cdot |V|^2)$.

\subsection*{SAT è in NP}
La dimostrazione avviene sfruttando l'equivalenza tra decidibilità non deterministica in tempo polinomiale e verificabilità deterministica in tempo polinomiale.
SAT richiede di stabilire se una data formula della logica proposizionale su $n$ variabili è soddisfacibile, ovvero esiste un assegnamento per cui risulta vera.
Il relativo certificato altro non è che una stringa binaria di lunghezza $n$, il cui bit di indice $i$ rappresenta l'assegnamento alla variabile $v_i$.
Di conseguenza, la dimensione del certificato è $O(n)$.

Bisogna poi dimostrare che il verificatore di certificati termini in tempo polinomiale. In questo caso particolare, il verificatore deve scorrere l'intera WFF e verificare che ogni clausola sia soddisfatta.
Ma il numero di possibili clausole è $2^k \binom{n}{k}$, con $k$ problema $k$-SAT a cui si sta facendo riferimento, ed è quindi $O(n^k)$.
Di conseguenza, il tempo richiesto dal verificatore per scorrere l'intera WFF è $O(n^k)$, e quindi polinomiale.

\section{Problemi di ottimizzazione}
\subsection{Algoritmi di approssimazione}
Un algoritmo di approssimazione è un particolare tipo di algoritmo che trova una soluzione sub-ottimale a un \textbf{problema di ottimizzazione} \footnote{Molto spesso, però, un problema di decisione può essere formulato come un problema di ottimizzazione, e vi si può quindi applicare un algoritmo di approssimazione. Vedi: problema Max SAT.}.
Sono usati sia per ottenere una soluzione approssimata per un problema NP-hard \footnote[][1cm]{I problemi NP-hard sono intrattabili, ovvero non sono conosciuti algoritmi che li risolvono in tempo polinomiale.} in tempo polinomiale, sia per trattare problemi per cui si conosce un algoritmo polinomiale, ma l'algoritmo risulta inapplicabile su istanze di grandi dimensioni.

Gli algoritmi di approssimazione si differenziano dagli algoritmi euristici: i primi, infatti, sono basati su limiti dimostrabili alla qualità della soluzione e ai tempi di esecuzione, mentre i secondi si basano su criteri non dimostrati e assunti veri.

\subsection*{Introduzione}
\begin{defn}
    Un \textbf{algoritmo di approssimazione} è un algoritmo efficiente che trova soluzioni approssimate a un problema di ottimizzazione, fornendo limiti dimostrabili alla qualità della soluzione ritornata.\\
    Con qualità della soluzione approssimata si intende la sua distanza dalla soluzione ottima.
\end{defn}

\marginnote{Non tutti i problemi di ottimizzazione ammettono algoritmi $f(n)$-approssimanti. Un esempio di problema non approssimabile è la ricerca dell'insieme indipendente in un grafo.}

\begin{defn}
    Un algoritmo di approssimazione per un problema $\Pi$ si dice \textbf{$f(n)$-approssimante}, con $f(n)$ funzione dipendente dalla dimensione dell'input $n$, se la soluzione $\appr(x)$ trovata dall'algoritmo dista dalla soluzione ottima $\opt(x)$ al massimo per un fattore moltiplicativo $f(n)$.
    $f(n)$ prende il nome di \textbf{rapporto di approssimazione} o garanzia di prestazione relativa.
    \[
    \begin{cases}
        \opt(x) \le \appr(x) \le f(n) \cdot \opt(x) & \tn{problemi di minimo}\\
        f(n) \cdot \opt(x) \le \appr(x) \le \opt(x) & \tn{problemi di massimo}
    \end{cases}
    \]
\end{defn}

\begin{rem}
    La $f(n)$-approssimazione non pone alcun vincolo sul tempo di esecuzione dell'algoritmo di approssimazione, ma solo sulla qualità della sua soluzione.\\
    Ad esempio, con algoritmo $2$-approssimante si può intendere sia un algoritmo polinomiale che un algoritmo esponenziale. 
\end{rem}

\subsection*{Classi di complessità di approssimazione}
I problemi di ottimizzazione approssimati, così come i problemi di decisione, possono essere organizzati in classi di complessità di approssimazione.

\begin{defn}
    \textbf{APX} è la classe dei problemi di ottimizzazione la cui versione di decisione appartiene a NP e che ammettono un algoritmo di approssimazione in tempo polinomiale, con rapporto di approssimazione costante.
\end{defn}

\begin{defn}
    Un problema di approssimazione $\Pi$ è un \textbf{Polynomial-Time Approxiamtion Scheme} (PTAS) sse per ogni rapporto di approssimazione costante $\epsilon > 1$, esiste un algoritmo $\epsilon$-approssimante che risolve $\Pi$ in tempo polinomiale.
\end{defn}

\begin{rem}
    Se un problema ammette un algoritmo $k$-approssimante in tempo polinomiale, allora ammette anche algoritmi $\epsilon$-approssimanti in tempo polinomiale, $\forall \epsilon > k$.
    L'algoritmo $\epsilon$-approssimante sarà di fatto l'algoritmo $k$-approssimante.
\end{rem}

\begin{defn}
    Un problema di approssimazione $\Pi$ è \textbf{APX-hard} se esiste una riduzione che preserva l'approssimazione \footnote{Vedere: Approxiamtion-preserving reduction, Wikipedia.} da qualsiasi problema in APX a $\Pi$.
\end{defn}

\begin{defn}
    Un problema è \textbf{APX-complete} se appartiene ad APX ed è APX-hard.\\
    Si possono considerare come i problemi più difficili da risolvere per la classe APX.
\end{defn}

\begin{rem}
    $\tn{PTAS} \subseteq \tn{APX}$\\
    Se $\tn{P} \neq \tn{NP}$, allora $\tn{PTAS} \subset \tn{APX}$, ovvero esistono problemi in APX che non sono PTAS, e $\tn{APX-complete} \neq \tn{APX} \setminus \tn{PTAS}$, ovvero esistono problemi in APX, ma non PTAS, che non sono APX-completi. Questi problemi prendono il nome di APX-intermediate.
\end{rem}

\subsection{Esempi}
\subsection*{Vertex Cover}
\subsubsection{Algoritmo $2$-approssimante per Vertex Cover}
L'algoritmo $2$-approssimante per il problema Vertex Cover è un algoritmo greedy che trova la copertura $C$ di un grafo $G$ calcolando il \textbf{matching massimale} \footnote{Il matching massimale di un grafo è l'insieme di archi senza vertici in comune che ha cardinalità massima.} di $G$. 

Per costruire il matching massimale, l'algoritmo considera tutti gli archi presenti in $E$. Estrae quindi un arco $(u,v)$ in maniera casuale, andando poi ad aggiungere i suoi estremi $u$ e $v$ al matching massimale. Rimuove poi da $E$ tutti gli archi cui almeno un estremo è $u$ o $v$. L'algoritmo itera fino a quando $E$ non è vuoto.

\begin{codebox}
    \Procname{$\proc{Vertex-Cover-Approx(V, E)}$}
    \li \While $E \neq $
    \End
\end{codebox}

\begin{thm}
    L'algoritmo \verb|Vertex-Cover-Approx| è $2$-approssimante, ovvero la copertura $C$ trovata dall'algoritmo di approssimazione contiene un numero di vertici $\appr(x) \le 2 \cdot \opt(x)$, con $\opt(x)$ numero di vertici contenuti nella copertura ottima.
\end{thm}

\begin{proof}
    DIMOSTRAZIONE
\end{proof}

\subsection*{Set Cover}
Il Set Cover Problem è un problema di ottimizzazione che chiede, dati un insieme universo $U$, una collezione di sottoinsiemi dell'insieme universo $S \subseteq \mathcal{P}(U)$ tale che $\bigcup\limits_{s \in S}^{\bullet} s = U$ e una funzione costo $\tn{c}: S \rightarrow \mathbb{Q}^+$, di trovare la set cover $C$ di $U$ di peso minimo, ovvero $C \subseteq S : \bigcup\limits_{c \in C}^{\bullet} c = U$ e il cui peso sia minimo.

Se il costo è uniforme, ovvero a ogni elemento di $S$ viene assegnato lo stesso costo, allora il problema chiede di trovare la set cover composta dal minor numero di elementi di $S$.

Si consideri ora il problema Set Cover nella sua versione di decisione.\\
\begin{problem}[lined]{Set Cover (di decisione)}
    Input: & \begin{minipage}[t]{0.8\linewidth}\begin{itemize}
        \setlength\itemsep{0em}
        \item un insieme universo $U$
        \item $S \subseteq \mathcal{P}(U)$ tale che $\bigcup\limits_{s \in S}^{\bullet} s = U$
        \item una funzione costo $\tn{c}: S \rightarrow \mathcal{Q}^+$
        \item $k \in Q^+$
    \end{itemize}\end{minipage}\\
    Output: & Esiste $C \subseteq S : \bigcup\limits_{e \in C}^{\bullet} e = U \land \sum\limits_{e \in C} \tn{c}(e) \le k$ ?
\end{problem}
\begin{thm}
    Il Set Cover Problem (di decisione) è un problema NP-completo.
\end{thm}
\begin{proof}
    La dimostrazione che Set Cover appartiene a NP si effettua dimostrando che un suo certificato può essere verificato in tempo polinomiale.\\
    Un certificato del problema è semplicemente una collezione di elementi di $S$. Il certificato cresce linearmente al crescere della cardinalità di $S$, e quindi ha dimensione polinomiale in funzione dell'input.\\
    La sua verifica avviene facilmente verificando che l'unione di tutti gli elementi di $S$ contenuti nel certificato sia l'insieme universo $U$ e che la somma dei loro costi sia minore o uguale al razionale positivo $k$ ricevuto in input. L'unione di due insiemi è polinomiale, e il confronto tra due numeri razionali è polinomiale. Di conseguenza, anche la verifica del certificato è polinomiale in funzione dell'input.

    Bisogna ora dimostrare che Set Cover è NP-hard. Per farlo basta trovare una riduzione da un problema NP-hard noto a Set Cover.\\
    Il problema NP-hard noto che si considera è Vertex Cover \footnote{Vertex Cover (di ottimizzazione) chiede, dato un grafo, di trovare un sottoinsieme di nodi tale che tutti gli archi del grafo abbiano almeno un estremo nel sottoinsieme di nodi scelto. Il problema si può facilmente riformulare come un problema di decisione.}. Un'istanza $(G = (E,V), j)$ di Vertex Cover può essere convertita in un'istanza $(U, S, \tn{c}, k)$ di Set Cover nel seguente modo:
    \begin{itemize}
        \item $U=E$ (avviene in tempo polinomiale, più precisamente costante);
        \item Supponendo di aver etichettato tutti i vertici in $V$ con $1 \le i \le |V|$, $S$ contiene tutti i sottoinsiemi $S_i \subseteq E : 1 \le i \le n$, dove $S_i$ contiene gli archi incidenti al vertice $i$ (avviene in tempo polinomiale, più precisamente $|V|^3$);
        \item la funzione di costo $\tn{c}$ viene definita come $\tn{c}(S_i) = 1$, in quanto considerare l'insieme $S_i$ significa prendere il vertice $i$ nella Vertex Cover del grafo $G$ (avviene in tempo polinomiale, più precisamente lineare in $|V|$);
        \item $k = j$ (avviene in tempo polinomiale, più precisamente costante).
    \end{itemize}
    
    L'istanza di Vertex Cover ha risposta positiva sse la relativa istanza di Set Cover ha risposta positiva.
    Infatti, se l'istanza di Vertex Cover ha una copertura $C$ con al più $j$ vertici, allora la copertura $C'$ per l'istanza di Set Cover è semplicemente quella formato dai sottoinsiemi $S_i, \forall i : i \in C$. Si può quindi affermare che $|C'| = |C|$. Ma siccome $|C| \le j \land k = j$, allora $|C'| \le k$.\\
    Viceversa, se l'istanza di Set Cover ammette una copertura $C', |C'| \le k$, allora l'istanza di Vertex Cover ammette copertura $C$ tale che $C = \cbra{i : 1 \le i \le |V| \land S_i \in C'}$, ovvero contiene tutti i vertici di indice $i$ per cui il relativo insieme di archi incidenti $S_i$ appartiene alla copertura $C'$. Di conseguenza, $|C| = |C'|$.
    Valendo inoltre $|C'| \le k$, e $j = k$, allora $|C| \le j$.

    Avendo dimostrato che Set Cover è in NP ed è NP-hard, allora Set Cover è un problema NP-completo.
\end{proof}

\subsection*{Travelling Salesman Problem simmetrico}
Il Travelling Salesman Problem simmetrico \footnote{Con simmetrico si intende che $\forall u, v \in V, d(u,v)=d(v,u)$, e quindi il grafo è non orientato.} (di ottimizzazione) chiede, dato un grafo non orientato pesato $G = (V, E, w)$, di trovare il ciclo Hamiltoniano di peso minore, se esiste.
\begin{problem}[lined]{Travelling Salesman Problem (di decisione)}
    Input: & \begin{minipage}[t]{0.8\linewidth}\begin{itemize}
        \setlength\itemsep{0em}
        \item un grafo $G = (V, E)$
        \item $k \in Q^+$
    \end{itemize}\end{minipage}\\
    Output: & Esiste un cammino hamiltoniano in $G$ di peso minore di $k$ ?
\end{problem}

Travelling Salesman Problem è un esempio di algoritmo senza un'approssimazione costante se $P \neq NP$: se TSP avesse un'approssimazione costante, allora Hamiltonian Path, un problema notoriamente NP completo, sarebbe risolvibile in tempo polinomiale e quindi $P = NP$, violando l'assunzione $P \neq NP$.

\begin{thm}
    Se $P \neq NP$, allora Travelling Salesman Problem (di ottimizzazione) non può avere un'approssimazione costante.
\end{thm}

\begin{proof}
    Si assuma esista un algoritmo $\epsilon$-approssimante per TSP e che $P \neq NP$.\\
    Sia $G = (V, E)$ un'istanza del problema Hamiltonian Path. Esistendo una riduzione Hamiltonian Path $\le_p$ Travelling Salesman Problem, è possibile costruire un'istanza di TSP a partire da $G$.
    Il grafo pesato $G' = (V', E', w)$ viene costruito come:
    \begin{itemize}
        \item $V' = V$
        \item $E' = V \times V$
        \item $\forall e \in E', \quad w(e) = \begin{cases}
            1 & e \in E\\
            \epsilon \cdot |V| + 1 & e \notin E
        \end{cases}$
    \end{itemize}
    Il valore ottimo $\opt(x)$ per l'algoritmo $\epsilon$-approssimante per TSP sarà $|V|$, mentre il valore della soluzione approssimata sarà, per definizione di algoritmo $\epsilon$-approssimante, minore di $\epsilon \cdot \opt(x)$.
    
    Nella risoluzione di TSP (di ottimizzazione) sul grafo $G'$ si possono verificare due casi:\\
    \textbf{Caso 1}: il cammino Hamiltoniano trovato dall'algoritmo $\epsilon$-approssimante include almeno uno degli archi di peso $\epsilon \cdot |V| + 1$. Di conseguenza, la soluzione approssimata di TSP ha costo $A(x) > \epsilon \cdot |V|$.\\
    Ma, per definizione di algoritmo $\epsilon$-approssimante, deve valere $|V| \le A(x) \le \epsilon \cdot |V|$. Di conseguenza, si ottiene una contraddizione, e non è possibile aver trovato un cammino Hamiltoniano che contiene un arco di peso $\epsilon \cdot |V| + 1$, ovvero uno degli archi aggiunti dalla riduzione.
    Di conseguenza, Hamiltonian Path sull'istanza $G$ non ammette cammino Hamiltoniano.
    
    \textbf{Caso 2}: il cammino Hamiltoniano trovato dall'algoritmo $\epsilon$-approssimante non include alcun arco di peso $\epsilon \cdot |V| + 1$, ma solo archi di peso $1$. Essendo il numero di archi in un cammino Hamiltoniano pari a $|V|$, allora il peso totale del cammino sarà $|V|$.\\
    Di conseguenza, vale $|V| \le A(x) \le \epsilon \cdot |V|$, in quanto $A(x) = |V|$.\\
    Ma essendo gli archi di peso $1$ archi che compaiono effettivamente nell'istanza di Hamiltonian Path, ed essendo il cammino Hamiltonanio trovato composto da soli archi di peso $1$, allora il cammino Hamiltoniano trovato con l'algoritmo $\epsilon$-approssimante di TSP è soluzione anche per Hamiltonian Path.

    Si può notare che l'algoritmo appena descritto si comporta di fatto come un algoritmo di decisione per il problema Hamiltonian Path, ed ha complessità polinomiale.
    Ma Hamiltonian Path è in NP e, quindi, si ottiene $P = NP$, violando l'ipotesi $P \neq NP$.
\end{proof}

\subsection*{Travelling Salesman Problem metrico}
Il Travelling Salesman Problem metrico è un tipo particolare di Travelling Salesman Problem in cui si considerano solo grafi simmetrici (e quindi non orientati) pesati in cui vale la disuguaglianza triangolare \footnote{Con disuguaglianza triangolare si intende $\forall u, v, z \in V, d(u,v) + d(v,z) \ge d(u,z)$}.

Travelling Salesman Problem metrico, al contrario del Travelling Salesman Problem simmetrico generico, ammette un algoritmo di approssimazione $\epsilon$-approssimante. Uno dei migliori algoritmi di approssimazione conosciuti per TSP metrico è l'algoritmo di Christofides, $\frac{3}{2}$-approssimante.
Esiste però un algoritmo più semplice e abbastanza intuitivo $2$-approssimante.

\subsubsection{Algoritmo $2$-approssimante per TSP metrico}
L'algoritmo $2$-approssimante per TSP metrico qui descritto prevede di costruire un Minimum Spanning Tree \footnote{Il Minimum Spanning Tree di un grafo è un albero che copre tutti i nodi del grafo ed ha costo minimo.} per il grafo, per poi costruire il ciclo Hamiltoniano a partire dal MST.

Innanzitutto bisogna trovare una relazione tra il costo del MST trovato, ovvero il suo peso, e il costo del ciclo Hamiltoniano ottimo del grafo $G$, ovvero il suo costo.
\begin{thm}
    Sia $T$ l'MST del grafo $G$, $c(T)$ il suo costo e $\opt(G)$ il costo del cammino Hamiltoniano ottimo di $G$, allora vale:
\end{thm}
\[
    c(T) \le \opt(G)
\]
\begin{proof}
    Se si considera un tour \footnote{Un tour è un ciclo che attraversa tutti i nodi del grafo, ma non è detto che lo faccia una e una sola volta per nodo. Il ciclo Hamiltoniano è un tipo particolare di tour, in cui ogni nodo è visitato esattamente una volta.} di $G$ e vi si rimuove un arco, allora si ottiene un albero di copertura, non necessariamente minimo (ovvero MST).

    Di conseguenza, supponendo che $T_1$ sia l'MST e $H$ sia il ciclo Hamiltoniano ottimo (che è un tour), togliendo un arco da $H$ si ottiene un nuovo albero di copertura $T_2$ di costo $c(T_2)$ e vale $c(T_2) \le c(H)$, in quanto si è rimosso un arco. Ma $T_2$ non è necessariamente albero di copertura minimo (lo è se $T_1 = T_2$, avendo assunto $T_1$ MST), e quindi vale $c(T_1) \le c(T_2)$.\\
    Di conseguenza vale $c(T_1) \le c(T_2) \le c(H)$ e quindi $c(T_1) \le c(H)$.
\end{proof}

L'MST del grafo viene costruito utilizzando l'algoritmo di Prim o l'algoritmo di Kruscal, entrambi di tempo polinomiale.\\
Una volta costruito l'MST $T$, allora $c(T) \le \opt(G)$ per il precedente teorema.

Bisogna ora trovare un modo di costruire un cammino Hamiltoniano $H$ a partire dal MST $T$ appena trovato.\\
Per farlo, si costruisce un grafo contenente tutti gli archi di $T$ duplicati, anche in questo caso in tempo polinomiale.
Si può notare che su questo grafo è possibile definire un ciclo euleriano \footnote{Un grafo $G$ ammette un ciclo euleriano sse tutti i nodi hanno grado pari.} $E$, ovvero un ciclo in cui si visitano tutti gli archi esattamente una volta. Il costo del ciclo euleriano $E$ sarà 
\[
    c(E) = 2 \cdot c(T)
\]

Il ciclo euleriano $E$ appena ottenuto non è però un ciclo Hamiltoniano. Per ottenere il ciclo Hamiltoniano $H$, che sarà il risultato dell'algoritmo di approssimazione, basta rimuovere dalla sequenza di nodi che descrive il ciclo euleriano i vertici già visitati, tranne l'ultimo elemento che chiude il ciclo.

\begin{thm}
    Sia $c(H)$ il costo del ciclo Hamiltoniano individuato dall'algoritmo di approssimazione e $c(E)$ il costo del ciclo euleriano da cui $H$ è stato generato.   \[
        c(H) \le c(E)
    \]
\end{thm}
\begin{proof}
    Nel TSP metrico, che è il caso di TSP che si sta considerando, vale la disuguaglianza triangolare.
    Se quindi, quando rimuovo nodi già visitati, sostituisco un cammino $u \leadsto v$ composto da più archi con l'arco che collega direttamente $u$ e $v$, allora il nuovo cammino avrà sicuramente costo inferiore, proprio perchè vale la disuguaglianza triangolare. 

    Di conseguenza, tutti i nodi già esplorati che vengono rimossi e la definizione di un collegamento diretto portano a un ciclo (Hamiltoniano) sicuramente di costo minore o uguale al ciclo euleriano.
\end{proof}

Sia quindi $H^*$ il ciclo Hamiltoniano ottimo, $H$ il ciclo Hamiltoniano trovato dall'algoritmo di approssimazione, $T$ l'MST e $E$ il ciclo euleriano definiti nel corso dell'esecuzione dell'algoritmo di approssimazione. Vale:
\[
    c(H) \le c(E) = 2 \cdot c(T) \le 2 \cdot c(H*^)
\]
Di conseguenza:
\[
    c(H^*) \le c(H) \le 2 \cdot c(H^*)
\]
Il rapporto di approssimazione di questo algoritmo di approssimazione è quindi costante e pari a $\epsilon = 2$.

\subsection*{Travelling Salesman Problem asimmetrico}

\subsection*{Travelling Purchaser Problem}

\subsection*{Consensus Sequence}
Consensus Sequence è il problema che chiede, data una collezione di stringhe, di trovare la stringa che minimizza la somma delle distanze di edit \footnote{La distanza di edit $d(s^{\prime}, s^{\prime\prime})$ è il numero di operazioni di inserimento, rimozione e sostituzione necessarie per ottenere $s^{\prime\prime}$ a partire da $s^{\prime}$.} da tutte le stringhe della collezione.

\begin{problem}[lined]{Consensus Sequence}
    Input: & \begin{minipage}[t]{0.8\linewidth}\begin{itemize}
        \setlength\itemsep{0em}
        \item una collezione di $k$ stringhe $S = \cbra{s_1, \ldots, s_k}$
    \end{itemize}\end{minipage}\\
    Output: & Stringa $s^*$ tale che $\sum\limits_{i=1}^k d(s^*, s_i)$ è minima.
\end{problem}

La stringa $s^*$ non deve necessariamente appartenere a $S$. 
\chapter{Algoritmi parametrici}
Un problema NP-hard è un problema difficile da risolvere nel caso generale. Si può osservare, però, non sempre tutte le istanze di un problema NP-hard sono difficili da risolvere, ma lo sono solo un sottoinsieme ristretto di istanze.\\
L'obiettivo di un algoritmo parametrico è quindi quello di "scartare" velocemente tutte le istanze facilmente risolvibili e di concentrarsi solo su quelle effettivamente importanti da risolvere.
Per farlo, un algoritmo parametrico introduce un parametro $k$ sull'input, e cerca di manipolare l'istanza di input in modo da poter definire un algoritmo con complessità $O(f(k) \cdot \tn{poly}(n))$.

\begin{defn}
    Un \textbf{problema parametrizzato} è un linguaggio $L \subseteq \Sigma^* \times \mathbb{N}$, dove $\Sigma$ è un alfabeto finito e l'intero $n \in \mathbb{N}$ è detto parametro.
\end{defn}

\begin{defn}
    Un problema parametrizzato è un \textbf{fixed-tractable problem} (FTP) sse esiste un algoritmo con complessità $O(f(k) \cdot \tn{poly}(n))$.

    $f(k)$ è una funzione che dipende \textit{solo} dal parametro $k$ e la cui crescita è arbitraria (può anche essere una funzione esponenziale o una funzione con velocità di crescita maggiore).
\end{defn}

I problemi parametrizzati sono quindi espressi in funzione di un'istanza e un parametro intero positivo. Data un'istanza, si vuole verificare se essa rispetta una proprietà che dipende da $k$.

\section{Esempi}
\subsection{Vertex Cover}
Il problema "classico" di Vertex Cover è un problema di ottimizzazione e chiede, dato un grafo, di trovare la minima Vertex Cover del grafo, dove con Vertex Cover si intende un insieme di nodi tale che ogni arco del grafo ha almeno un estremo in questo insieme di nodi.

Il problema di ottimizzazione può essere risolto in tempo esatto in tempo esponenziale, o può essere risolto in tempo polinomiale con un algoritmo $2$-approssimante.

Il problema può essere riformulato come un problema di decisione parametrizzato.
\begin{problem}[lined]{Vertex Cover (di decisione e parametrizzato)}
    Input: & \begin{minipage}[t]{0.8\linewidth}\begin{itemize}
        \setlength\itemsep{0em}
        \item un grafo $G = (V, E)$
        \item $k \in \mathbb{N}$ 
    \end{itemize}\end{minipage}\\
    Output: & Esiste $C \subseteq V : |C| \le k \land \forall (u, v) \in E, u \in C \lor v \in C$?
\end{problem}
Si vuole quindi verificare se esiste una Vertex Cover di cardinalità (numero di nodi) minore o uguale a $k$.

L'algoritmo più ingenuo per risolvere il problema prevede di enumerare tutti sottoinsiemi di nodi di dimensione minore o uguale a $k$, e verificare se almeno uno di essi è una Vertex Cover. 
La complessità è quindi $O(n^k \cdot T(n))$, dove $O(n^k)$ è il numero di sottoinsiemi di cardinalità minore o uguale a $k$ e $T(n)$ è il tempo necessario per verificare se un sottoinsieme è una Vertex Cover.

Si può notare che l'algoritmo appena proposto non è un algoritmo FPT, in quanto $f(k)$ non dipende solo da $k$ ma anche da $n$.

Un possibile algoritmo FTP può essere formulato tramite ricorsione ed è semplicemente un algoritmo a forza bruta.

$\tn{Vertex-Cover}(V, E, k)$:\\
\textbf{Caso base}: 
    \begin{minipage}[t]
    {0.8\linewidth}\begin{itemize}
        \setlength\itemsep{0em}
        \item $k = 0 \land |E| = 0 \longrightarrow \tn{YES}$
        \item $k = 0 \land |E| \neq 0 \longrightarrow \tn{NO}$
    \end{itemize}\end{minipage}
    
\textbf{Passo ricorsivo}:
    \[
        \tn{Vertex-Cover}(V \setminus \cbra{i}, E \setminus \cbra{i, j} \, \forall j \in V, k-1) \lor \tn{Vertex-Cover}(V \setminus \cbra{j}, E \setminus \cbra{i, j} \, \forall i \in V, k-1)
    \]

La complessità di questo algoritmo ricorsivo è $O(2^k \cdot n)$, ed è quindi un algoritmo FTP.

L'algoritmo ricorsivo può però essere ulteriormente migliorato facendo due considerazioni.
\begin{property}
    Se un nodo $v \in V$ ha grado $\tn{deg}(v) = 0$, allora esso non appartiene alla Vertex Cover.
\end{property}

\begin{property}
    Se un nodo $v \in V$ ha grado $\tn{deg}(v) > k$, allora esso appartiene sicuramente alla Vertex Cover.
\end{property}
Se infatti esso non appartenesse alla Vertex Cover, allora per 
poter coprire tutti gli archi incidenti sarebbe necessario 
inserire tutti i nodi adiacenti. Ma il numero di nodi adiacenti è 
strettamente maggiore di $k$, e non si avrebbe quindi una 
soluzione per l'istanza.

Ricevuto quindi un grafo in input, è possibile rimuovere tutti i nodi che rispettano una delle due condizioni, scartandoli se il loro grado è uguale a $0$ o inserendoli nella Vertex Cover se il loro grado è strettamente maggiore di $k$.\\
Il grafo che si ottiene ha sicuramente un numero massimo di nodi pari a $k$, e prende il nome di \textbf{kernel}.
Il kernel è la parte "difficile" da risolvere dell'istanza ricevuta in input.

Si può puoi applicare il precedente algoritmo ricorsivo a questo nuovo grafo, e la complessità della sua applicazione sarà $O(2^k \cdot k)$, in quanto $n = k$.

Considerando tutto l'algoritmo, la sua complessità è $O(n^2 + 2^k \cdot k)$, dove $O(n^2)$ è il tempo necessario alla rimozione dei nodi che rispettano una delle due proprietà e $O(2^k \cdot k)$ è il tempo dell'algoritmo ricorsivo sul sottografo.

Per Vertex Cover vale inoltre un'importante proprietà (qui non dimostrata):
\[
    |V| > k^2 + k \longrightarrow \tn{NO}
\]
Questo non significa che se $|V| \le k^2 + k$ allora il grafo ha una Vertex Cover di cardinalità minore o uguale a $k$. \upperAccE possibile che ci sia, ma non è assicurato: per stabilirlo è quindi necessario applicare l'algoritmo FTP precedentemente descritto.



\end{document}
