\documentclass[nobib, oneside, openany]{tufte-book}
% nobib     =   do not use package natbib, default in tufte-book.
%               it conflicts with biblatex.
% oneside   =   no blank page between chap and page do not move
%               left or right based on the page number.
% openany   =

% ------------------------------------
% |            GEOMETRY              |
% ------------------------------------
\setcounter{tocdepth}{2}
\setcounter{secnumdepth}{2}

\makeatletter
% Paragraph indentation and separation for normal text
\renewcommand{\@tufte@reset@par}{
    \setlength{\RaggedRightParindent}{0pt}
    \setlength{\JustifyingParindent}{0pt}
    \setlength{\parindent}{0pt}
    \setlength{\parskip}{4pt}
}
\@tufte@reset@par

% Paragraph indentation and separation for marginal text
\renewcommand{\@tufte@margin@par}{
    \setlength{\RaggedRightParindent}{0pt}
    \setlength{\JustifyingParindent}{0pt}
    \setlength{\parindent}{0pt}
    \setlength{\parskip}{4pt}
}


\makeatletter
\renewcommand{\maketitlepage}{
    \begingroup
        \setlength{\parindent}{0pt}
            {\fontsize{24}{24}\selectfont\textit{\@author}\par}
        \vspace{1.75in}
            {\fontsize{36}{54}\selectfont\@title\par}
        \vspace{0.5in}
            {\fontsize{14}{14}\selectfont\textsf{\smallcaps{\@date}}\par}
        \vfill{\fontsize{14}{14}\selectfont\textit{\@publisher}\par}
        \thispagestyle{empty}
    \endgroup
}
\makeatother

\geometry{
    %showframe,             % DEBUG ONLY -- displays the margins
	left=13mm,              % left margin
	textwidth=135mm,        % width of main text
	marginparsep=8mm,       % gutter between main text block and margin notes
	marginparwidth=50mm     % width of margin notes
}
\fontsize{10}{20}\selectfont

% chapter format
\titleformat{\chapter}
    {\huge\rmfamily\itshape\color{red}}             % format applied to label+text
    {\llap{\colorbox{red}{\parbox{1.5cm}{\hfill\itshape\huge\color{white}\thechapter}}}}
    {2pt}   % horizontal separation between label and title body
    {}      % before the title body
    []      % after the title body

% section format
\titleformat{\section}
    {\normalfont\Large\itshape\color{orange}}       % format applied to label+text
    {\llap{\colorbox{orange}{\parbox{1.5cm}{\hfill\color{white}\thesection}}}}
    {1em}   % horizontal separation between label and title body
    {}      % before the title body
    []      % after the title body

% subsection format
\titleformat{\subsection}
    {\normalfont\large\itshape\color{blue}}% format applied to label+text
    {\llap{\colorbox{blue}{\parbox{1.5cm}{\hfill\color{white}\thesubsection}}}}
    {1em}   % horizontal separation between label and title body
    {}      % before the title body
    []      % after the title body

% ------------------------------------
% |        GENERIC PACKAGES          |
% ------------------------------------
% Math
\usepackage{amsmath,amsthm,amssymb,amsfonts}
\usepackage{relsize, xfrac}

% Pseudocode
\usepackage{clrscode3e}         % Pseudocode as in "Introduction to algorithms"
\usepackage{minted}             % Code highlighting

% Hyperlinks in PDFs
\usepackage{hyperref}

% Lorem Ipsum text
\usepackage{lipsum}

% Images
\usepackage{graphicx}
\setkeys{Gin}{width=\linewidth,totalheight=\textheight,keepaspectratio}
\graphicspath{/images}

% Verb environment
\usepackage{fancyvrb}           % Customization of verb environments.
\fvset{fontsize=\normalsize}    % Use a smaller font for the verb environment.

% Language
\usepackage[italian]{babel}

% Bibliography
\usepackage{csquotes}
\usepackage[
    backend=biber,
    style=alphabetic,
    sorting=ynt
]{biblatex}
\addbibresource{bib/bibliography.bib}

% Filter out some warning
\usepackage{silence}
  \WarningFilter*{latex}
    {Marginpar on page \thepage\space moved}
  \WarningFilter{biblatex}          % ONLY IF BIBLATEX DOES NOT USE IBID STYLE REF.
    {Patching footnote}

% IDK
\usepackage{booktabs}               % Makes prettier tables.
\usepackage{multicol}               % Small sections of multiple columns in
                                    % documents and tables.

\usepackage{bookmark}
\usepackage[
    activate={true,nocompatibility},
    final,
    tracking=true,
    nopatch=footnote,
    kerning=true,
    spacing=false,
    factor=1100,
    stretch=10,
    shrink=10
]{microtype}

\usepackage{pgf,tikz,tikz-cd}

\usepackage{fancyhdr}               % Custom headers and footers.
\pagestyle{fancyplain}              % Makes all pages in the document conform
                                    % to the custom headers and footers.
\fancyhead{}                        % No page header - if you want one, create
                                    % it in the same way as the footers below.
\fancyfoot[L]{}                     % Empty left footer.
\fancyfoot[C]{}                     % Empty center footer.
\fancyfoot[R]{\thepage}             % Page numbering for right footer.
\renewcommand{\headrulewidth}{0pt}  % Remove header underlines.
\renewcommand{\footrulewidth}{0pt}  % Remove footer underlines.
\setlength{\headheight}{13.6pt}     % Customize the height of the header.
\allowdisplaybreaks

\usepackage{tabularx,environ} % For problem definition


% ------------------------------------
% |      THEOREM ENVIRONMENTS        |
% ------------------------------------
\newtheorem{thm}{Theorem}
\newtheorem{cor}[thm]{Corollary}
\newtheorem{prop}[thm]{Proposition}
\newtheorem{lem}[thm]{Lemma}
\newtheorem{conj}[thm]{Conjecture}
\newtheorem{quest}[thm]{Question}
\newtheorem{claim}{Claim}
\newtheorem{property}{Proprietà}

\theoremstyle{definition}
\newtheorem{defn}[thm]{Definition}
\newtheorem{defns}[thm]{Definitions}
\newtheorem{con}[thm]{Construction}
\newtheorem{exmp}[thm]{Example}
\newtheorem{exmps}[thm]{Examples}
\newtheorem{notn}[thm]{Notation}
\newtheorem{notns}[thm]{Notations}
\newtheorem{addm}[thm]{Addendum}
\newtheorem{exer}[thm]{Exercise}

\theoremstyle{remark}
\newtheorem{rem}[thm]{Osservazione}
\newtheorem{ans}[thm]{Answer}
\newtheorem{rems}[thm]{Remarks}
\newtheorem{warn}[thm]{Warning}
\newtheorem{sch}[thm]{Scholium}

% ------------------------------------
% |       COMMAND DEFINITIONS        |
% ------------------------------------
% Operators
\newcommand{\Mod}[1]{\ (\text{mod}\ #1)}
\newcommand{\N}{\mathbb{N}}
\newcommand{\Z}{\mathbb{Z}}
\newcommand{\Q}{\mathbb{Q}}
\newcommand{\R}{\mathbb{R}}
\newcommand{\C}{\mathbb{C}}
\newcommand{\OR}{\,|\,}
\newcommand{\AND}{\,\&\,}
\newcommand{\NOT}{!}

% Parenthesis
\newcommand{\bra}[1]{\left(#1\right)}
\newcommand{\sbra}[1]{\left[#1\right]}
\newcommand{\cbra}[1]{\left\{#1\right\}}
\newcommand{\norm}[1]{\left\lVert#1\right\rVert}
\newcommand{\ang}[1]{\left\langle#1\right\rangle}

% Shortened versions
\newcommand{\tn}[1]{\textnormal{#1}}

% Other
\DeclareMathOperator*{\argmin}{argmin}
\DeclareMathOperator*{\argmax}{argmax}
\newcommand{\upperAccE}{È \,}

% ------------------------------------
% |    DOCUMENT-SPECIFIC PACKAGES    |
% ------------------------------------

% ------------------------------------
% |    DOCUMENT-SPECIFIC COMMANDS    |
% ------------------------------------

% Abbreviations for pattern matching
\newcommand{\occ}{\tn{Occ}}
\newcommand{\bwt}{\tn{BWT}}
\newcommand{\lf}{\tn{LF}}

% Cost function for approx algorithms
\newcommand{\opt}{\tn{opt}}
\newcommand{\appr}{\tn{A}}

% ------------------------------------
% |  DOCUMENT-SPECIFIC ENVIRONMENTS  |
% ------------------------------------
\makeatletter           % https://tex.stackexchange.com/a/199244/26355
\newcolumntype{\expand}{}
\long\@namedef{NC@rewrite@\string\expand}{\expandafter\NC@find}
\NewEnviron{problem}[2][]{%
    \def\problem@arg{#1}%
    \def\problem@framed{framed}%
    \def\problem@lined{lined}%
    \def\problem@doublelined{doublelined}%
    \ifx\problem@arg\@empty%
        \def\problem@hline{}%
    \else%
        \ifx\problem@arg\problem@doublelined%
        \def\problem@hline{\hline\hline}%
        \else%
        \def\problem@hline{\hline}%
        \fi%
    \fi%
    \ifx\problem@arg\problem@framed%
        \def\problem@tablelayout{|>{\bfseries}lX|c}%
        \def\problem@title{\multicolumn{2}{|%
        >{\raisebox{-\fboxsep}}%
        p{\dimexpr\textwidth-4\fboxsep-2\arrayrulewidth\relax}%
        |}{%
            \textsc{\Large #2}%
        }}%
    \else
        \def\problem@tablelayout{>{\bfseries}lXc}%
        \def\problem@title{\multicolumn{2}{>%
        {\raisebox{-\fboxsep}}%
        p{\dimexpr\textwidth-4\fboxsep\relax}%
        }{%
            \textsc{\Large #2}%
        }}%
    \fi%
    \bigskip\par\noindent%
    \renewcommand{\arraystretch}{1.2}%
    \begin{tabularx}{\textwidth}{\expand\problem@tablelayout}%
        \problem@hline%
        \problem@title\\[2\fboxsep]%
        \BODY\\\problem@hline%
    \end{tabularx}%
    \medskip\par%
}
\makeatother


% ------------------------------------
% |         Main section             |
% ------------------------------------
\title{Teoria della\\Computazione}
\author{Mattia Bolognini}
\date{Ultima modifica: \today}

\begin{document}

\maketitle

\tableofcontents

\input{chap/teoria_computabilità}
\input{chap/teoria_complessità}
\section{Problemi di ottimizzazione}
\subsection{Algoritmi di approssimazione}
Un algoritmo di approssimazione è un particolare tipo di algoritmo che trova
una soluzione sub-ottimale a un \textbf{problema di ottimizzazione}
\footnote{Molto spesso, però, un problema di decisione può essere formulato
come un problema di ottimizzazione, e vi si può quindi applicare un algoritmo
di approssimazione. Vedi: problema Max SAT.}.
Sono usati sia per ottenere una soluzione approssimata per un problema NP-hard
\footnote[][1cm]{I problemi NP-hard sono intrattabili, ovvero non sono
conosciuti algoritmi che li risolvono in tempo polinomiale.} in tempo
polinomiale, sia per trattare problemi per cui si conosce un algoritmo
polinomiale, ma l'algoritmo risulta inapplicabile su istanze di grandi
dimensioni.

Gli algoritmi di approssimazione si differenziano dagli algoritmi euristici:
i primi, infatti, sono basati su limiti dimostrabili alla qualità della
soluzione e ai tempi di esecuzione, mentre i secondi si basano su criteri
non dimostrati e assunti veri.

\subsection*{Introduzione}
\begin{defn}
    Un \textbf{algoritmo di approssimazione} è un algoritmo efficiente che
    trova soluzioni approssimate a un problema di ottimizzazione, fornendo
    limiti dimostrabili alla qualità della soluzione ritornata.\\
    Con qualità della soluzione approssimata si intende la sua distanza dalla
    soluzione ottima.
\end{defn}

\marginnote{Non tutti i problemi di ottimizzazione ammettono algoritmi
$f(n)$-approssimanti. Un esempio di problema non approssimabile è la ricerca
dell'insieme indipendente in un grafo.}

\begin{defn}
    Un algoritmo di approssimazione per un problema $\Pi$ si dice
    \textbf{$f(n)$-approssimante}, con $f(n)$ funzione dipendente dalla
    dimensione dell'input $n$, se la soluzione $\appr(x)$ trovata
    dall'algoritmo dista dalla soluzione ottima $\opt(x)$ al massimo per un
    fattore moltiplicativo $f(n)$.
    $f(n)$ prende il nome di \textbf{rapporto di approssimazione} o garanzia
    di prestazione relativa.
    \[
    \begin{cases}
        \opt(x) \le \appr(x) \le f(n) \cdot \opt(x) & \tn{problemi di minimo}\\
        f(n) \cdot \opt(x) \le \appr(x) \le \opt(x) & \tn{problemi di massimo}
    \end{cases}
    \]
\end{defn}

\begin{rem}
    La $f(n)$-approssimazione non pone alcun vincolo sul tempo di esecuzione
    dell'algoritmo di approssimazione, ma solo sulla qualità della sua
    soluzione.\\
    Ad esempio, con algoritmo $2$-approssimante si può intendere sia un
    algoritmo polinomiale che un algoritmo esponenziale.
\end{rem}

\subsection*{Classi di complessità di approssimazione}
I problemi di ottimizzazione approssimati, così come i problemi di decisione,
possono essere organizzati in classi di complessità di approssimazione.

\begin{defn}
    \textbf{APX} è la classe dei problemi di ottimizzazione la cui versione
    di decisione appartiene a NP e che ammettono un algoritmo di
    approssimazione in tempo polinomiale, con rapporto di approssimazione
    costante.
\end{defn}

\begin{defn}
    Un problema di approssimazione $\Pi$ è un
    \textbf{Polynomial-Time Approxiamtion Scheme} (PTAS) sse per ogni rapporto
    di approssimazione costante $\epsilon > 1$, esiste un algoritmo
    $\epsilon$-approssimante che risolve $\Pi$ in tempo polinomiale.
\end{defn}

\begin{rem}
    Se un problema ammette un algoritmo $k$-approssimante in tempo polinomiale,
    allora ammette anche algoritmi $\epsilon$-approssimanti in tempo
    polinomiale, $\forall \epsilon > k$.
    L'algoritmo $\epsilon$-approssimante sarà di fatto l'algoritmo
    $k$-approssimante.
\end{rem}

\begin{defn}
    Un problema di approssimazione $\Pi$ è \textbf{APX-hard} se esiste una
    riduzione che preserva l'approssimazione
    \footnote{Vedere: Approxiamtion-preserving reduction, Wikipedia.}
    da qualsiasi problema in APX a $\Pi$.
\end{defn}

\begin{defn}
    Un problema è \textbf{APX-complete} se appartiene ad APX ed è APX-hard.\\
    Si possono considerare come i problemi più difficili da risolvere per la
    classe APX.
\end{defn}

\begin{rem}
    $\tn{PTAS} \subseteq \tn{APX}$\\
    Se $\tn{P} \neq \tn{NP}$, allora $\tn{PTAS} \subset \tn{APX}$, ovvero
    esistono problemi in APX che non sono PTAS, e
    $\tn{APX-complete} \neq \tn{APX} \setminus \tn{PTAS}$,
    ovvero esistono problemi in APX, ma non PTAS, che non sono APX-completi.
    Questi problemi prendono il nome di APX-intermediate.
\end{rem}

\subsection{Algoritmo $2$-approssimante per Vertex Cover}
L'algoritmo $2$-approssimante per il problema Vertex Cover è un algoritmo
greedy che trova la copertura $C$ di un grafo $G$ calcolando il
\textbf{matching massimale} \footnote{Il matching massimale di un grafo è
l'insieme di archi senza vertici in comune che ha cardinalità massima.} di $G$.

Per costruire il matching massimale, l'algoritmo considera tutti gli archi
presenti in $E$. Estrae quindi un arco $(u,v)$ in maniera casuale, andando
poi ad aggiungere i suoi estremi $u$ e $v$ al matching massimale.
Rimuove poi da $E$ tutti gli archi cui almeno un estremo è $u$ o $v$.
L'algoritmo itera fino a quando $E$ non è vuoto.

\begin{codebox}
    \Procname{$\proc{Vertex-Cover-Approx(V, E)}$}
    \li \While $E \neq $
    \End
\end{codebox}

\begin{thm}
    L'algoritmo \verb|Vertex-Cover-Approx| è $2$-approssimante, ovvero la
    copertura $C$ trovata dall'algoritmo di approssimazione contiene un
    numero di vertici $\appr(x) \le 2 \cdot \opt(x)$, con $\opt(x)$ numero
    di vertici contenuti nella copertura ottima.
\end{thm}

\begin{proof}
    DIMOSTRAZIONE
\end{proof}

\subsection{Set Cover è NP-completo}
Il Set Cover Problem è un problema di ottimizzazione che chiede, dati un
insieme universo $U$, una collezione di sottoinsiemi dell'insieme universo
$S \subseteq \mathcal{P}(U)$ tale che
$\bigcup\limits_{s \in S}^{\bullet} s = U$ e una funzione costo
$\tn{c}: S \rightarrow \mathbb{Q}^+$, di trovare la set cover $C$ di $U$ di
peso minimo, ovvero $C \subseteq S : \bigcup\limits_{c \in C}^{\bullet} c = U$
e il cui peso sia minimo.

Se il costo è uniforme, ovvero a ogni elemento di $S$ viene assegnato lo
stesso costo, allora il problema chiede di trovare la set cover composta dal
minor numero di elementi di $S$.

Si consideri ora il problema Set Cover nella sua versione di decisione.\\
\begin{problem}[lined]{Set Cover (di decisione)}
    Input: & \begin{minipage}[t]{0.8\linewidth}\begin{itemize}
        \setlength\itemsep{0em}
        \item un insieme universo $U$
        \item $S \subseteq \mathcal{P}(U)$ tale che
        $\bigcup\limits_{s \in S}^{\bullet} s = U$
        \item una funzione costo $\tn{c}: S \rightarrow \mathcal{Q}^+$
        \item $k \in Q^+$
    \end{itemize}\end{minipage}\\
    Output: & Esiste
    $C \subseteq S : \bigcup\limits_{e \in C}^{\bullet} e = U \land
    \sum\limits_{e \in C} \tn{c}(e) \le k$ ?
\end{problem}
\begin{thm}
    Il Set Cover Problem (di decisione) è un problema NP-completo.
\end{thm}
\begin{proof}
    La dimostrazione che Set Cover appartiene a NP si effettua dimostrando
    che un suo certificato può essere verificato in tempo polinomiale.\\
    Un certificato del problema è semplicemente una collezione di elementi
    di $S$. Il certificato cresce linearmente al crescere della cardinalità di
    $S$, e quindi ha dimensione polinomiale in funzione dell'input.\\
    La sua verifica avviene facilmente verificando che l'unione di tutti gli
    elementi di $S$ contenuti nel certificato sia l'insieme universo $U$ e che
    la somma dei loro costi sia minore o uguale al razionale positivo $k$
    ricevuto in input. L'unione di due insiemi è polinomiale, e il confronto
    tra due numeri razionali è polinomiale. Di conseguenza, anche la verifica
    del certificato è polinomiale in funzione dell'input.

    Bisogna ora dimostrare che Set Cover è NP-hard. Per farlo basta trovare
    una riduzione da un problema NP-hard noto a Set Cover.\\
    Il problema NP-hard noto che si considera è Vertex Cover
    \footnote{Vertex Cover (di ottimizzazione) chiede, dato un grafo,
    di trovare un sottoinsieme di nodi tale che tutti gli archi del grafo
    abbiano almeno un estremo nel sottoinsieme di nodi scelto. Il problema
    si può facilmente riformulare come un problema di decisione.}.
    Un'istanza $(G = (E,V), j)$ di Vertex Cover può essere convertita in
    un'istanza $(U, S, \tn{c}, k)$ di Set Cover nel seguente modo:
    \begin{itemize}
        \item $U=E$ (avviene in tempo polinomiale, più precisamente costante);
        \item Supponendo di aver etichettato tutti i vertici in $V$ con
        $1 \le i \le |V|$, $S$ contiene tutti i sottoinsiemi
        $S_i \subseteq E : 1 \le i \le n$, dove $S_i$ contiene gli archi
        incidenti al vertice $i$
        (avviene in tempo polinomiale, più precisamente $|V|^3$);
        \item la funzione di costo
        $\tn{c}$ viene definita come $\tn{c}(S_i) = 1$,
        in quanto considerare l'insieme $S_i$ significa prendere il
        vertice $i$ nella Vertex Cover del grafo $G$
        (avviene in tempo polinomiale, più precisamente lineare in $|V|$);
        \item $k = j$ (avviene in tempo polinomiale, più precisamente costante).
    \end{itemize}

    L'istanza di Vertex Cover ha risposta positiva sse la relativa istanza di
    Set Cover ha risposta positiva.
    Infatti, se l'istanza di Vertex Cover ha una copertura $C$ con al più
    $j$ vertici, allora la copertura $C'$ per l'istanza di Set Cover è
    semplicemente quella formato dai sottoinsiemi $S_i, \forall i : i \in C$.
    Si può quindi affermare che $|C'| = |C|$.
    Ma siccome $|C| \le j \land k = j$, allora $|C'| \le k$.\\
    Viceversa, se l'istanza di Set Cover ammette una copertura $C', |C'| \le k$,
    allora l'istanza di Vertex Cover ammette copertura $C$ tale che
    $C = \cbra{i : 1 \le i \le |V| \land S_i \in C'}$, ovvero contiene tutti
    i vertici di indice $i$ per cui il relativo insieme di archi incidenti
    $S_i$ appartiene alla copertura $C'$. Di conseguenza, $|C| = |C'|$.
    Valendo inoltre $|C'| \le k$, e $j = k$, allora $|C| \le j$.

    Avendo dimostrato che Set Cover è in NP ed è NP-hard, allora Set Cover è
    un problema NP-completo.
\end{proof}

\subsection{Inapprossimabilità Travelling Salesman Problem simmetrico}
Il Travelling Salesman Problem simmetrico \footnote{Con simmetrico si intende
che $\forall u, v \in V, d(u,v)=d(v,u)$, e quindi il grafo è non orientato.}
(di ottimizzazione) chiede, dato un grafo non orientato pesato $G = (V, E, w)$,
di trovare il ciclo Hamiltoniano di peso minore, se esiste.
\begin{problem}[lined]{Travelling Salesman Problem (di decisione)}
    Input: & \begin{minipage}[t]{0.8\linewidth}\begin{itemize}
        \setlength\itemsep{0em}
        \item un grafo $G = (V, E)$
        \item $k \in Q^+$
    \end{itemize}\end{minipage}\\
    Output: & Esiste un cammino hamiltoniano in $G$ di peso minore di $k$ ?
\end{problem}

Travelling Salesman Problem è un esempio di algoritmo senza un'approssimazione
costante se $P \neq NP$: se TSP avesse un'approssimazione costante,
allora Hamiltonian Path, un problema notoriamente NP completo, sarebbe
risolvibile in tempo polinomiale e quindi $P = NP$, violando l'assunzione
$P \neq NP$.

\begin{thm}
    Se $P \neq NP$, allora Travelling Salesman Problem (di ottimizzazione)
    non può avere un'approssimazione costante.
\end{thm}

\begin{proof}
    Si assuma esista un algoritmo $\epsilon$-approssimante per TSP e che
    $P \neq NP$.\\
    Sia $G = (V, E)$ un'istanza del problema Hamiltonian Path.
    Esistendo una riduzione Hamiltonian Path $\le_p$ Travelling Salesman
    Problem, è possibile costruire un'istanza di TSP a partire da $G$.
    Il grafo pesato $G' = (V', E', w)$ viene costruito come:
    \begin{itemize}
        \item $V' = V$
        \item $E' = V \times V$
        \item $\forall e \in E', \quad w(e) = \begin{cases}
            1 & e \in E\\
            \epsilon \cdot |V| + 1 & e \notin E
        \end{cases}$
    \end{itemize}
    Il valore ottimo $\opt(x)$ per l'algoritmo $\epsilon$-approssimante per
    TSP sarà $|V|$, mentre il valore della soluzione approssimata sarà,
    per definizione di algoritmo $\epsilon$-approssimante,
    minore di $\epsilon \cdot \opt(x)$.

    Nella risoluzione di TSP (di ottimizzazione) sul grafo $G'$ si possono
    verificare due casi:\\
    \textbf{Caso 1}: il cammino Hamiltoniano trovato dall'algoritmo
    $\epsilon$-approssimante include almeno uno degli archi di peso
    $\epsilon \cdot |V| + 1$.
    Di conseguenza, la soluzione approssimata di TSP ha costo
    $A(x) > \epsilon \cdot |V|$.\\
    Ma, per definizione di algoritmo $\epsilon$-approssimante,
    deve valere $|V| \le A(x) \le \epsilon \cdot |V|$.
    Di conseguenza, si ottiene una contraddizione, e non è possibile aver
    trovato un cammino Hamiltoniano che contiene un arco di peso
    $\epsilon \cdot |V| + 1$, ovvero uno degli archi aggiunti dalla riduzione.
    Di conseguenza, Hamiltonian Path sull'istanza $G$ non ammette cammino
    Hamiltoniano.

    \textbf{Caso 2}: il cammino Hamiltoniano trovato dall'algoritmo
    $\epsilon$-approssimante non include alcun arco di peso
    $\epsilon \cdot |V| + 1$, ma solo archi di peso $1$.
    Essendo il numero di archi in un cammino Hamiltoniano pari a $|V|$,
    allora il peso totale del cammino sarà $|V|$.\\
    Di conseguenza, vale $|V| \le A(x) \le \epsilon \cdot |V|$,
    in quanto $A(x) = |V|$.\\
    Ma essendo gli archi di peso $1$ archi che compaiono effettivamente
    nell'istanza di Hamiltonian Path, ed essendo il cammino Hamiltonanio
    trovato composto da soli archi di peso $1$, allora il cammino Hamiltoniano
    trovato con l'algoritmo $\epsilon$-approssimante di TSP è
    soluzione anche per Hamiltonian Path.

    Si può notare che l'algoritmo appena descritto si comporta di fatto come
    un algoritmo di decisione per il problema Hamiltonian Path,
    ed ha complessità polinomiale.
    Ma Hamiltonian Path è in NP e, quindi, si ottiene $P = NP$,
    violando l'ipotesi $P \neq NP$.
\end{proof}


\subsection{Algoritmo $2$-approssimante per TSP metrico}
Il Travelling Salesman Problem metrico è un tipo particolare di Travelling
Salesman Problem in cui si considerano solo grafi simmetrici
(e quindi non orientati) pesati in cui vale la disuguaglianza triangolare
\footnote{Con disuguaglianza triangolare si intende
$\forall u, v, z \in V, d(u,v) + d(v,z) \ge d(u,z)$}.

Travelling Salesman Problem metrico, al contrario del Travelling Salesman
Problem simmetrico generico, ammette un algoritmo di approssimazione
$\epsilon$-approssimante. Uno dei migliori algoritmi di approssimazione
conosciuti per TSP metrico è l'algoritmo di Christofides,
$\frac{3}{2}$-approssimante.
Esiste però un algoritmo più semplice e abbastanza intuitivo
$2$-approssimante.

L'algoritmo $2$-approssimante per TSP metrico qui descritto prevede di
costruire un Minimum Spanning Tree \footnote{Il Minimum Spanning Tree di
un grafo è un albero che copre tutti i nodi del grafo ed ha costo minimo.}
per il grafo, per poi costruire il ciclo Hamiltoniano a partire dal MST.

Innanzitutto bisogna trovare una relazione tra il costo del MST trovato,
ovvero il suo peso, e il costo del ciclo Hamiltoniano ottimo del grafo $G$,
ovvero il suo costo.
\begin{thm}
    Sia $T$ l'MST del grafo $G$, $c(T)$ il suo costo e $\opt(G)$ il costo
    del cammino Hamiltoniano ottimo di $G$, allora vale:
\end{thm}
\[
    c(T) \le \opt(G)
\]
\begin{proof}
    Se si considera un tour \footnote{Un tour è un ciclo che attraversa
    tutti i nodi del grafo, ma non è detto che lo faccia una e una sola
    volta per nodo. Il ciclo Hamiltoniano è un tipo particolare di tour,
    in cui ogni nodo è visitato esattamente una volta.} di $G$ e vi si
    rimuove un arco, allora si ottiene un albero di copertura, non
    necessariamente minimo (ovvero MST).

    Di conseguenza, supponendo che $T_1$ sia l'MST e $H$ sia il ciclo
    Hamiltoniano ottimo (che è un tour), togliendo un arco da $H$ si
    ottiene un nuovo albero di copertura $T_2$ di costo $c(T_2)$ e vale
    $c(T_2) \le c(H)$, in quanto si è rimosso un arco. Ma $T_2$ non è
    necessariamente albero di copertura minimo (lo è se $T_1 = T_2$,
    avendo assunto $T_1$ MST), e quindi vale $c(T_1) \le c(T_2)$.\\
    Di conseguenza vale $c(T_1) \le c(T_2) \le c(H)$ e quindi $c(T_1) \le c(H)$.
\end{proof}

L'MST del grafo viene costruito utilizzando l'algoritmo di Prim o
l'algoritmo di Kruscal, entrambi di tempo polinomiale.\\
Una volta costruito l'MST $T$, allora $c(T) \le \opt(G)$ per il precedente
teorema.

Bisogna ora trovare un modo di costruire un cammino Hamiltoniano $H$ a partire
dal MST $T$ appena trovato.\\
Per farlo, si costruisce un grafo contenente tutti gli archi di $T$ duplicati,
anche in questo caso in tempo polinomiale.
Si può notare che su questo grafo è possibile definire un ciclo euleriano
\footnote{Un grafo $G$ ammette un ciclo euleriano sse tutti i nodi
hanno grado pari.} $E$, ovvero un ciclo in cui si visitano tutti gli archi
esattamente una volta. Il costo del ciclo euleriano $E$ sarà
\[
    c(E) = 2 \cdot c(T)
\]

Il ciclo euleriano $E$ appena ottenuto non è però un ciclo Hamiltoniano.
Per ottenere il ciclo Hamiltoniano $H$, che sarà il risultato dell'algoritmo
di approssimazione, basta rimuovere dalla sequenza di nodi che descrive il
ciclo euleriano i vertici già visitati, tranne l'ultimo elemento che chiude
il ciclo.

\begin{thm}
    Sia $c(H)$ il costo del ciclo Hamiltoniano individuato dall'algoritmo
    di approssimazione e $c(E)$ il costo del ciclo euleriano da cui $H$ è
    stato generato.
    \[
        c(H) \le c(E)
    \]
\end{thm}
\begin{proof}
    Nel TSP metrico, che è il caso di TSP che si sta considerando,
    vale la disuguaglianza triangolare.
    Se quindi, quando rimuovo nodi già visitati, sostituisco un cammino
    $u \leadsto v$ composto da più archi con l'arco che collega direttamente
    $u$ e $v$, allora il nuovo cammino avrà sicuramente costo inferiore,
    proprio perchè vale la disuguaglianza triangolare.

    Di conseguenza, tutti i nodi già esplorati che vengono rimossi e la
    definizione di un collegamento diretto portano a un ciclo (Hamiltoniano)
    sicuramente di costo minore o uguale al ciclo euleriano.
\end{proof}

Sia quindi $H^*$ il ciclo Hamiltoniano ottimo, $H$ il ciclo Hamiltoniano
trovato dall'algoritmo di approssimazione, $T$ l'MST e $E$ il ciclo euleriano
definiti nel corso dell'esecuzione dell'algoritmo di approssimazione. Vale:
\[
    c(H) \le c(E) = 2 \cdot c(T) \le 2 \cdot c(H^*)
\]
Di conseguenza:
\[
    c(H^*) \le c(H) \le 2 \cdot c(H^*)
\]
Il rapporto di approssimazione di questo algoritmo di approssimazione è
quindi costante e pari a $\epsilon = 2$.

\subsection{Algoritmo $2-\frac{1}{k}$-approssimante per Closest String}
Closest String è il problema che chiede, data una collezione di stringhe,
di trovare la stringa $s^*$ che minimizza la somma delle distanze di edit
\footnote{La distanza di edit $d(s^{\prime}, s^{\prime\prime})$ è
il numero di operazioni di inserimento, rimozione e sostituzione necessarie
per ottenere $s^{\prime\prime}$ a partire da $s^{\prime}$.} da tutte le
stringhe della collezione.
La stringa $s^*$ non deve necessariamente appartenere a $S$.

\begin{problem}[lined]{Closest String (di ottimizzazione)}
    Input: & una collezione di $k$ stringhe $S = \cbra{s_1, \ldots, s_k}$\\
    Output: & Stringa $s^*$ tale che $\sum\limits_{i=1}^k d(s^*, s_i)$ è minima.
\end{problem}

\begin{problem}[lined]{Closest String (di decisione e parametrizzato)}
    Input: & \begin{minipage}[t]{0.8\linewidth}\begin{itemize}
        \setlength\itemsep{0em}
        \item una collezione di $k$ stringhe $S = \cbra{s_1, \ldots, s_k}$
        \item un parametro $c$
    \end{itemize}\end{minipage}\\
    Output: & $\exists s^* \in \Sigma^*$ tale che
    $\sum\limits_{i=1}^k d(s^*, s_i) \le c$?
\end{problem}

Il problema della Closest String nella versione di decisione è NP-completo.

Closest String è un problema difficile da risolvere. Può quindi
essere utile trovare un algoritmo che trovano una soluzione sub-ottimale
in tempo ridotto (almeno polinomiale).
Il problema ammette
un algoritmo $2-\frac{1}{k}$-approssimante, ovvero un
algoritmo il cui output è una stringa la cui somma delle distanze di edit,
nel caso peggiore, è pari a $2-\frac{1}{k}$ volte il valore
ottimale dato dalla stringa ottima, non necessariamente appartenente alla
collezione di stringhe,
con $k$ numero di stringhe nella collezione.\\
Si può notare, quindi, che, all'aumentare delle stringhe nella collezione,
l'errore di approssimazione si riduce.\\
Questo algoritmo non fa altro che restituire, come soluzione approssimata,
la stringa appartenente alla collezione $S$ ricevuta in input che minimizza
la somma delle distanze di edit.

Per dimostrare effettivamente la qualità della soluzione, ovvero che
l'algoritmo individuato è $2-\frac{1}{k}$-approssimante, bisogna trovare
un lower bound e un upper bound al costo della stella definito dalla stringa
individuata dall'algoritmo.\\
Si può affermare che il costo della soluzione restituita, ovvero
la somma delle distanze di edit, è sicuramente maggiore o uguale
al costo della soluzione ottima, ovvero la stringa che minimizza la
somma delle distanze di edit, proprio per definizione di soluzione ottima.\\
Per quanto riguarda il limite superiore al costo della soluzione, invece,
la dimostrazione sfrutta il fatto che la distanza di edit sia una metrica,
ovvero che essa sia:
\begin{itemize}
    \item riflessiva, ovvero $d(s, s) = 0$
    \item simmetrica, ovvero $d(s, s') = d(s', s)$
    \item vale la disuguaglianza triangolare, ovvero
    \[
    d(s_i,s_j) + d(s_j,s_k) \ge d(s_i,s_k), \forall s_i,s_j,s_k \in \Sigma^*
    \]
\end{itemize}

Valendo la disuguaglianza triangolare, si può affermare che la somma delle distanze
di una stringa da tutte le altre stringhe della collezione è sicuramente
minore o uguale alla somma delle distanze di edit della stringa ottimale da
tutte le stringhe della collezione più $k-1$ la distanza di edit tra la stringa
ottimale e la stringa scelta.
Sia $c$ la funzione che, data una stringa $s \in \Sigma^*$, restituisce il costo
della stringa, ovvero la somma delle distanze di edit di $s$ da tutte le stringhe
della collezione $S$. Allora:
\[
    c(s_i) \le c(s^*) + (k-1)d(s^*, s_i) \quad \forall s_i \in S
\]

Ma allora vale anche che la somma di tutti i costi delle singole $k$ stringhe della
collezione è minore o uguale a $k$ volte il costo della stringa ottimale più
$(k-1)$ volte il costo della stringa ottimale.
\[
    c(s_1)+\ldots+c(s_k) \le
    \underbrace{\bra{c(s^*) + (k-1)d(s^*, s_1)} +
                \ldots +
                \bra{c(s^*) + (k-1)d(s^*, s_k)}}
            _{k \text{ volte}}
\]
che è uguale a:
\[
    c(s_1)+\ldots+c(s_k) \le k \cdot c(s^*) + (k-1) \cdot c(s^*) = (2k-1) \cdot c(s^*)
\]
Sia $s_j$ la stringa restituita dall'algoritmo approssimante, ovvero la stringa
della collezione che minimizza la somma delle distanze.
Ma la somma di tutti i $k$ costi delle $k$ stringhe della collezione è sicuramente
maggiore di $k$ volte il costo minore tra tutti i costi delle stringhe della collezione,
ovvero il costo di $s_j$.
\[
    k \cdot c(s_j) \le c(s_1)+\ldots+c(s_k) \le (2k-1) \cdot c(s^*)
\]
Di conseguenza:
\[
    k \cdot c(s_j) \le (2k-1) \cdot c(s^*)
\]
Ma è già stato detto che $c(s^*) \le c(s_j)$, e di conseguenza
$k \cdot c(s^*) \le k \cdot c(s_j)$. Vale allora:
\[
    k \cdot c(s^*) \le k \cdot c(s_j) \le (2k-1) \cdot c(s^*)
\]
che equivale a:
\[
    c(s^*) \le c(s_j) \le (2-\frac{1}{k}) \cdot c(s^*)
\]
Si è quindi dimostrato che l'algoritmo fornito è $(2-\frac{1}{k})$-approssimante.

\chapter{Algoritmi parametrici}
Un problema NP-hard è un problema difficile da risolvere nel caso generale. Si può osservare, però, non sempre tutte le istanze di un problema NP-hard sono difficili da risolvere, ma lo sono solo un sottoinsieme ristretto di istanze.\\
L'obiettivo di un algoritmo parametrico è quindi quello di "scartare" velocemente tutte le istanze facilmente risolvibili e di concentrarsi solo su quelle effettivamente importanti da risolvere.
Per farlo, un algoritmo parametrico introduce un parametro $k$ sull'input, e cerca di manipolare l'istanza di input in modo da poter definire un algoritmo con complessità $O(f(k) \cdot \tn{poly}(n))$.

\begin{defn}
    Un \textbf{problema parametrizzato} è un linguaggio $L \subseteq \Sigma^* \times \mathbb{N}$, dove $\Sigma$ è un alfabeto finito e l'intero $n \in \mathbb{N}$ è detto parametro.
\end{defn}

\begin{defn}
    Un problema parametrizzato è un \textbf{fixed-tractable problem} (FTP) sse esiste un algoritmo con complessità $O(f(k) \cdot \tn{poly}(n))$.

    $f(k)$ è una funzione che dipende \textit{solo} dal parametro $k$ e la cui crescita è arbitraria (può anche essere una funzione esponenziale o una funzione con velocità di crescita maggiore).
\end{defn}

I problemi parametrizzati sono quindi espressi in funzione di un'istanza e un parametro intero positivo. Data un'istanza, si vuole verificare se essa rispetta una proprietà che dipende da $k$.

\section{Esempi}
\subsection{Vertex Cover}
Il problema "classico" di Vertex Cover è un problema di ottimizzazione e chiede, dato un grafo, di trovare la minima Vertex Cover del grafo, dove con Vertex Cover si intende un insieme di nodi tale che ogni arco del grafo ha almeno un estremo in questo insieme di nodi.

Il problema di ottimizzazione può essere risolto in tempo esatto in tempo esponenziale, o può essere risolto in tempo polinomiale con un algoritmo $2$-approssimante.

Il problema può essere riformulato come un problema di decisione parametrizzato.
\begin{problem}[lined]{Vertex Cover (di decisione e parametrizzato)}
    Input: & \begin{minipage}[t]{0.8\linewidth}\begin{itemize}
        \setlength\itemsep{0em}
        \item un grafo $G = (V, E)$
        \item $k \in \mathbb{N}$ 
    \end{itemize}\end{minipage}\\
    Output: & Esiste $C \subseteq V : |C| \le k \land \forall (u, v) \in E, u \in C \lor v \in C$?
\end{problem}
Si vuole quindi verificare se esiste una Vertex Cover di cardinalità (numero di nodi) minore o uguale a $k$.

L'algoritmo più ingenuo per risolvere il problema prevede di enumerare tutti sottoinsiemi di nodi di dimensione minore o uguale a $k$, e verificare se almeno uno di essi è una Vertex Cover. 
La complessità è quindi $O(n^k \cdot T(n))$, dove $O(n^k)$ è il numero di sottoinsiemi di cardinalità minore o uguale a $k$ e $T(n)$ è il tempo necessario per verificare se un sottoinsieme è una Vertex Cover.

Si può notare che l'algoritmo appena proposto non è un algoritmo FPT, in quanto $f(k)$ non dipende solo da $k$ ma anche da $n$.

Un possibile algoritmo FTP può essere formulato tramite ricorsione ed è semplicemente un algoritmo a forza bruta.

$\tn{Vertex-Cover}(V, E, k)$:\\
\textbf{Caso base}: 
    \begin{minipage}[t]
    {0.8\linewidth}\begin{itemize}
        \setlength\itemsep{0em}
        \item $k = 0 \land |E| = 0 \longrightarrow \tn{YES}$
        \item $k = 0 \land |E| \neq 0 \longrightarrow \tn{NO}$
    \end{itemize}\end{minipage}
    
\textbf{Passo ricorsivo}:
    \[
        \tn{Vertex-Cover}(V \setminus \cbra{i}, E \setminus \cbra{i, j} \, \forall j \in V, k-1) \lor \tn{Vertex-Cover}(V \setminus \cbra{j}, E \setminus \cbra{i, j} \, \forall i \in V, k-1)
    \]

La complessità di questo algoritmo ricorsivo è $O(2^k \cdot n)$, ed è quindi un algoritmo FTP.

L'algoritmo ricorsivo può però essere ulteriormente migliorato facendo due considerazioni.
\begin{property}
    Se un nodo $v \in V$ ha grado $\tn{deg}(v) = 0$, allora esso non appartiene alla Vertex Cover.
\end{property}

\begin{property}
    Se un nodo $v \in V$ ha grado $\tn{deg}(v) > k$, allora esso appartiene sicuramente alla Vertex Cover.
\end{property}
Se infatti esso non appartenesse alla Vertex Cover, allora per 
poter coprire tutti gli archi incidenti sarebbe necessario 
inserire tutti i nodi adiacenti. Ma il numero di nodi adiacenti è 
strettamente maggiore di $k$, e non si avrebbe quindi una 
soluzione per l'istanza.

Ricevuto quindi un grafo in input, è possibile rimuovere tutti i nodi che rispettano una delle due condizioni, scartandoli se il loro grado è uguale a $0$ o inserendoli nella Vertex Cover se il loro grado è strettamente maggiore di $k$.\\
Il grafo che si ottiene ha sicuramente un numero massimo di nodi pari a $k$, e prende il nome di \textbf{kernel}.
Il kernel è la parte "difficile" da risolvere dell'istanza ricevuta in input.

Si può puoi applicare il precedente algoritmo ricorsivo a questo nuovo grafo, e la complessità della sua applicazione sarà $O(2^k \cdot k)$, in quanto $n = k$.

Considerando tutto l'algoritmo, la sua complessità è $O(n^2 + 2^k \cdot k)$, dove $O(n^2)$ è il tempo necessario alla rimozione dei nodi che rispettano una delle due proprietà e $O(2^k \cdot k)$ è il tempo dell'algoritmo ricorsivo sul sottografo.

Per Vertex Cover vale inoltre un'importante proprietà (qui non dimostrata):
\[
    |V| > k^2 + k \longrightarrow \tn{NO}
\]
Questo non significa che se $|V| \le k^2 + k$ allora il grafo ha una Vertex Cover di cardinalità minore o uguale a $k$. \upperAccE possibile che ci sia, ma non è assicurato: per stabilirlo è quindi necessario applicare l'algoritmo FTP precedentemente descritto.



\end{document}
