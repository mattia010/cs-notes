\chapter{Teoria della complessità computazionale}
\section{Problemi di decisione}
\subsection*{Classi di complessità}
I problemi di decisione, o più precisamente i linguaggi, possono essere classificati in base alla loro complessità computazionale. Un problema viene classificato in base all'algoritmo di complessità più bassa tra tutti gli algoritmi di risoluzione conosciuti per il problema.
La classificazione non indica, però, che il problema non può avere un algoritmo migliore: la complessità migliore fornisce quindi solo un upper-bound alla complessità del miglior algoritmo possibile.

\begin{thm}
    La classe $\textnormal{PTIME} \left( f(n) \right)$ è la classe che contiene tutti i problemi di decisione risolvibili in tempo $O(f(n))$ da una macchina di Turing deterministica.
    \begin{center}
        $\textnormal{DTIME} \left( f(n) \right) = \left\{ L \subseteq \Sigma^* \; | \; L \textnormal{ è deciso da una DTM in tempo } O \left( f(n) \right) \right\}$
    \end{center}
\end{thm}

\begin{rem}
    La classe dei problemi decidibili in tempo polinomiale su una macchina di Turing deterministica è:
    \begin{center}
        $\textnormal{P} = \left\{ \bigcup\limits_{c \ge 0} \textnormal{DTIME}(n^c) \right\}$
    \end{center}
\end{rem}

\begin{thm}
    La classe $\textnormal{NTIME} \left( f(n) \right)$ è la classe che contiene tutti i problemi di decisione risolvibili in tempo $O(f(n))$ da una macchina di Turing non deterministica.
    \begin{center}
        $\textnormal{NTIME} \left( f(n) \right) = \left\{ L \subseteq \Sigma^* \; | \; L \textnormal{ è deciso da una NDTM in tempo } O \left( f(n) \right) \right\}$
    \end{center}
\end{thm}

\begin{rem}
    La classe dei problemi decidibili in tempo polinomiale su una macchina di Turing non deterministica è:
    \begin{center}
        $\textnormal{NP} = \left\{ \bigcup\limits_{c \ge 0} \textnormal{NTIME}(n^c) \right\}$
    \end{center}
\end{rem}

\begin{rem}
    Vale la relazione $\textnormal{P} \subseteq \textnormal{NP}$, in quanto ogni macchina deterministica può essere simulata in tempo polinomiale da una macchina non deterministica usando un solo ramo di computazione.
\end{rem}

È possibile definire due ulteriori classi, coP e coNP, che contengono i problemi complementari dei problemi di decisione contenuti rispettivamente in P e NP. Con problema complementare $\overline{L}$ si intende quel problema definito allo stesso modo del problema $L$, ma in cui le risposte fornite sono invertite: $w \in L \rightarrow w \notin \overline{L}$ e $w \in \overline{L} \rightarrow w \notin L$. 
\begin{center}
    coP = $\left\{ L \subseteq \Sigma^* \; | \; \overline{L} \in P \right\}$\\
    coNP = $\left\{ L \subseteq \Sigma^* \; | \; \overline{L} \in NP \right\}$
\end{center}

È importante notare che coP $\neq \overline{P}$: $\overline{P}$ contiene infatti tutti i problemi che non sono in P (quindi anche quelli non decidibili), mentre coP contiene tutti i problemi complementari ai problemi in P. Il concetto è quello di applicare l'operatore di complemento ai singoli problemi (linguaggi) in P, non a tutto l'insieme P.
Lo stesso ragionamento può essere applicato a coNP $\neq$ NP.

\begin{thm}
    \textnormal{P = coP}
\end{thm}
\begin{proof}
    Sia un linguaggio $\textnormal{L} \in \textnormal{P}$, allora esiste, ovviamente, una macchina di Turing deterministica $M$ che decide $L$ in tempo $O(n^k)$.
    Si può costruire un'altra macchina di Turing deterministica $M'$ che, dato un input $x$, restituisce YES sse $M(x) = \textnormal{NO}$ e restituisce NO sse $M(x) = \textnormal{YES}$. Per farlo, serve semplicemente invertire gli stati di accettazione e di non accettazione che compongono la FSM di $M$.

    Siccome $M$ ha complessità $O(n^k)$, anche $M'$ avrà complessità $O(n^k)$.
    Di conseguenza, $\overline{L} \in \textnormal{P}$ per qualunque linguaggio $L$, e quindi P = coP.
\end{proof}

\begin{rem}
    Non è possibile applicare lo stesso ragionamento usato per dimostrare P = coP anche per NP = coNP.
\end{rem}
\begin{proof}
    La computazione di una macchina di Turing non deterministica può essere rappresentata come un albero, in cui ognuno dei rami porta a YES o NO (si assume quindi che tutti i rami terminino).
    INSERIRE IMMAGINE
    Invertire gli stati di accettazione della macchina, come fatto per P = coP, significa, di fatto, invertire i risultati forniti dai rami di esecuzione.
    INSERIRE IMMAGINE
    Si può notare, però, che una stringa appartenente al linguaggio deciso dalla macchina non deterministica continuerà ad appartenere al linguaggio anche invertendo gli stati di accettazione (e quindi i risultati forniti dai diversi rami): di conseguenza, non si è ancora riusciti a costruire una macchina di Turing non deterministica che accetti il linguaggio complementare di un linguaggio in NP.

    Se si dimostrasse che coNP = NP, allora si sarebbe dimostrato P = NP.
\end{proof}


\subsection*{Equivalenza tra decidibilità non det. in tempo $O(n^k)$ e verificabilità det. in tempo $O(n^k)$}
Per dimostrare che un linguaggio appartiene a NP, si potrebbe pensare di dimostrare che esso può essere deciso in tempo polinomiale da una macchina di Turing non deterministica. Può però risultare complicato costruire una macchina di questo tipo, e per questo, spesso, si preferisce dimostrare l'appartenenza a NP dimostrando che una stringa sull'alfabeto può essere verificata (ovvero si può affermare se appartiene o meno al linguaggio) da una macchina di Turing deterministica, detta \textit{verificatore}, in tempo polinomiale, usando una stringa aggiuntiva detta \textit{certificato}.

\begin{thm}
    Sia $L$ un linguaggio,
    \begin{center}
        $\textnormal{L} \in \textnormal{NP} \leftrightarrow \exists a \in \mathbb{N} \land V \textnormal{ DTM}$ tali che $\forall x \in \Sigma^*, \exists w \in \Sigma^{|x|^a}\; : \; V(x, w) = Y$
    \end{center}
    dove $w$ è detta certificato e ha lunghezza polinomiale in funzione della dimensione dell'input $|x|$.
\end{thm}
Un certificato indica, di fatto, quale dei possibili rami di esecuzione della macchina di Turing non deterministica, che riconosce $L$, deve essere valutato dal verificatore.

\begin{proof}
    In una macchina non deterministica che lavora in tempo polinomiale, un singolo ramo di esecuzione richiede ovviamente tempo polinomiale.
    Un certificato specifica quale ramo di esecuzione considerare, specificando quale sottoalbero analizzare quando ci si trova ad un certo nodo dell'albero.
    Essendo però il tempo di esecuzione di un ramo di esecuzione della macchina non deterministica limitato polinomialmente nella dimensione dell'input, allora anche il numero di nodi dell'albero attraversati dal ramo sarà limitato polinomialmente nella dimensione dell'input (non è possibile infatti che lo spazio cresca più velocemente del tempo, altrimenti non si riuscirebbe ad esplorare tutto lo spazio occupato).
    Di conseguenza, il certificato, che descrive un singolo ramo di esecuzione, avrà dimensione polinomiale nella dimensione dell'input.
    Verificare un certificato significa verificare il risultato a cui porta il ramo di esecuzione della macchina non deterministica da lui descritto.
    Ma, essendo il certificato di dimensione polinomiale, il ramo può essere simulato, e quindi verificato, da una macchina di Turing deterministica in tempo polinomiale, che prende il nome di verificatore.

    Viceversa, se esiste un verificatore, ovvero una macchina di Turing deterministica che simula, in tempo polinomiale, un certo ramo di esecuzione descritto da un certificato, allora è possibile costruire la macchina di Turing non deterministica che decide, in tempo polinomiale, il linguaggio $L$ considerato.
    Basta infatti generare tutti i possibili certificati $w \in \Sigma^*$. Dato un input $x$, quindi, la sua appartenenza al linguaggio può essere verificata generando tutti i possibili certificati, e applicando il verificatore a ogni possibile coppia $x$-certificato: se il verificatore restituisce un risultato positivo per almeno un certificato, allora $x \in L$; se invece nessun certificato porta a YES, allora $x \notin L$.
    Ma queste operazioni di verifica di tutti i possibili certificati possono essere effettuate parallelamente da una macchina di Turing non deterministica, qualsiasi sia l'input $x$.
    Di conseguenza, la macchina non deterministica che decide, in tempo polinomiale, il linguaggio $L$ è quella che esegue, per un certo ramo di esecuzione, il verificatore applicato a uno dei certificati.
\end{proof}

\subsection*{Riduzioni}
\begin{defn}
    Un linguaggio $L_A$ è riducibile polinomialmente a un linguaggio $L_B$, indicato con $L_A \le_{P} L_B$, sse esiste una funzione $f: \Sigma^* \rightarrow \Sigma^*$ calcolabile in tempo polinomiale da una macchina di Turing deterministica tale che $\forall w \in \Sigma^*, w \in L_A \leftrightarrow f(w) \in L_B$. Questo tipo di riduzioni prende il nome di riduzioni di Karp.
\end{defn}
Le riduzioni di Karp non sono l'unico tipo di riduzioni esistenti:
\begin{itemize}
    \item riduzioni di Turing: un problema DA FINIRE 
    \item riduzione di Cook: riduzione di Turing rappresentata da una funzione calcolabile in tempo polinomiale.
\end{itemize}
La riducibilità da un linguaggio $L_A$ a un linguaggio $L_B$ viene indicata $L_A \le_{P} L_B$ perchè le riduzioni definiscono una \textit{relazione d'ordine parziale} sui problemi (e quindi sui linguaggi): il problema $L_A$ è sicuramente \textit{non più difficile} da risolvere del problema $L_B$, in quanto un algoritmo che risolve il problema definito da $L_B$ può essere usato per risolvere il problema definito da $L_A$.

Per le riduzioni vale anche la \textit{transitività}: 
\begin{center}
    $L_A \le_P L_B \land L_B \le_P L_C \rightarrow L_A \le_P L_C$
\end{center}
Siano infatti $f_1$ la funzione che riduce $L_A$ in $L_B$ e $f_2$ la funzione che riduce $L_B$ in $L_C$, allora la funzione che riduce $L_A$ in $L_C$ è semplicemente la composizione di $f_1$ e $f_2$, ovvero $f_2 \circ f_1$. Inoltre, la composizione di due funzioni calcolabili in tempo polinomiale è a sua volta una funzione calcolabile in tempo polinomiale.

\begin{rem}
    L'esistenza di una riduzione $L_A \le_P L_B$ non implica l'esistenza della riduzione $L_B \le_P L_A$.
\end{rem}

\subsection*{NP-completezza}
\begin{defn}
    Un linguaggio $L$ è NP-hard sse $\forall L' \in \textnormal{NP}, L' \le_P L$, ovvero ogni linguaggio in NP può essere ridotto a $L$.
\end{defn}

Un problema NP-hard non è necessariamente un problema in NP: potrebbe infatti appartenere anche a classi di complessità maggiore, come EXPTIME. Teoricamente, un problema NP-hard potrebbe anche appartenere a P: se si riuscisse a dimostrarlo, allora P = NP.

\begin{defn}
    Un linguaggio $L$ è NP-completo sse $L \in \textnormal{NP} \land L \in \textnormal{NP-hard}$.
\end{defn}

I problemi NP-completi si possono quindi considerare come i problemi più difficili della classe NP.
Se si trovasse un algoritmo efficiente (almeno polinomiale) per un problema NP-completo, allora esso potrebbe essere applicato a tutti i problemi in NP, e di conseguenza P = NP.

\begin{thm}[di Cook-Levin]
    Il problema SAT è NP-completo.    
\end{thm}
\begin{proof}
    Per dimostrare che un problema è NP-completo bisogna, per definizione, dimostrare che è in NP ed è NP-hard.
    SAT è in NP perchè esiste una macchina di Turing deterministica che può verificare, in tempo polinomiale, un suo certificato, anch'esso di lunghezza polinomiale.
    SAT è invece NP-hard perchè, come dimostrato da Cook e Levin, ogni ramo di computazione (o meglio la sequenza di configurazioni che lo definisce) di una macchina di Turing non deterministica può essere descritto usando una CNF-WFF.
    Essendo i problemi in NP risolvibili da una macchina non deterministica in tempo polinomiale, ed essendo la macchina non deterministica descrivibile tramite CNF-WFF, allora la computazione di ogni problema in NP può essere descritta tramite CNF-WFF.
    Di conseguenza, ogni problema in NP può essere ridotto al problema SAT.
\end{proof}

\begin{rem}
    Un problema $P_1 \in NP$ può essere dimostrato NP-completo sse un altro problema NP-completo $P_2$ può essere ridotto in tempo polinomiale a $P_1$.
\end{rem}

\subsection*{Relazioni sulle classi di complessità}
Alcune relazioni dimostrate sulle classi di complessità descritte nelle precedenti sezioni sono:
\begin{itemize}
    \item P = coP;
    \item $\textnormal{P} \subseteq \textnormal{NP}$;
    \item $\textnormal{NP-hard} \cap \textnormal{NP} = \textnormal{NP-complete} \neq \emptyset$;
\end{itemize}


\subsection{Esempi}
\subsection{Il problema dell'arco minimo è in P}
Il problema dell'arco minimo è un problema di ricerca che, dato un grafo $G$, chiede di ritornare l'arco minimo di $G$, ovvero l'arco di peso minore.
Lo si può, però, trattare anche come un problema di decisione: in questo caso, il problema dell'arco minimo, dati in input un grafo $G$ e un arco $e$, chiede di stabilire se $e$ è l'arco minimo di $G$. 
In realtà, il problema di decisione dell'arco minimo è solitamente formulato diversamente: dati un grafo $G$ e un valore $k$, esiste un arco $e$ con $w(e) \le k$?.
Per trovare il peso minimo all'interno del grafo basta risolvere più volte il problema, decrementando man mano il valore $k$. Se il problema afferma che ci sono archi di peso minore o uguale a $k$, ma non sono presenti archi di peso minore o uguale a $k-1$, allora $k$ è il peso minimo.
Una volta trovato il peso minimo, per trovare l'arco minimo basta risolvere più volte il problema modificando il peso di un singolo arco (rendendolo quello di peso maggiore): se la modifica varia la soluzione del problema sul peso minimo $k$, allora l'arco modificato è quello di peso minimo; altrimenti, l'arco modificato non è quello di peso minimo e bisogna quindi procedere con un'ulteriore modifica.

Se il grafo viene rappresentato con lista di adiacenza, allora la complessità dell'algoritmo che analizza tutti gli archi è $O(|E|)$. Se invece il grafo è rappresentato con matrice di adiacenza, allora l'algoritmo avrà complessità $O(|V|^2)$.
Di conseguenza, il problema appartiene a P.

Questo approccio può essere utilizzato anche per risolvere qualsiasi problema di ottimizzazione, ovvero di ricerca di un minimo o di un massimo.


\subsection{Il problema di connettività di un grafo è in P}
Il problema di connettività di un grafo, dati un grafo $G$ e due vertici $s, t \in V$, chiede di stabilire se esiste un cammino da $s \leadsto t$.
È quindi un problema di decisione.
Il problema è risolvibile usando l'algoritmo DFS a partire da $s$.
Al termine dell'esecuzione dell'algoritmo, se $t$ appartiene alla stessa componente connessa di cui $s$ è radice, allora 
$t$ è raggiungibile da $s$ ed esiste quindi $s \leadsto t$.
La complessità dell'algoritmo DFS è $O(|V| + |E|)$.
Di conseguenza, il problema è in P.

\subsection{Il problema di primalità è in P}
Il problema di primalità è un problema di decisione che, dato un intero $k$, chiede di stabilire se $k$ è un numero primo.

L'approccio più ingenuo prevede di verificare che $d \nmid k, \forall d \in \mathbb{N}, 2 \le d \le \sqrt{k}$, per un totale di $\sqrt{k} - 1$ volte.
Si potrebbe quindi pensare che la sua complessità sia $O(k^{\frac{1}{2}})$. Ma la dimensione dell'input non è costante, ma $n = \lceil \log_2 k \rceil$.
Di conseguenza, $k = 2^n$ e la complessità sarà $O({(2^n)}^\frac{1}{2}) = O(2^n)$, e quindi esponenziale.

Un algoritmo migliore è AKS, che ha complessità polinomiale $O(n^12)$, e di conseguenza il problema di primalità è trattabile.
Negli ultimi anni si è riusciti ad abbassare ulteriormente la complessità, arrivando a $O(n^6)$.

\subsection{UNSAT è in coNP}
UNSAT è il problema di decisione complementare rispetto a SAT.
Data una formula della logica proposizionale, UNSAT chiede di stabilire se la formula è insoddisfacibile, ovvero non esistono assegnamenti per cui risulta vera.

Si può sicuramente affermare che $\textnormal{UNSAT} \in \textnormal{coNP}$, in quanto il problema complementare di UNSAT, ovvero SAT, può essere risolto in tempo polinomiale da una macchina non deterministica.

Più difficile è invece affermare che $\textnormal{UNSAT} \in \textnormal{NP}$. Non si è, infatti, ancora riusciti a costruire un certificato, polinomiale nel numero di variabili $n$ della formula, che possa essere verificato in tempo polinomiale da una macchina deterministica: l'unico certificato al momento conosciuto per UNSAT è quello che contiene tutti i possibili assegnamenti da verificare, che è esponenziale in $n$ (i possibili assegnamenti su $n$ variabili sono $2^n$).

\subsection{Traveling Salesman Problem è in NP}
La dimostrazione avviene sfruttando l'equivalenza tra decidibilità non deterministica in tempo polinomiale e verificabilità deterministica in tempo polinomiale.
TSP richiede di verificare, dato il grafo $G$ e un intero $k$, se esiste un ciclo hamiltoniano di peso minore di $k$.
Il relativo certificato altro non è che la sequenza $\langle v_1, \ldots, v_n \rangle$ dei nodi attraversati in un cammino del grafo $G$.
Di conseguenza, la dimensione del certificato è $O(|G|)$.

Bisogna poi dimostrare che il verificatore di certificati termini la sua esecuzione in tempo polinomiale.
In questo caso particolare, il verificatore deve:
\begin{enumerate}
    \item verificare che il certificato ricevuto rappresenti un ciclo hamiltoniano, ovvero tutti i vertici compaiano al suo interno una e una sola volta.
    \item verificare che la somma dei pesi degli archi che compongono il cammino sia minore di $k$.
\end{enumerate}
Entrambi i controlli possono essere effettuati in tempo polinomiale.
Di conseguenza, il verificatore ha tempo di esecuzione polinomiale in $|G|$.

Uno dei migliori algoritmi conosciuti per la risoluzione di TSP è basato sulla programmazione dinamica, ed ha complessità $O(2^{|V|} \cdot |V|^2)$.

\subsection{SAT è in NP}
La dimostrazione avviene sfruttando l'equivalenza tra decidibilità non deterministica in tempo polinomiale e verificabilità deterministica in tempo polinomiale.
SAT richiede di stabilire se una data formula della logica proposizionale su $n$ variabili è soddisfacibile, ovvero esiste un assegnamento per cui risulta vera.
Il relativo certificato altro non è che una stringa binaria di lunghezza $n$, il cui bit di indice $i$ rappresenta l'assegnamento alla variabile $v_i$.
Di conseguenza, la dimensione del certificato è $O(n)$.

Bisogna poi dimostrare che il verificatore di certificati termini in tempo polinomiale. In questo caso particolare, il verificatore deve scorrere l'intera WFF e verificare che ogni clausola sia soddisfatta.
Ma il numero di possibili clausole è $2^k \binom{n}{k}$, con $k$ problema $k$-SAT a cui si sta facendo riferimento, ed è quindi $O(n^k)$.
Di conseguenza, il tempo richiesto dal verificatore per scorrere l'intera WFF è $O(n^k)$, e quindi polinomiale.