\section{Algoritmi parametrici}
Un problema NP-hard è un problema difficile da risolvere: affrontare quindi
un problema di questo tipo usando un algoritmo a forza bruta risulta
complicato.
Alcuni problemi NP-hard, però, ammettono un particolare tipo di algoritmi,
appunto gli algoritmi parametrici, che permettono di risolvere un sottoinsieme
di istanze del problema in tempo trattabile, quando un certo parametro $k$
è fissato e assume piccoli valori. Questo parametro esprime solitamente
un vincolo che si vuole imporre alla soluzione del problema (es. Esiste
una Vertex Cover del grafo $G$ con un numero di vertici minore di $k$?).
Di conseguenza, l'applicazione di un algoritmo parametrico, per alcuni
problemi e in alcuni casi, può portare a dei vantaggi.

Un algoritmo parametrico permette quindi di esprimere la complessità di un
algoritmo non solo in funzione della dimensione dell'input, ma anche del
valore assunto dal parametro $k$.
Questo permette una migliore classificazione dei problemi NP-hard: alcuni
di essi, infatti, ammettono un algoritmo FPT e, quindi, sono facilmente
risolvibili quando il parametro è fissato e di piccoli valori, mentre
altri problemi NP-hard non ammettono alcun algoritmo FPT, e tutte le loro
istanze saranno difficili da risolvere.

A differenza degli algoritmi approssimanti, gli algoritmi parametrici
restituiscono sempre una soluzione esatta al problema.

\begin{defn}
    Un \textbf{problema parametrizzato} è un linguaggio
    $L \subseteq \Sigma^* \times \mathbb{N}$, dove $\Sigma$ è un alfabeto
    finito e l'intero $n \in \mathbb{N}$ è detto parametro.
\end{defn}

\begin{defn}
    Un problema parametrizzato è un \textbf{fixed-parameter tractable}
    sse esiste un algoritmo con complessità $O(f(k) \cdot \tn{poly}(n))$.

    $f(k)$ è una funzione che dipende \textit{solo} dal parametro $k$ e la
    cui crescita è arbitraria (può anche essere una funzione esponenziale o
    una funzione con velocità di crescita maggiore).
\end{defn}

I problemi parametrizzati sono quindi espressi in funzione di un'istanza e
un parametro intero positivo. Data un'istanza, si vuole verificare se essa
rispetta una proprietà che dipende da $k$.

\subsection{Esempi di algoritmi parametrici}
\subsection*{Algoritmo parametrico per Vertex Cover}
Il problema "classico" di Vertex Cover è un problema di ottimizzazione e
chiede, dato un grafo, di trovare la minima Vertex Cover del grafo, dove con
Vertex Cover si intende un insieme di nodi tale che ogni arco del grafo ha
almeno un estremo in questo insieme di nodi.

Il problema di ottimizzazione può essere risolto in tempo esatto in tempo
esponenziale, o può essere risolto in tempo polinomiale con un algoritmo
$2$-approssimante.

Il problema può essere riformulato come un problema di decisione
parametrizzato.
\begin{problem}[lined]{Vertex Cover (di decisione e parametrizzato)}
    Input: & \begin{minipage}[t]{0.8\linewidth}\begin{itemize}
        \setlength\itemsep{0em}
        \item un grafo $G = (V, E)$
        \item $k \in \mathbb{N}$
    \end{itemize}\end{minipage}\\
    Output: & Esiste $C \subseteq V : |C| \le k \land
    \forall (u, v) \in E, u \in C \lor v \in C$?
\end{problem}
Si vuole quindi verificare se esiste una Vertex Cover di cardinalità
(numero di nodi) minore o uguale a $k$.

L'algoritmo più ingenuo per risolvere il problema prevede di enumerare
tutti sottoinsiemi di nodi di dimensione minore o uguale a $k$, e
verificare se almeno uno di essi è una Vertex Cover.
La complessità è quindi $O(n^k \cdot T(n))$, dove $O(n^k)$ è il numero
di sottoinsiemi di cardinalità minore o uguale a $k$ e $T(n)$ è il tempo
necessario per verificare se un sottoinsieme è una Vertex Cover.

Si può notare che l'algoritmo appena proposto non è un algoritmo FPT,
in quanto $f(k)$ non dipende solo da $k$ ma anche da $n$.

Un possibile algoritmo FPT può essere formulato tramite ricorsione ed è
semplicemente un algoritmo a forza bruta.

$\tn{Vertex-Cover}(V, E, k)$:\\
\textbf{Caso base}:
    \begin{minipage}[t]
    {0.8\linewidth}\begin{itemize}
        \setlength\itemsep{0em}
        \item $k = 0 \land |E| = 0 \longrightarrow \tn{YES}$
        \item $k = 0 \land |E| \neq 0 \longrightarrow \tn{NO}$
    \end{itemize}\end{minipage}

\textbf{Passo ricorsivo}:
    \[
        \tn{Vertex-Cover}(V \setminus \cbra{i}, E \setminus \cbra{i, j}
        \, \forall j \in V, k-1) \lor
        \tn{Vertex-Cover}(V \setminus \cbra{j}, E \setminus \cbra{i, j}
        \, \forall i \in V, k-1)
    \]

La complessità di questo algoritmo ricorsivo è $O(2^k \cdot n)$, ed è
quindi un algoritmo FTP.

L'algoritmo ricorsivo può però essere ulteriormente migliorato facendo
due considerazioni.
\begin{property}
    Se un nodo $v \in V$ ha grado $\tn{deg}(v) = 0$, allora esso non
    appartiene alla Vertex Cover.
\end{property}

\begin{property}
    Se un nodo $v \in V$ ha grado $\tn{deg}(v) > k$, allora esso
    appartiene sicuramente alla Vertex Cover.
\end{property}
Se infatti esso non appartenesse alla Vertex Cover, allora per
poter coprire tutti gli archi incidenti sarebbe necessario
inserire tutti i nodi adiacenti. Ma il numero di nodi adiacenti è
strettamente maggiore di $k$, e non si avrebbe quindi una
soluzione per l'istanza.

Ricevuto quindi un grafo in input, è possibile rimuovere tutti i nodi
che rispettano una delle due condizioni, scartandoli se il loro grado
è uguale a $0$ o inserendoli nella Vertex Cover se il loro grado è
strettamente maggiore di $k$.\\
Il grafo che si ottiene ha sicuramente un numero massimo di nodi pari a
$k$, e prende il nome di \textbf{kernel}.
Il kernel è la parte "difficile" da risolvere dell'istanza ricevuta in
input.

Si può puoi applicare il precedente algoritmo ricorsivo a questo nuovo
grafo, e la complessità della sua applicazione sarà $O(2^k \cdot k)$, in
quanto $n = k$.

Considerando tutto l'algoritmo, la sua complessità è $O(n^2 + 2^k \cdot k)$,
dove $O(n^2)$ è il tempo necessario alla rimozione dei nodi che rispettano
una delle due proprietà e $O(2^k \cdot k)$ è il tempo dell'algoritmo
ricorsivo sul sottografo.

Per Vertex Cover vale inoltre un'importante proprietà (qui non dimostrata):
\[
    |V| > k^2 + k \longrightarrow \tn{NO}
\]
Questo non significa che se $|V| \le k^2 + k$ allora il grafo ha una Vertex
Cover di cardinalità minore o uguale a $k$. \upperAccE possibile che ci sia,
ma non è assicurato: per stabilirlo è quindi necessario applicare
l'algoritmo FTP precedentemente descritto.


\subsection*{Algoritmo parametrico per Closest String}
Closest String è quel problema che chiede, dato un insieme di stringhe
di uguale lunghezza $S = \cbra{s_1, \ldots, s_k}$, di trovare la stringa
$s^* \in \Sigma^*$ che minimizza la somma delle distanze di Hamming di $s^*$
da tutte le stringhe della collezione.

Il problema può, però, essere riformulato come un problema di decisione
parametrizzato: dato un insieme di $k$ stringhe di uguale lunghezza
$S = \cbra{s_1, \ldots, s_k}$ e un parametro $d$, si vuole stabilire
se esiste una stringa
$s^* \in \Sigma^*$ tale che $\forall 1 \le i \le k, d_H(s^*, s_i) \le d$,
ovvero la stringa $s^*$ ha distanza di Hamming \footnote{La distanza
di Hamming è una metrica.} minore o uguale a $d$ da tutte le
stringhe della collezione.

\begin{problem}[lined]{Consensus Sequence (di decisione e parametrizzato)}
    Input: & \begin{minipage}[t]{0.8\linewidth}\begin{itemize}
        \setlength\itemsep{0em}
        \item una collezione di $k$ stringhe $S = \cbra{s_1, \ldots, s_k}$,
        tutte di uguale lunghezza
        \item un parametro $d$
    \end{itemize}\end{minipage}\\
    Output: & $\exists s^* \in \Sigma^*$ tale che
    $\forall 1 \le i \le k, d_H(s^*, s_i) \le d$?
\end{problem}

L'algoritmo parametrico per risolvere Closest String si pone come obiettivo
quello di trovare un kernel di un'istanza, ovvero la parte dell'istanza
effettivamente difficile da risolvere.\\
Per prima cosa, tutte le stringhe della collezione vengono poste in una matrice
$k \times n$, con $k$ numero di stringhe e $n$ lunghezza delle stringhe.
Su una riga della matrice, di conseguenza, comparirà una delle parole della
collezione $S$.
Una volta costruita la matrice, è possibile rimuovere tutte le colonne clean,
ovvero tutte le colonne che presentano lo stesso simbolo in tutte le loro celle.
Questo viene fatto in quanto quel simbolo sicuramente dovrà comparire nella
stringa di output nella posizione associata alla colonna clean,
e non è quindi necessario che l'algoritmo valuti anche
questa colonna nel corso della sua esecuzione.
Tutte le altre colonne, che presentano almeno due simboli differenti,
sono definite colonne dirty.

\begin{thm}
    Se il numero di colonne dirty è maggiore di $k \cdot d$, il problema non
    ammette una Closest String. Di conseguenza, la risposta al problema
    di decisione parametrizzato sarà negativa.
\end{thm}

All'inizio dell'algoritmo, inoltre, si verifica se una delle stringhe
della collezione stessa ricevuta in input è soluzione del problema di
decisione parametrizzato.
Se, infatti, una stringa della collezione è già soluzione,
allora si può facilmente ritornare una risposta positiva.\\
Se invece nessuna
stringa della collezione è una soluzione valida, allora l'algoritmo procede,
costruendo la matrice e valutando colonne clean e dirty.

Si supponga che $s^*$ sia la Closest String per la collezione $S$, dato il
parametro $d$. Si può affermare, per definizione di Closest String, che:
\[
    d_H(s^*, s_i) \le d \land d_H(s^*, s_j) \le d
    \quad \forall 1 \le i, j \le k
\]
Ma la distanza di Hamming è una metrica, e quindi vale la disuguaglianza
triangolare. Di conseguenza, si può affermare che, se esiste
la Closest String $s^*$ per la collezione $S$, allora la distanza tra due
qualsiasi
stringhe della collezione deve essere sicuramente minore o uguale a
$2 \cdot d$:
\[
    d_H(s_i, s_j) \le 2 \cdot d \quad \forall 1 \le i, j \le k
\]
\begin{thm}
    Se almeno una coppia di stringhe della collezione $S$ presenta una distanza
    di Hamming maggiore di $2d$, allora l'istanza non ammette soluzione,
    e la risposta al problema di decisione è negativa.
\end{thm}


L'algoritmo riceve in input la collezione $S$, il parametro $d$ e una stringa
$\hat{s}$, che dovrà essere modificata in modo da ottenere la stringa ottimale
$s^*$.
L'esecuzione è basata sulla costruzione di un \textbf{bounded-search tree}, e
sfrutta chiamate ricorsive all'algoritmo con $\hat{s}$ e $d$ modificati.
A ogni iterazione (chiamata ricorsiva), l'algoritmo verifica innanzitutto
se una delle stringhe della collezione è già soluzione. Se ciò non si verifica,
controlla allora se le stringhe della collezione presentano una distanza di
Hamming da $\hat{s}$ maggiore di $k \cdot d$, o se due stringhe della
collezione presentano una distanza di Hamming maggiore di $2 \cdot d$:
in entrambi i casi, infatti, il ramo di esecuzione non ammette soluzione.\\
Se nessuno di questi casi si verifica, l'algoritmo sceglie una delle stringhe
della collezione che presenta una distanza di Hamming maggiore di $d$ dalla
stringa $\hat{s}$, e modifica $\hat{s}$ in $d+1$ posizioni per cui $\hat{s}$
differisce dalla stringa della collezione scelta, in modo che
in ognuna di queste posizioni venga inserito il simbolo presente nella
stringa della collezione scelta.
Ogni modifica definisce, a partire dal nodo corrente dell'albero,
un nuovo ramo di computazione, che si tradurrà di fatto in una chiamata ricorsiva
all'algoritmo, con $\hat{s}$ modificata nella singola posizione associata
a quel ramo e il valore di $d$ decrementato di 1.
La scelta di $d+1$ posizioni da modificare, inoltre, assicura che almeno una
delle $d+1$ modifiche inserisca in $\hat{s}$ un simbolo sicuramente presente
nella stringa ottimale $s^*$. Si ricorda che non è però possibile sapere
a priori quale delle modifiche sia quella corretta.


La complessità di questo algoritmo parametrico è pari a
$O(k \cdot n + k \cdot d^{d+1})$.
$d$ è fissato, e quindi costante. Di conseguenza, la complessità non è
esponenziale, ma lineare.



\chapter{Esempi di riduzioni}
\section{Riduzione da SAT a 3-SAT}
\section{Riduzione da Vertex Cover a Set Cover}
\section{Riduzione da 3-SAT a Indipendent Set}
\section{Riduzione da 3-SAT a Vertex Cover}
\section{Riduzione da 3-SAT a Set Cover}
\section{Riduzione da Set Cover a SAT}
