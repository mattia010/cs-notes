\chapter{Apprendimento non supervisionato}
L'apprendimento non supervisionato è un tipo di apprendimento in cui non si è a
conoscenza, prima della fase di apprendimento, delle classi in cui devono essere
classificate le istanze, ed è quindi l'algoritmo che si occupa di individuare 
raggruppamenti (e quindi delle classi) con caratteristiche comuni.

Non conoscendo le classi tra cui suddividere le istanze, le istanze di
addestramento non hanno alcuna label associata.


\section{Clustering}
Gli algoritmi di clustering permettono di suddividere un insieme di dati in
cluster, ovvero sottoinsiemi di elementi (più o meno) omogenei.
Sono basati su tecniche di similarità/dissimilarità tra elementi o cluster,
che solitamente definiscono la similarità di due elementi come la distanza
tra essi.\\
La scelta della metrica, ovvero la distanza da calcolare, influisce molto
sull'efficacia degli algoritmi di clustering.

Tutte le tecniche di clustering, dovendo calcolare una distanza tra elementi,
sono efficienti solo su attributi dotati di ordine interno naturale, come i
numeri.
Un insieme di elementi senza un ordinamento naturale, come l'insieme dei colori,
può essere ordinato artificiosamente in moltissimi modi diversi, e l'algoritmi
può quindi produrre cluster diversi per ogni ordinamento diverso.
Di conseguenza, i raggruppamenti non sono naturali, ma sono influenzati
dall'ordinamento definito.

Un algoritmo di clustering presenta i seguenti vantaggi:
\begin{itemize}
    \item non è necessario conoscere il dominio delle istanze, e quindi le
    classi in cui esse sono realmente divise;
    \item il controllo umano sull'algoritmo è ridotto, e quindi viene
    ridotto l'errore umano;
    \item tutte le classi univocamente caratterizzate sono identificate.
\end{itemize}
Alcuni svantaggi sono invece:
\begin{itemize}
    \item le nuove classi definite dall'algoritmo in fase di apprendimento
    possono non avere alcun significato;
    \item colui che definisce l'algoritmo ha poco controllo sull'apprendimento
    e i risultati ottenuti;
    \item non adatti ad attributi definiti su un insieme con ordinamento
    arbitrario.
\end{itemize}

\subsection{Classificazione degli algoritmi di clustering}
Gli algoritmi di clustering possono essere classificati in:
\begin{itemize}
    \item \textbf{algoritmi partizionali} \footnote{Alcuni esempi di algoritmi
    di clustering partizionali sono K-means, il metodo di Forgy e le
    aggregazioni dinamiche.}: l'algoritmo crea una $k$-partizione dell'insieme
    delle istanze, e l'appartenenza di un elemento a un cluster è definita dalla
    sua distanza da un punto rappresentante il cluster.\\
    In un approccio partizionale, l'algoritmo deve essere a conoscenza del
    numero $k$ di cluster che devono formare la partizione.\\
    \item \textbf{algoritmi gerarchici} \footnote{Alcuni esempi di algoritmi
    gerarchici sono il Neighbour Joining e il metodo del centroide.}: questi
    algoritmi creano una gerarchia di partizioni a partire da una partizione di
    partenza (solitamente l'intero insieme delle istanze o la partizione formata
    da $n$ sottoinsieme, con $n$ numero di istanze.).\\
    La gerarchia di partizioni può essere rappresentata tramite un dendrogramma.
    \item \textbf{algoritmi basati sulla teoria dei grafi} 
            \footnote{Alcuni esempi di algoritmi di clustering basati sulla
            teoria dei grafi sono Highly Connected Subgraph (HCS) e CLustering
            Identification via Connectivity Kernels (CLICK).}.
    \item \textbf{algoritmi euristici}
            \footnote{Alcuni esempi di algoritmi di clustering euristici sono
            Clustering Affinity Search Technique (CAST) e Self-Organizing Maps
            (SOM)}.
\end{itemize}
Gli algoritmi di clustering possono anche essere suddivisi in esclusivi, se ogni
elemen\-to può appartenere a un solo cluster, o non-esclusivi, se un elemento
può appartenere a cluster diversi.


\section{Clustering partizionale}
\subsection{k-means}
$k$-means è un algoritmo di clustering partizionale divisivo 
            \footnote{Con algoritmo divisivo si intende un algoritmo che inizia
            la sua esecuzione considerando un unico intero insieme di istanze, e
            cerca di definire i cluster partizionando iterativamente l'insieme
            precedente.}
, che definisce i $k$ custering cercando di minimizzare le distanze tra gli
elementi e i cluster a cui appartengono.

Essendo un algoritmo partizionale, $k$-means è efficace solo su dati con un
ordinamento naturale (solitamente numeri).

L'idea dell'algoritmo è quella di partire fissando $k$ centroidi iniziali, e di
modificare la loro posizione a ogni iterazione, in modo tale da definire $k$
cluster il più possibile omogenei.
Un approccio del genere può richiedere però un numero di iterazioni altissimo
(e quindi un'alta potenza di calcolo) e, di conseguenza, molto spesso si impone
un numero di iterazioni massimo.\\
L'algoritmo termina quando è stato raggiunto il numero massimo di iterazioni o
quando converge, ovvero i la posizione dei centroidi non viene più modificata.

\subsection{Vantaggi e svantaggi di $k$-means}
I vantaggi di $k$-means sono principalmente due:
\begin{itemize}
    \item facile da implementare;
    \item tempo di calcolo pari a $O(tkn)$ 
            \footnote{Dove $t$ è il numero di iterazioni massimo, $k$ è il
            numero di cluster e $n$ è il numero di elementi.}
    , e quindi abbastanza contenuto. 
\end{itemize}
Presenta però seri svantaggi:
\begin{itemize}
    \item la suddivisione in cluster dipende fortemente dalla scelta dei $k$
    centroidi iniziali;
    \item nonostante non si abbia alcuna conoscenza sul dominio degli elementi,
    l'algoritmo richiede di specificare il numero $k$ di cluster desiderati;
    \item non esiste un $k$ ottimale;
    \item è sensibile alla densità degli elementi: non è detto che elementi
    presenti in una regione di spazio densa vengano raggruppati nello stesso
    cluster, come naturalmente ci si aspetterebbe;
    \item è sensibile alla disposizione geometrica degli elementi 
            \footnote{Un esempio di dispozione geometrica che mette in
            difficoltà l'algoritmo $k$-means è la doppia spirale.}
    . Una possibile soluzione a questo problema è quella di utilizzare un numero
    superiore di cluster, per poi procedere alla loro unione fino a ottenere i
    $k$ cluster desiderati.
\end{itemize}

\subsection{Determinare il miglior numero $k$ di cluster}
L'algoritmo $k$-means, per essere eseguito, necessita del numero $k$ di cluster
in cui si vuole suddividere gli elementi.
Non avendo però alcuna (o una limitata) conoscenza del dominio, specificare il
$k$ ideale può non essere banale.\\
Per questo motivo, il miglior $k$ può essere trovato procedendo con un'analisi
delle silhouette medie dei cluster, ovvero misurando quanto sono raggruppati
(o eventualmente dispersi) gli elementi appartenenti a un cluster.

Sia $C_I$ l'$I$-esimo cluster. La silhouette $s(i)$ di un elemento $i \in C_I$
viene calcolata come:
\begin{align*}
    s(i) &= \begin{cases}
        \frac{b(i)-a(i)}{\max\cbra{a(i),b(i)}} & |C_I| > 1\\
        0 & |C_I| = 1
    \end{cases}\\
    a(i) = \frac{1}{|C_I| - 1} &\sum\limits_{j \in C_I, i \neq j} d(i, j)
    \quad \quad \quad
    b(i) = \min\limits_{J \neq I} \frac{1}{|C_J|} \sum\limits_{j \in C_J} d(i,j)
\end{align*}
dove $a(i)$ è la distanza media di $i$ da ogni altro elemento all'interno del
cluster a cui appartiene (distanza media intra-cluster), mentre $b(i)$ è la
minima distanza media di $i$ da ogni altro elemento di un qualsiasi altro
cluster di cui $i$ non è membro (distanza media inter-cluster).\\
$a(i)$ può essere vista come una misura della buona assegnazione di $i$ al
cluster corrente (più il valore è piccolo, più l'assegnamento è corretto).

Per un qualsiasi elemento $i$, $-1 \le s(i) \le 1$. Più il valore $s(i)$ si
avvicina a $1$, più l'elemento appartiene al cluster ideale. Più il valore
$s(i)$ si avvicina a $-1$, invece, più l'elemento dovrebbe essere appartenere al
cluster neighbour, ovvero al cluster più vicino diverso da quello corrente.
Un $s(i) = 0$ indica che l'elemento risiede alla frontiera che separa il cluster
corrente e il cluster neighbour. 

Si può procedere con una prima esecuzione dell'algoritmo $k$-means con un valore
arbitrario per $k$ (scelto comunque attraverso alcune euristiche).
Al termine dell'esecuzione, vengono calcolate la silhouette media su tutti gli
elementi del dataset. Se la silhouette si avvicina a $1$, allora il $k$ scelto
è giusto e il clustering effettuato è naturale. 
Se invece $k$ si avvicina a $-1$, allora il $k$ scelto è sbagliato e bisogna
ripetere l'esecuzione dell'algoritmo $k$-means diminuendo o aumentando il valore
di $k$ (non si può sapere a priori se bisogna aumentarlo o diminuirlo, si sa
solo che bisogna farlo) 
\footnote{Alcuni algoritmi moderni permettono di effettuare il clustering
direttamente con la misura di silhouette, senza quindi aver bisogno di un
esecuzione multipla dell'algoritmo di clustering.}
.


\subsection{Similarità tra elementi: distanza}
Gli algoritmi di clustering si basano sul concetto di similarità tra elementi.
Essendo i dati su cui l'algoritmo di clustering opera dotati di un ordine
naturale, un elemento può essere visto come un vettore.
La similarità tra vettori può essere quindi vista come distanza tra essi: più i
vettori sono vicini, più sono simili.

I tre tipi di distanze che si possono calcolare su vettori $n$-dimensionali
sono:
\begin{itemize}
    \item distanza di Manhattan
    \[
        d(\vec{x}, \vec{y}) = \sum\limits_{i=1}^n (x_i - y_i)
    \]
    La distanza di Manhattan non è invariante rispetto a traslazione e rotazioni
    dei vettori nello spazio e non aumenta l'influenza di componenti dei vettori
    con distanze maggiori, in quanto non eleva al quadrato.
    \item distanza euclidea
    \[
        d(\vec{x}, \vec{y}) = 
        \bra{\sum\limits_{i=1}^n (x_i - y_i)^2}^{\frac{1}{2}}
    \]
    La distanza euclidea è invariante rispetto aa traslazioni e rotazioni dei
    vettori nello spazio.
    \item distanza di Minkowski \footnote{Vedere spazio-tempo di Minkowski.}
    \[
        d(\vec{x}, \vec{y}) = 
        \bra{\sum\limits_{i=1}^n (x_i - y_i)^k}^{\frac{1}{k}} \quad k \in 
    \]
    La distanza di Minkowski è una generalizzazione della distanza di Manhattan
    ($q=1$) e della distanza euclidea ($q=2$).
    è inoltre una metrica sse $p \ge 1$, ovvero soddisfa la disuguaglianza
    triangolare solo per questi valori.
\end{itemize}