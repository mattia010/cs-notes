\chapter{Introduzione}
Il machine learning (d'ora in poi ML) è una branca dell'intelligenza artificiale
(d'ora in poi AI), il cui obiettivo è quello di individuare e saper riconoscere
pattern all'interno di una grande mole di dati.
Mentre con AI si intende qualsiasi algoritmo che imita il comportamento umano, e
include quindi, oltre agli algoritmi basati sull'apprendimento, anche quelli
rule-based, con ML si fa riferimento a qualsiasi algoritmo che, basandosi su una
grande mole di dati e grazie a metodi statistici, può individuare pattern per
risolvere un compito ben specifico.

Un algoritmo di ML si può dividere in due fasi:
\begin{enumerate}
    \item \textit{fase di apprendimento}, in cui l'algoritmo riesce a estrarre
    pattern da una grande mole di dati, detti istanze di addestramento, in modo
    tale da categorizzarle;
    \item \textit{fase di inferenza}, in cui l'algoritmo categorizza un'istanza
    passata in input.
\end{enumerate}
L'apprendimento può essere di tre tipi:
\begin{itemize}
    \item supervisionato (detto anche apprendimento induttivo): per ogni istanza
    fornita in input, viene fornito anche il relativo \textit{label}, ovvero la
    categoria a cui l'istanza appartiene;
    \item non supervisionato: le istanze non sono categorizzate, ed è quindi
    compito dell'algoritmo individuare eventuali pattern nascosti;
    \item per rinforzo: l'algoritmo interagisce con un sistema complesso, e
    riceve un premio o una punizione in base a come si relaziona con esso.
    L'algoritmo, ovviamente, cercherà di essere punito il meno possibile, e
    quindi tenderà ad adottare il comportamento migliore per il sistema
    complesso in cui è inserito.
\end{itemize}
Esiste una funzione ideale, detta \textit{target}, che rappresenta il miglior
modo di categorizzare le istanze, e di risolvere quindi il problema che si sta
affrontando. Un algoritmo di ML non fa altro che definire un'ulteriore funzione,
detta \textit{ipotesi}, che, idealmente, è la migliore approssimazione del
target, e dovrebbe quindi permettere di categorizzare correttamente il maggior
numero di istanze.

L'ipotesi elaborata dall'algoritmo può essere valutata, in modo da verificare la
sua capacità predittiva. Alcune domande che ci si può porre per verificare la
sua bontà sono "La categorizzazione delle istanze è corretta?" e "Quanti sono i
falsi positivi? E i falsi negativi?".
Un modo di verificare la qualità dell'ipotesi è tramite l'analisi della sua
regressione, ovvero il suo errore nel categorizzare le istanze.

\section{Overfitting}
L'ipotesi generata in fase di addestramento deve essere consisistene con il
dataset, ovvero deve fornire, per ogni istanza, un risultato uguale a quello
definito dalla label associata.
Essere consistente con il dataset non implica, però, una buona capacità
predittiva: un'ipotesi troppo specifica può infatti faticare ad effettuare
previsioni corrette.

Se la funzione risulta quindi consistente ma non è capace di fare previsioni su
istanze nuove, allora si dice che l'algoritmo è andato in \textit{overfitting}.

Sia $h$ un'ipotesi elaborata da un algoritmo di apprendimento, $h$ è un
overfitting sui dati di addestramento se esiste un'ipotesi $h'$ tale che:
\begin{itemize}
    \item $\textnormal{error}_{\textnormal{ TRAINING}} (h) < 
    \textnormal{error}_{\textnormal{ TRAINING}} (h')$
    \item $\textnormal{error}_{\textnormal{ INFERENCE}} (h) > 
    \textnormal{error}_{\textnormal{ INFERENCE}} (h')$
\end{itemize}
dove l'errore di training è calcolato su tutto il dataset di addestramento,
mentre l'errore di inferenza è calcolato su tutto il dominio delle istanze.
Per poter già individuare in fase di addestramento questo caso, e quindi cercare
di evitarlo, il dataset viene partizionato in tre sottoinsiemi:
\begin{itemize}
    \item istanze di addestramento: istanze effettivamente usate dall'algoritmo
    per elaborare le ipotesi. Coprono circa l'80\% del dataset;
    \item istanze di validazione: istanze utilizzate per verificare la
    consistenza di un'ipotesi. L'ipotesi è consistente sse concorda con la
    funzione target su tutte queste istanze. Coprono circa il 10\% del dataset;
    \item istanze di test: istanze usate per individuare casi di overfitting.
    Coprono circa il 10\% del dataset.
\end{itemize}