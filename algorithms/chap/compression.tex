\chapter{Algoritmi di compressione}
\section{LZ77/LZ78}
Algo di compressione che usano poca memoria principale nel corso della compressione:
l'algo comprime mentre legge, e quindi non si necessita che tutto il testo
sia in memoria per la compressione, ma si può caricare solo un pezzo
da leggere e scorrere una volta compresso.

Sono inoltre algo molto semplici da implementare, ma efficaci.

Si legge la porzione di testo $T[i:n]$ (quindi da $i$ fino alla fine del testo)
supponendo che $T[1:i-1]$ sia già stato compresso.\\
Se l'input è $T[i:n]$, l'output è una sequenza di triple $(o, |P|, S[k])$,
dove $o$ è un offset all'indietro dalla posizione $i$, che indica
la posizione di partenza di un'occorrenza del prefisso $P$ in $T[1:i]$,
$|P|$ è la lunghezza del prefisso che si è trovato e
$T[|P|]$ è il simbolo che segue il prefisso che si sta considerando. \\
Si cerca infatti il prefisso $P$ del testo $T[i:n]$ che occorre in $T[1:i-1]$.
Il testo compresso sarà questa sequenza di triple.

