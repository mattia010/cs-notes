\chapter{Calculus of Communicating Systems}
\section{Introduzione}
Il Calculus of Communicating System (CCS) è un algebra di processi
\footnote{Altri esempi di algebre di processi sono Communicating Sequential
Processes, Algebra of Communicating Processes, il $\pi$-calcolo, l'ambient
calculus e il join-calculus.}
\footnote{Una delle caratteristiche che distinguono le algebre di processi
da altri modelli che permettono di modellare la concorrenza, come le reti
di Petri, è che le algebre di processi utilizzano il concetto di canale
per la comunicazione tra processi. Il concetto di canale delle algebre di
processi astrae quello che nell'implementazione è un canale ben definito con
una complessa struttura, che permette di far comunicare i vari processi.},
ovvero un formalismo per modellare sistemi concorrenti,
basato sui concetti di azioni e operatori.
In particolare, il CCS permette di descrivere
i processi che compongono il sistema concorrente come espressioni, che combinano
vecchi processi per generarne di nuovi. Permette, inoltre, di modellare
le interazioni e le sincronizzazioni tra i processi del sistema utilizzando
le azioni.
Come tutte le algebre di processi, si assume che i processi interagiscano
tramite scambio di messaggi, e che quindi non condividano memoria.

Il principale utilizzo del CCS è quello di verificare se sistemi concorrenti
sono corretti dal punto di vista di proprietà desiderate sul sistema, come
l'assenza di deadlock e livelock.

Il CCS, essendo un algebra di processi, non permette di esprimere la true
concurrency del sistema concorrente, ma la modellazione del sistema altro non
è che una simulazione sequenziale non deterministica dello stato globale
del sistema.
Se l'obiettivo è quello di rappresentare la true concurrency, altri modelli
formali si rivelano più adatti, come le reti di Petri.

\section{Sintassi e semantica}
Il CCS è fortemente legato al concetto di azione, che modella la comunicazione
che avviene tra due processi. Proprio perchè la comunicazione coinvolge due
processi, ogni azione ha una rispettiva coazione, che rappresenta
l'azione opposta effettuata dall'altro processo coinvolto nell'interazione.
Un esempio che permette di capire facilmente questo concetto è quello
che coinvolge le azioni \verb|prendere| e \verb|dare|: se una persona
prende un oggetto da un'altra persona, allora quest'ultima avrà dato l'oggetto
alla prima.

\begin{rem}
    $\overline{a}$ è la coazione dell'azione $a$. $\overline{\overline{a}}$,
    ovvero la coazione della coazione di $a$, è l'azione $a$ stessa. ovvero:
    \[
        \overline{\overline{a}} = a
    \]
\end{rem}

Sia quindi $A$ l'insieme dei nomi di azioni che possono essere effettuate nel sistema,
e sia $\overline{A}$ l'insieme dei nomi delle coazioni delle azioni contenute in $A$.
L'insieme di tutte le possibili azioni Act è definito come:
\[
    \text{Act} = A \cup \overline{A} \cup \cbra{\tau}
\]
dove $\tau$ è l'azione di sincronizzazione tra processi, che avviene
quando due processi effettuano azione e coazione l'uno nei confronti dell'altro.

Come già detto, i processi CCS non sono altro che espressioni CCS,
che combinano altri processi CCS usando appositi operatori.
L'insieme $P$ dei processi CCS validi è quindi definito come:
\begin{itemize}
    \item processo nullo:
    \[
        \text{NIL } \in P
    \]
    dove NIL è il processo che non fa niente.
    \item azione/prefisso:
    \[
        \alpha.p_1 \in P, \quad \alpha \in \text{Act}, p_1 \in P
    \]
    \item identificazione di processo:
    \[
        A \in P
    \]
    dove $A = p_1$ è un identificatore per il processo $p_1$.
    \item somma:
    \[
        p_1 + p_2 \in P, \quad p_1, p_2 \in P
    \]
    \item composizione parallela:
    \[
        p_1 \,|\, p_2 \in P, \quad p_1, p_2 \in P
    \]
    \item restrizione:
    \[
        p_1 \setminus \alpha \in P, \quad p_1 \in P, \alpha \in \text{Act}
    \]
    \item renaming:
    \[
        p_1[\sfrac{\beta}{\alpha}], \quad p_1 \in P, \alpha, \beta \in \text{Act}
    \]
\end{itemize}

La semantica degli operatori CCS è la seguente:
\begin{itemize}
    \item processo nullo: NIL è il processo che non fa niente.
    \item azione/prefisso: $\alpha.p_1$ è il processo che può effettuare
    l'azione $\alpha$ per poi comportarsi come il processo $p_1$.
    \item identificazione di processo: $A = p_1$ permette di far riferimento
    al processo $p_1$ usando l'identificatore $A$. Nel processo $p_1$ può
    a sua volta comparire l'identificatore $A$, permettendo quindi definizioni
    ricorsive.
    \item somma: $p_1 + p_2$ è il processo che può comportarsi come il
    processo $p_1$ o come il processo $p_2$. Si comporta quindi come un
    operatore di scelta.
    \item composizione parallela: nel processo $p_1 \,|\, p_2$,
    i processi $p_1$ e $p_2$ esistono contemporaneamente e vengono eseguiti
    in maniera concorrente.
    \item restrizione: $p_1 \setminus \alpha$ è il processo $p_1$ che però
    non può compiere l'azione $\alpha$, a meno che non effettui una
    sincronizzazione sull'azione.
    \item renaming: $p_1[\sfrac{\beta}{\alpha}]$ è il processo $p_1$ in
    cui tutte le azioni di nome $\alpha$ sono state rinominate come $\beta$.
\end{itemize}

\section{Processi CCS come sistemi di transizioni etichettati}
I processi CCS sono di fatto espressioni manipolabili algebricamente.
Può però essere utile interpretare i processi CCS come sistemi di transizione
etichettati, in modo da facilitare la loro comprensione, trattazione
e renderli rappresentabili graficamente.

Si ricorda che un sistema di transizioni etichettato è una quadrupla
$(S, E, T, s_0)$ dove $S$ è un insieme di stati, $E$ è un insieme di etichette,
$T$ è un insieme di archi etichettati, e $s_0$ è lo stato iniziale.
Tradurre un processo CCS in un LTS è molto semplice: $S$ è infatti l'insieme
dei possibili processi CCS raggiungibili a partire dal processo considerato,
mentre $E = Act$, ovvero è l'insieme di tutte le possibili azioni.
Si possono poi facilmente definire anche $T$ e $s_0$.

\section{Equivalenza semantica tra processi}
Può essere utile, in alcuni casi, stabilire se due processi si comportano allo
stesso modo quando interagiscono con un particolare sistema concorrente:
in questo caso, infatti, i due processi sono intercambiali all'interno del
sistema, e l'utilizzo di un processo o dell'altro all'interno del sistema
concorrente non fa variare il comportamento dell'intero sistema concorrente.
Due processi possono comportarsi allo stesso modo anche se diversi tra
loro, e per questo viene introdotta la nozione di equivalenza all'osservazione:
due processi sono equivalenti all'osservazione sse interagiscono allo stesso
modo con un particolare sistema concorrente, e il loro comportamento
risulta identico a un osservatore esterno. Solo in questo caso,
i due processi possono essere considerati congruenti, ed essere sostituiti
nel sistema concorrente senza mofificarne il comportamento.
Si deve quindi definire un criterio, ovvero una relazione equivalenza
tra processi, che permetta di stabilire se due processi siano effettivamente
equivalenti all'osservazione, e quindi intercambiabili.

La relazione di equivalenza \footnote{Una relazione di equivalenza è riflessiva,
simmetrica e transitiva.}
tra processi che si sta cercando deve avere
le seguenti proprietà:
\begin{itemize}
    \item se due processi sono isomorfi, allora sicuramente sono equivalenti
    secondo la relazione di equivalenza.
    \item se due processi sono equivalenti secondo la relazione, allora
    sicuramente sono equivalenti alle tracce.
    \item la relazione deve astrarre dagli stati e dalle azioni di sincronizzazione.
    \item la relazione deve preservare i deadlock.
    \item la relazione deve essere una congruenza rispetto agli operatori del
    CCS.
\end{itemize}

D'ora in poi, i processi CCS verranno interpretati come sistemi di transizione
etichettati, in modo da definire più facilmente l'equivalenza semantica
tra processi.

\subsection*{Isomorfismo}
Il primo approccio che si può usare per affrontare il problema dell'equivalenza
all'osservazione è quello di verificare se due processi CCS, o meglio i loro
LTS, sono isomorfi.

Due sistemi di transizioni etichettati $(S_1, \text{Act}_1, T_1, s_1)$ e
$(S_2, \text{Act}_2, T_2, s_2)$ sono isomorfi sse esistono due funzioni biunivoche
$\alpha: S_1 \rightarrow S_2$ e $\beta: \text{Act}_1 \rightarrow \text{Act}_2$
tali che:
\begin{itemize}
    \item $\alpha(s_1) = s_2$
    \item $\forall s,s' \in S_1, \forall a \in \text{Act}_1 \quad
    (s,a,s') \in T_1 \leftrightarrow (\alpha (s), \beta (a), \alpha (s')) \in T_2$
\end{itemize}

L'isomorfismo tra LTS è una congruenza rispetto agli operatori CCS, ma
non astrae dagli stati e non preserva i deadlock. L'isomorfismo si rivela
quindi essere troppo stringente per verificare l'equivalenza
all'osservazione.

\subsection*{Equivalenza forte alle tracce}
Si introduce quindi la nozione di equivalenza alle tracce, in particolare
di equivalenza forte alle tracce.
Le tracce forti di un processo $p \in P$ sono tutte le possibili sequenze
di azioni eseguibili all'interno del sistema concorrente a partire dal processo
$p$ considerato.
\[
    \text{Tracce}(p) = \cbra{w \in \text{Act}^* |
    \exists p' \in \text{Proc. CCS}, p \strongTrans{w} p'}
\]
Due processi si dicono equivalenti alle tracce forti $P_1 \strongTrace P_2$ sse
$\text{Tracce}(P_1) = \text{Tracce}(P_2)$.

L'equivalenza alle tracce forte è implicata dall'isomorfismo ed è una congruenza
rispetto agli operatori CCS, ma non astrae dalle azioni si sincronizzazione
e non preserva i deadlock. Di conseguenza, anche l'equivalenza forte
alle tracce si rivela essere troppo restrittiva per analizzare l'equivalenza
semantica tra processi.

\subsection*{Equivalenza debole alle tracce}
Si può quindi provare a definire la nozione di equivalenza debole alle tracce.
Per definire l'equivalenza alle tracce debole è necessario prima definire
la relazione di transizione debole $\weakTrans{\alpha}$.
\begin{defn}
    Siano $p, p' \in P$ due processi CCS e $\alpha \in \text{Act}$,
    allora $p \weakTrans{\alpha} p'$ sse:
    \begin{itemize}
        \item $p \strongTrans{\tau*} p'$ se $\alpha = \tau$
        \item $p \strongTrans{\tau*} \ldots \strongTrans{\alpha} \ldots \strongTrans{\tau*} p'$
        se $\alpha \neq \tau$
    \end{itemize}

    Sia ora $w \in \text{Act}^*$, $w = a_1 \ldots a_n$. $p \weakTrans{w} p'$ sse
    $p \weakTrans{a_1} \ldots \weakTrans{a_n} p'$.
\end{defn}

\begin{defn}
    Siano $p, q \in P$ due processi CCS. I due processi sono
    debolmente equivalenti alle tracce, ovvero $p \weakTrace q$, sse
    $\text{Tracce}_{\Rightarrow}(p) = \text{Tracce}_{\Rightarrow}(q)$.

    Con $\text{Tracce}_{\Rightarrow}(p)$ si intendono tutte le transizioni deboli
    che è possibile compiere a partire da $p$, ovvero:
    \[
        \text{Tracce}_{\Rightarrow}(p) = \cbra{w \in \text{Act}^* |
        p \weakTrans{w} p', p' \in P}
    \]
\end{defn}

L'equivalenza alle tracce debole $\weakTrace$ è ovviamente meno restrittiva dell'equivalenza
alle tracce forte.
Di conseguenza, se due processi sono fortemente equivalenti alle tracce,
essi implicitamente sono debolemente equivalenti alle tracce. Vale quindi
$\strongTrace \, \subseteq \, \weakTrace$.

L'equivalenza alle tracce debole è implicata dall'isomorfismo,
è una congruenza rispetto agli operatori CCS, e permette di astrarre
dagli stati e dalle azioni di sincronizzazione.
Come per la sua versione forte, però, l'equivalenza alle tracce debole non
permette però di preservare i deadlock: di conseguenza, anche l'equivalenza
debole alle tracce non è adatta per confrontare i processi.

\subsection*{Bisimulazione forte}
Si introduce quindi la nozione di bisimulazione, in particolare di bisimulazione
forte.

\begin{defn}
    La relazione di bisimulazione forte
    $\strongBisim \subseteq P \times P$
    è una relazione binaria tale che $\forall p, q \in P$, $p \strongBisim q$ sse:
    \begin{itemize}
        \item $\forall \alpha \in \text{Act}, \quad p \strongTrans{\alpha} p'
        \implies \exists q' \in P : q \strongTrans{\alpha} q' \land p' \strongBisim q'$
        \item $\forall \alpha \in \text{Act}, \quad q \strongTrans{\alpha} q'
        \implies \exists p' \in P : p \strongTrans{\alpha} p' \land p' \strongBisim q'$
    \end{itemize}
\end{defn}

\begin{rem}
    Due processi sono fortemente bisimili sono sicuramente fortemente
    equivalenti alle tracce.
    Non è detto però che, viceversa, due processi fortemente equivalenti
    alle tracce siano fortemente bisimili.
    Si può quindi affermare che la bisimulazione forte è più restrittiva
    rispetto equivalenza forte alle tracce.
\end{rem}

La bisimulazione forte è implicata dall'isomorfismo, preserva i deadlock ed è
una congruenza rispetto agli operatori CCS. \textit{Non permette}, però,
\textit{di astrarre dalle azioni di sincronizzazione}, e non si rivela quindi adatta a verificare
l'equivalenza semantica di processi.

\subsection*{Bisimulazione debole e congruenza tra processi}
\begin{defn}
    Una relazione di bisimulazione debole $\weakBisim$ è una relazione
    binaria tale che $\forall p, q \in P$, $p \weakBisim q$ sse:
    \begin{itemize}
        \item $\forall \alpha \in \text{Act}, \quad p \strongTrans{\alpha} p'
        \implies \exists q' \in P : q \weakTrans{\alpha} q' \land p' \weakBisim q'$
        \item $\forall \alpha \in \text{Act}, \quad q \strongTrans{\alpha} q'
        \implies \exists p' \in P : p \weakTrans{\alpha} p' \land p' \weakBisim q'$
    \end{itemize}
\end{defn}

La bisimulazione debole $\weakBisim$ è ovviamente meno restrittiva della bisimulazione
forte $\strongBisim$. Di conseguenza, $\strongBisim \subseteq \weakBisim$.

\begin{rem}
    La bisimulazione debole è più restrittiva dell'equivalenza alle tracce debole.
    Se due processi sono debolmente bisimili, allora sono anche debolmente
    equivalenti alle tracce. Non è sempre vero il contrario.
\end{rem}
La bisimulazione debole permette di preservare i deadlock nell'interazione
con l'ambiente e permette di astrarre l'equivalenza da azioni non osservabili,
ovvero azioni $\tau$, e cicli non osservabili, ovvero cicli $\tau$.
\upperAccE, inoltre, implicata dall'isomorfismo.
La bisimulazione debole non è però una congruenza per gli operatori del CCS,
in quanto non lo è per l'operatore + e la ricorsione: di conseguenza,
anche la bisimulazione debole non è adatta per analizzare l'equivalenza
semantica di due processi di un sistema concorrente.

Di conseguenza, è possibile definire una nuova relazione che goda
di tutte le proprietà possedute dalla bisimulazione debole, ma che al contempo
sia una anche una relazione di congruenza rispetto a tutti gli operatori
del CCS, compresi + e ricorsione.
Per farlo, però, la nuova relazione dovrà necessariamente essere definita tramite
assiomi.
Viene quindi definita $\congr$ come la più grande
relazione di congruenza contenuta in $\weakBisim$ per il CCS senza ricorsione,
e con agenti finiti.

Assiomi della relazione $\congr$:
\begin{itemize}
    \item $p+(q+r) \congr (p+q)+r \quad \land \quad
    p\,|\,(q\,|\,r) \congr (p\,|\,q)\,|\,r$
    \item $p+q \congr q+p \quad \land \quad p\,|\,q \congr q\,|\,p$
    \item $p+p \congr p \quad \land \quad p\,|\,p \notCongr p$
    \item $p+\text{NIL} \congr p \quad \land \quad p\,|\,\text{NIL} \congr p$
    \item $p+\tau .p \congr \tau .p$
    \item $\mu . \tau .p \congr \mu .p$
    \item $\mu .(p+ \tau .q) \congr \mu .(p+ \tau .q) + \mu .q$
    \item $p\,|\,q \congr \sum_i \alpha_i .(p_i \,|\,q) +
    \sum_j \beta_j .(p \,|\,q_j) + \sum_{\alpha_i = \overline{\beta_j}}
    \tau .(p_i, q_j)$ \\ (teorema di espansione di Miller)
    \item $p\sbra{f} \congr \sum_i f(\alpha_i).(p_i\sbra{f}) \quad
    \forall f \tn{ funzione di etichettatura}$
    \item $p \setminus L \congr \sum\limits_{\alpha_i, \overline{\alpha_i} \notin L}
    \alpha_i.(p_i \setminus L) \quad \forall L \subseteq A$
\end{itemize}
Gli assiomi della relazione di congruenza $\congr$ sono corretti e completi.

\subsection*{Processi deterministici}
\begin{defn}
    Un processo CCS si dice \textbf{processo deterministico} sse
    \[
        \forall x \in \text{Act}, \quad
        p \strongTrans{x} p' \land p \strongTrans{x} p" \implies p' = p"
    \]
    ovvero da uno stesso stato non può partire la stessa azione verso due stati
    diversi. Con azione si intende anche l'azione di sincronizzazione $\tau$.
\end{defn}

\begin{defn}
    Se due processi sono deterministici, allora:
    \begin{itemize}
        \item $p \strongTrace q \implies p \strongBisim q$
        \item $p \weakTrace q \implies p \weakBisim q$
    \end{itemize}
    Ciò non è sempre garantito se almeno uno dei due processi è non deterministico.
\end{defn}



\section{Gioco dell'attaccante-difensore}
FINIRE SLIDE 3


\section{Programmazione concorrente con le algebre di processo}
ERLANG e GO
