\chapter{Model checking}
\section{Introduzione}
Quando si sviluppano sistemi software complessi, è importante che il sistema
rispetti tutti i requisiti definiti in fase di progettazione.
Questi requisiti, molto spesso, sono espressi in linguaggio naturale, e vengono
poi affiancati da rappresentazioni grafiche, come diagrammi UML.
Definire però correttamente tutti i requisiti di un sistema software
complesso risulta molto
: oltre che garantire la presenza di tutti i requisiti richiesti, bisogna
controllare che essi non siano ambigui e non si contraddicano.
Verificare queste caratteristiche algoritmicamente risulta però complicato:
il linguaggio naturale, usato appunto per descrivere i requisiti, risulta
molto complesso da analizzare e, allo stato attuale impreciso.
Per questo, si è deciso di trattare i requisiti e la loro verifica in
maniera più formale, ed è stato introdotto il model checking.
Il model checking è un metodo per verificare algoritmicamente la correttezza di
sistemi formali. In particolare, il model checking prevede di rappresentare
il sistema considerato come un modello formale \footnote{Con
modello del sistema si intende la sua implementazione.},
solitamente un sistema a transizioni etichettato, e di descrivere la specifica
\footnote{La specifica è una proprietà che deve essere rispettata dal sistema.},
ovvero i requisiti, usando un linguaggio preciso e non ambiguo, solitamente
la logica temporale, e utilizza algoritmi ben definiti per stabilire
se la specifica, o altre proprietà desiderate, è verificata nel modello
che rappresenta il sistema. Se una certa proprietà non è rispettata nel modello,
inoltre, il model checking prevede di fornire una dimostrazione, ovvero una
traccia di esecuzione che funge da controesempio alla proprietà considerata.

Una logica temporale è un'estensione della logica proposizionale classica,
in cui la struttura di interpretazione è una successione di istanti di tempo
distinti.
Non esiste un'unica logica temporale, ma ne esistono molteplici, tutte
sotto del $\mu$-calcolo, una logica modale.
Qui verranno analizzate la Linear Temporal Logic e la Computational Tree
Logic.


\section{Modelli di Kripke}
Il model checking viene effettuato su modelli detti modelli di Kripke,
che altro non sono che sistema a transizioni non etichettati in cui
a ogni stato è associato un sottoinsieme di proprietà valide in quello stato.
Vengono quindi introdotte alcune nozioni necessarie per trattare i modelli di
Kripke.

\begin{defn}
    Un \textbf{sistema di transizioni} è una coppia $(Q, T)$ dove:
    \begin{itemize}
        \item $Q$ è un insieme di stati \footnote[][1cm]{L'insieme di stati
        può essere infinito, anche non numerabile. In queste note, a meno che
        venga esplicitato, si tratteranno sistemi di transizioni finiti,
        ovvero con un insieme di stati finito.}.
        \item $T \subseteq Q \times Q$ è la relazione di transizione, ovvero
        quella relazione che stabilisce i collegamenti tra stati del sistema.
    \end{itemize}
\end{defn}

\begin{rem}
    In un sistema di transizioni non etichettato, per ogni coppia di nodi $u$
    e $v$ può essere presente al massimo un arco $(u, v)$ che li collega.
\end{rem}

\begin{defn}
    Un \textbf{cammino} è una sequenza finita di stati
    $q_0 q_1 \ldots q_k, q_i \in Q$.
\end{defn}

\begin{defn}
    Un \textbf{cammino massimale} è un cammino che non può essere esteso.
\end{defn}

\begin{defn}
    Il \textbf{modello di Kripke} è una tripla:
    \[
        (Q, T, I)
    \]
    dove:
    \begin{itemize}
        \item $(Q, T)$ è un sistema di transizioni che rappresenta il sistema
        concorrente;
        \item $I: Q \rightarrow \mathcal{P}(AP)$ è una funzione che associa a
        ogni stato un sottoinsieme di proposizioni atomiche vere in quello
        stato. $I$ viene detta \textbf{funzione di interpretazione}.\\
        $I(q)$ indica il sottoinsieme di proposizioni atomiche vere in $q$.
    \end{itemize}
\end{defn}

A ogni stato $q$ è associato un insieme di cammini massimali per quello stato.\\
Una WFF è vera in uno stato $q$ sse è vera per tutti i cammini massimali che
partono da $q$. Se una WFF $p$ è vera in un certo cammino $\pi$, allora si
scrive $\pi \vDash p$.

Un sistema concorrente rappresentato tramite un modello di Kripke soddisfa una
specifica, ovvero una WFF della LTL, sse la specifica è vera per tutti i
cammini massimali che partono dallo stato iniziale.

Due modelli di Kripke $M_1$ e $M_2$, con stati iniziali rispettivamente
$q_0$ e $s_0$, sono equivalenti rispetto a una logica $L$ sse $\forall \alpha \in L$,
ovvero per tutte le formule della logica,
\[
    M_1, q_0 \vDash \alpha \longleftrightarrow M_2, s_0 \vDash \alpha
\]
ovvero se una formula logica vale nello stato iniziale di un modello, allora
deve valere anche nello stato iniziale dell'altro modello.

\section{Logiche temporali}
\subsection{Linear Temporal Logic}
La Linear Temporal Logic è una logica modale temporale in cui le modalità
fanno riferimento al tempo. In particolare, LTL permette di esprimere
proprietà su un tempo discreto, organizzato linearmente, infinito
e orientato al futuro.

La LTL presenta però dei limiti: essa non può, infatti, esprimere
proprietà riguardanti insiemi di cammini, ovvero non può richiedere
che una certa proprietà sia rispettata da tutti i cammini o da almeno un
cammino.
Un esempio di proprietà che non può essere espressa in LTL è la reset property,
ovvero che da ogni stato di qualsiasi cammino sia possibile raggiungere
uno stato in cui vale una certa proprietà.

\subsection*{Sintassi e semantica LTL}
Sia $AP$ l'insieme delle proposizioni atomiche, $p \in AP$ una proposizione
atomica, e $\alpha, \beta$ due WFF della LTL.
L'insieme $WFF_{\tn{LTL}}$ delle formule ben formate della LTL è definito come:
\begin{itemize}
    \item $p \in WFF_{\tn{LTL}}$,
    \item $\tn{true}, \tn{false} \in WFF_{\tn{LTL}}$
    \item $\alpha \lor \beta \in WFF_{\tn{LTL}}$ e $\lnot \alpha \in WFF_{\tn{LTL}}$
    \item $\nextOp \alpha \in WFF_{\tn{LTL}}$
    \item $\finallyOp \alpha \in WFF_{\tn{LTL}}$
    \item $\globallyOp \alpha \in WFF_{\tn{LTL}}$
    \item $\untilOp \alpha \in WFF_{\tn{LTL}}$
\end{itemize}

Siano $\pi = q_0 q_1 \ldots, q_i \in Q$ un cammino definito sul modello di
Kripke, e $\alpha, \beta \in WFF_{\tn{LTL}}$.
\begin{itemize}
    \item $\pi \vDash p \longleftrightarrow p \in I(q_0)$, ovvero $p$ vale
    nello stato iniziale $q_0$ di $\pi$.
    \item $\pi \vDash \lnot \alpha \longleftrightarrow \pi \nvDash \alpha$
    \item $\pi \vDash \alpha \lor \beta \longleftrightarrow \pi \vDash \alpha \lor \pi \vDash \beta$
    \item $\pi \vDash \nextOp \beta \longleftrightarrow \pi^{(1)} \vDash \beta$,
    ovvero $\beta$ vale nello stato successivo $\pi^{(1)}$.
    \marginnote{$\nextOp = \tn{neXt}$\\
    $\finallyOp = \tn{Finally}$\\
    $\globallyOp = \tn{Globally}$\\
    $\untilOp = \tn{Until}$}
    \item $\pi \vDash \finallyOp \beta \longleftrightarrow
    \exists i \in \mathbb{N} : \pi^{(i)} \vDash \beta$,
    ovvero $\beta$ vale in almeno uno stato del cammino $\pi$.
    \item $\pi \vDash \globallyOp \beta \longleftrightarrow
    \forall i \in \mathbb{N} : \pi^{(i)} \vDash \beta$,
    ovvero $\beta$ vale in ogni stato del cammino $\pi$.
    \item $\pi \vDash \alpha \untilOp \beta \longleftrightarrow
    (\pi \vDash \finallyOp \beta) \land
    (\forall h \in \mathbb{N}, h < i : \pi^{(i)} \vDash \alpha)$,
    ovvero $\beta$ è vera in almeno uno stato $q_i$ del cammino e in ogni stato
    precedente a $q_i$ vale $\alpha$.
    \item $\pi \vDash \alpha \weakOp \beta \longleftrightarrow
    \pi \vDash \globallyOp \alpha \lor (\alpha \lor \beta)$, ovvero
    si comporta come l'operatore $\untilOp$ ma non è detto che $\beta$ prima
    o poi diventi vera.
    \item $\pi \vDash \alpha \releaseOp \beta \longleftrightarrow
    \pi \vDash \beta \weakOp (\alpha \land \beta)$, ovvero vale $\beta$
    fino a quando non valgono contemporaneamente $\alpha$ e $\beta$ nello
    stesso stato. Non è però detto che $(\alpha \land \beta)$ prima o
    poi sia vero (c'è l'operatore weak), e quindi in questo caso
    sarà sempre vero $\beta$.
\end{itemize}

In realtà, gli operatori LTL visti fin'ora non formano un insieme minimale
di operatori: alcuni di essi, infatti, possono essere derivati dagli altri.
Di conseguenza, è possibile definire un insieme minimale di operatori.
LTL ammette più insiemi minimali di operatori; uno di questi è
$\cbra{\nextOp, \untilOp}$.

\subsection*{Esempi di formule LTL}
\begin{itemize}
    \item $\finallyOp \globallyOp \alpha$: $\alpha$ è invariante da un certo
    istante in poi.
    \item $\globallyOp \finallyOp \alpha$: $\alpha$ è vera in un numero
    infinito di stati.
    \item $\globallyOp \lnot (c_1 \land c_2)$: mutua esclusione. $c_1$ e
    $c_2$ sono le sezioni critiche dei due processi che interagiscono.
    \item $\globallyOp (\text{req} \rightarrow \nextOp \finallyOp \text{ack})$:
    ogni richiesta viene sempre prima o poi gestita.
    \item $\globallyOp (\text{req} \rightarrow (\text{req} \untilOp \text{ack}))$:
    fino a quando la richiesta viene gestita, la richiesta rimane pending.
\end{itemize}

\subsection*{Algoritmo di model checking per LTL}
L'algoritmo di model checking per LTL prevede di costruire un automa di
B\"uchi per la formula LTL e costruire un automa a stati finiti etichettato
a partire dal modello di Kripke, per poi verificare se l'intersezione
tra i due automi, ovvero il loro prodotto sincrono, riconosce il linguaggio
vuoto. Solo in questo caso, infatti, la formula LTL risulta verificata
sul modello di Kripke preso in considerazione.

\begin{defn}
    Un \textbf{automa di Buchi} è una quintupla $(Q, \Sigma, \delta, q_0, F)$
    dove:
    \begin{itemize}
        \item $Q$ è un insieme finito di stati;
        \item $\Sigma$ è l'alfabeto di input;
        \item $\delta : Q \times \Sigma \rightarrow Q$ è la funzione di transizione;
        \item $q_0$ è lo stato iniziale.;
        \item $F$ è un sottoinsieme di stati finali.
    \end{itemize}
    A differenza di un classico automa a stati finiti, però, la condizione di
    accettazione di un automa di B\"uchi è che la sequenza (infinita)
    di transizioni compiute sulla parola in ingresso attraversi almeno uno
    stato finale per un numero infinito di volte.
    Di conseguenza, l'automa di B\"uchi può accettare anche stringhe infinite
    sull'alfabeto finito $\Sigma$.
\end{defn}

Siano $M = (Q, T, I)$ il modello di Kripke considerato, $q_0$ il suo stato
iniziale, e $\alpha$ la formula LTL che si vuole verificare su $M$.
L'algoritmo che permette di stabilire se $\alpha$ è valida nello stato iniziale
di $M$, ovvero $q_0$, è il seguente:
\begin{enumerate}
    \item Costruisco l'automa di B\"uchi $B_{\lnot \alpha}$ che riconosce
    tutte le stringhe in cui non vale la formula $\alpha$;
    \item Trasformo l'automa di B\"uchi $M$ in un automa a stati finiti $S$,
    i cui archi sono etichettati con elementi di $2^{AP}$, ovvero i sottoinsiemi
    di proposizioni atomiche che la funzione di interpretazione $I$ assegna
    agli stati del modello di Kripke.
    \item Calcolo il prodotto sincrono $A_{\lnot \alpha} S$ tra l'automa
    $A_{\lnot \alpha}$ e l'automa $S$.
    \item Se il linguaggio riconosciuto dall'automa generato dal prodotto sincrono
    tra i due automi è l'insieme vuoto, allora in $q_0$ vale $\alpha$, ovvero
    $M, q_0 \vDash \alpha$. Di conseguenza, la formula $\alpha$ è verificata
    nel modello $M$.
\end{enumerate}


La complessità dell'algoritmo di model checking per formule LTL
risulta essere:
\[
    T(M, \alpha) = O(|M| \cdot 2^{|\alpha|})
\]
ed è quindi esponenziale.
Le formule che solitamente si vogliono verificare, però, sono molto spesso
di dimensioni ridotte, e la crescita esponenziale è contenuta.
Al contrario, i modelli di Kripke presi in analisi sono solitamente di grandi
dimensioni e sono, di fatto, il fattore più determinante nella crescita della
funzione di complessità.

\subsection{Computational Tree Logic}
La Computational Tree Logic è una logica temporale in cui il tempo è discreto
e organizzato in una struttura ad albero, in cui il futuro da intraprendere
ad un determinato punto di branch non è predeterminato.

La CTL permette di esprimere un sottoinsieme stretto di formule della LTL
e, quindi, non coincide con LTL.
Un esempio di proprietà esprimile in CTL ma non LTL è la reset property, ovvero
$A \globallyOp E \finallyOp p$.

Sintassi:
\begin{itemize}
    \item Proposizione atomiche sono formule valide
    \item la negazione di formule ben formate è una formula ben formata
    \item stessa cosa per implicazione, congiunzione, disgiunzione.
    \item $AX \alpha, EX \alpha$ sono FBF
    \item $AF \alpha, EF \alpha$ sono FBF
    \item $AG \alpha, EG \alpha$ sono FBF
    \item $A(\alpha \untilOp \beta), E(\alpha \untilOp \beta)$ sono FBF
\end{itemize}

A ed E sono quantificatori sui cammini: A indica per ogni cammino, mentre
E indica esiste un cammino tale che.

\subsection*{Nozioni utili per definire un algoritmo di model cheching per CTL}
Un insieme parzialmente ordinato $(A, \le)$, detto anche poset, è una struttura
algebrica formata da un insieme di elementi $A$ e una relazione d'ordine parziale
$\le$ definita su $A$.
Una relazione binaria è una relazione d'ordine parziale sse è riflessiva,
antisimmetrica e transitiva.
In una relazione d'ordine parziale $R$, non è garantito che
$\forall x, y \in A, (x,y) \in R \lor (y,x) \in R$, ovvero non è garantito
che tutti gli elementi siano confrontabili.

Sia $B \subseteq A$ un sottoinsieme di elementi di $A$:
\begin{itemize}
    \item $x \in A$ è un maggiorante di $B$ sse $\forall y \in B, y \le x$.
    \item $x \in A$ è un minorante di $B$ sse $\forall y \in B, x \le y$.
\end{itemize}
Non è sempre garantito che qualsiasi sottoinsieme $B$ ammetta maggiorante
e minorante.
Se $B$ ammette almeno un maggiorante, allora si dice che $B$ è limitato
superiormente.
Se $B$ ammette almeno un minorante, allora si dice che $B$ è limitato
inferiormente.

$x \in B$ è il minimo di $B$ sse $\forall y \in B, x \le y$.
$x \in B$ è il massimo di $B$ sse $\forall y \in B, y \le x$.

L'estremo inferiore di $B$ è il massimo dei minoranti.
L'estremo superiore di $B$ è il minimo dei maggioranti.

\begin{defn}
    Un \textbf{reticolo} (detto anche lattice) è un insieme parzialmente ordinato $(L, \le)$ in cui
    qualsiasi coppia di elementi ammette un estremo superiore e un estremo
    inferiore, ovvero:
    \[
        \forall x, y \in L, \exists x \bigvee y \land \exists x \bigwedge y
    \]
\end{defn}

\begin{defn}
    Un reticolo è un \textbf{reticolo completo} sse
    \[
        \forall B \subseteq L, \exists \bigvee B \land \exists \bigwedge B
    \]
\end{defn}

\begin{defn}
    Dati due poset $(L_1, \le_1)$ e $(L_2, \le_2)$.
    Una funzione $f: L_1 \rightarrow L_2$ si dice \textbf{monotona} sse
    \[
        \forall x,y \in L_1, \quad x \le_1 y \rightarrow f(x) \le_2 f(y)
    \]
\end{defn}

\begin{defn}
    Data una funzione generica $f: X \rightarrow X$.
    Un elemento $x \in X$ è un punto fisso per la funzione $f$ sse $f(x) = x$.
\end{defn}

\begin{exmp}
    I punti fissi della funzione $f: \mathbb{R} \rightarrow \mathbb{R}, f(x) = x^2$
    sono 0 e 1.\\
    La funzione $f: \mathbb{R}^+ \rightarrow \mathbb{R}, f(x) = \log_2(x)$
    non ha alcun punto fisso.
\end{exmp}

\begin{thm}[Teorema di Knaster-Tarski]
    Sia $(L, \le)$ un reticolo completo, e $f: L \rightarrow L$ una funzione
    monotona non descrescente. Allora l'insieme dei punti fissi di $f$ è un reticolo completo
    e, di conseguenza, $f$ ammette un minimo punto fisso e un massimo punto
    fisso.

    \begin{proof}
        La dimostrazione qui effettuata considera solo un caso particolare,
        ovvero quello in cui il reticolo completo è $(2^A, \subseteq)$,
        e non si può quindi considerare completa.

        Si può innanzitutto costruire un insieme $Z$ di \textit{punti prefissi}.
        \[
            Z = \cbra{T \subseteq A \, | \, f(T) \subseteq T}
        \]
        ovvero l'insieme contenente tutti gli elementi del reticolo completo
        che sono sottoinsiemi della loro stessa immagine.
        Sicuramente $Z \neq \emptyset$, in quanto vale sicuramente
        $f(A) \subseteq A$, e quindi $A \in Z$.

        La dimostrazione è strutturata in 3 punti:
        \begin{enumerate}
            \item dimostrare che l'insieme $Z$ contiene tutti i punti
            fissi della funzione $f$.
            \item dimostrare che il minimo $m$ di $Z$ appartiene a $Z$ stesso.
            \item dimostrare che $m$ è punto fisso della funzione $f$.
        \end{enumerate}
        Dopo aver dimostrato questi tre punti, si può concludere che
        $m$ è il minimo punto fisso della funzione $f$.

        \begin{enumerate}
            \item Si può notare che, se la funzione $f$ ammette punti fissi,
            allora i punti fissi di $f$ sicuramente appartengono a $Z$.
            \item Una volta calcolato l'insieme $Z$, che altro non è che un
            insieme di elementi del reticolo, è possibile calcolare
            il suo estremo inferiore. Questa operazione è sicuramente possibile,
            in quanto $(2^A, \subseteq)$ è un reticolo completo e, di conseguenza,
            ogni sottoinsieme di $2^A$ ammette estremo inferiore ed estremo superiore.\\
            \[
                m = \bigcap Z
            \]
            Ovviamente si può affermare che $m \subseteq S, \forall S \in Z$, per
            definizione di estremo inferiore nel reticolo completo $(2^A, \subseteq)$.

            Essendo però $f$ una funzione monotona su $(2^A, \subseteq)$,
            allora vale:
            \[
                m \subseteq S \longrightarrow f(m) \subseteq f(S) \subseteq S \quad
                \forall S \in Z
            \]
            $f(m)$ è sottoinsieme di ogni elemento di $Z$, ovvero
            $f(m) \subseteq S, \forall S \in Z$, sse $f(m) \subseteq \bigcap Z$.
            Però $\bigcap Z = m$ e, di conseguenza, vale $f(m) \subseteq m$.
            Ma, per come è definito l'insieme $Z$, allora
            vale sicuramente $m \in Z$. Di conseguenza, si è dimostrato che il
            minimo di $Z$ appartiene a $Z$ stesso.
            \item Manca solo dimostrare che $m$ è effettivamente un punto fisso
            della funzione $f$. Avendo dimostrato che $f(m) \subseteq m$, dimostrare
            la relazione opposta, ovvero $m \subseteq f(m)$, permetterebbe
            di concludere che $m$ è punto fisso, in quanto di otterrebbe
            $f(m) = m$.

            Si è precedentemente dimostrato nel punto 2 che vale
            $f(m) \subseteq m$. Essendo $f$ una funzione monotona,
            applicando la funzione $f$ a entrambi i membri della disuguaglianza
            si ottiene $f(f(m)) \subseteq f(m)$.
            Ma, per come è stato definito $Z$, allora vale $f(m) \in Z$.
            L'estremo inferiore $m$ di $Z$, per definizione di estremo inferiore,
            è sicuramente sottoinsieme di ogni elemento di $Z$: si può quindi
            affermare che $m \subseteq f(m)$.

            Avendo dimostrato $f(m) \subseteq m$ e $m \subseteq f(m)$,
            allora si può concludere che $f(m) = m$. Di conseguenza,
            $m$ è un punto fisso di $f$.
        \end{enumerate}
        Avendo dimostrato che $Z$ contiene tutti i punti fissi della funzione $f$,
        il minimo $m$ di $Z$ appartiene a $Z$ stesso e $m$ è un punto fisso,
        allora si può concludere che $m$, calcolato come l'intersezione di
        tutti i sottoinsiemi di $A$ contenuti in $Z$, è il suo minimo
        punto fisso.

        Lo stesso ragionamento può essere applicato per dimostrare l'esistenza
        del massimo punto fisso: l'unica differenza sta nel fatto che
        il sottoinsieme $m$ non verrà più calcolato come estremo inferiore
        di $Z$, ma come estremo superiore.
    \end{proof}
\end{thm}

Il teorema di Knaster-Tarski afferma che una funzione monotona
crescente su un reticolo completo ammette sempre un minimo punto fisso
e un massimo punto fisso. Non specifica, però, un vero e proprio
algoritmo di calcolo per questi punti fissi.
Il metodo di calcolo viene, invece, esposto nel teorema di Kleene.

\begin{thm}[Teorema di Kleene]
    Sia $(L, \le)$ un reticolo completo, e $f$ una funzione continua
    secondo Scott, e quindi monotona.\\
    Il minimo punto fisso di $f$ è l'estremo superiore della
    catena di Kleene crescente su $f$ che parte da $\bot$:
    \[
        \bot \le f(\bot) \le f(f(\bot)) \le \ldots \le f^n(\bot) \le \ldots
    \]
    dove $\bot$ è il minimo del reticolo completo $(L, \le)$.

    Il massimo punto fisso di $f$ è l'estremo superiore della catena
    di Kleene crescente su $f$ che parte da $\top$:
    \[
        \top \le f(\top) \le f(f(\top)) \le \ldots \le f^n(\top) \le \ldots
    \]
    dove $\top$ è il massimo del reticolo completo $(L, \le)$.

    \begin{rem}
        Il minimo punto fisso è calcolato applicato iterativamente $f$
        a partire dal minimo del reticolo completo $(L, \le)$, fino
        a quando il risultato fornito dalla funzione è uguale all'ultimo
        argomento.

        Il massimo punto fisso è calcolato allo stesso modo, ma a partire
        dal massimo del reticolo completo $(L, \le)$.
    \end{rem}
\end{thm}

\subsection*{Algoritmo per model checking con CTL}
Per poter spiegare l'algoritmo di model checking per CTL, è necessario prima
capire cos'è l'estensione di una formula $\alpha$.
\begin{defn}
    Sia $M = (Q, T, I)$ un modello di Kripke e $\alpha$ una formula CTL.
    L'\textbf{estensione} di $\alpha$ è un sottoinsieme di stati del modello
    in cui vale $\alpha$.
    \[
        \llbracket \alpha \rrbracket = \cbra{q \in Q \,|\, M, q \vDash \alpha}
    \]
\end{defn}

Si consideri ora un modello di Kripke $M = (Q, T, I)$ su cui si vuole
valutare una formula CTL $\alpha$.
$(2^Q, \le)$ è un reticolo completo, dove $\le$ è una relazione di ordine
temporale.
\upperAccE possibile definire, per ogni operatore CTL, un'apposita funzione,
la quale permette di calcolare l'estensione dell'operatore CTL applicato
a una formula $\alpha$ a partire dall'estensione di $\alpha$. In particolare,
le estensioni degli operatori CTL sono minimi o massimi punti fissi
per le rispettive funzioni associate.

COMPLETARE CON TUTTE LE REGOLE

Si consideri $\alpha \equiv A \finallyOp \beta$.
Si può definire la funzione $f_{\alpha} : 2^Q \longrightarrow 2^Q$ come:
\[
    f_{\alpha}(H) = \llbracket \beta \rrbracket \cup
    \cbra{q \in Q \, | \, \forall (q, q') \in T : q' \in H}
    \quad H \subseteq Q
\]
ovvero la funzione $f_{\alpha}$ aggiunge all'estensione $\llbracket \beta \rrbracket$
tutti gli stati del modello di Kripke per cui tutti gli archi uscenti
finiscono in uno stato contenuto in $\llbracket \beta \rrbracket$.

Si può affermare che $\llbracket \alpha \rrbracket$, ovvero l'estensione
di $\alpha$, è il minimo punto fisso di $f_{\alpha}$, e può essere calcolato
secondo il teorema di Kleene.

Si consideri ora $\alpha \equiv E \globallyOp \beta$.
Si può definire la funzione $g_{\alpha} : 2^Q \longrightarrow 2^Q$ come:
\[
    g_{\alpha}(H) = \llbracket \beta \rrbracket \cap
    \cbra{q \in Q \, | \, \exists(q, q') \in T : q' \in H} \quad
    H \subseteq Q
\]
ovvero la funzione $g_{\alpha}$ aggiunge all'estensione
$\llbracket \beta \rrbracket$ tutti gli stati del modello di Kripke
per cui esiste almeno un arco uscente che finisce in uno stato contenuto
in $\llbracket \beta \rrbracket$.

Si può affermare che $\llbracket \alpha \rrbracket$, ovvero l'estensione
di $\alpha$, è il massimo punto fisso di $g_{\alpha}$, e può essere
calcolato secondo il teorema di Kleene.

La complessità dell'algoritmo di model checking per CTL è pari a:
\[
    T(M, \alpha) = O(|M| \cdot |\alpha|)
\]
SI può quindi affermare che il model checking di una formula CTL
risulta computazionalmente più efficiente del model checking di una formula
LTL.

\subsection{Confronto tra logiche temporali}
Alcune proprietà possono essere espresse sia in LTL che in CTL.
Esempi sono le invarianti:
\[
    A \globallyOp \lnot p \equiv G \lnot p
\]
Reattività:
\[
    A \globallyOp (p \rightarrow A \finallyOp  q) \equiv
    G (p \rightarrow \finallyOp q)
\]

Alcune proprietà CTL non possono essere espresse in LTL.
Un esempio è la reset property, ovvero che da ogni stato raggiungibile
in ogni cammino sia sempre possibile raggiungere uno stato in cui vale
$p$. Reset property in CTL:
\[
    A \globallyOp E \finallyOp p
\]

Un esempio di formula LTL che invece non si può esprimere in CTL è
$\finallyOp \globallyOp p$, ovvero che in ogni cammino, prima o poi
$p$ diventa sempre vera.

\section{Limiti del model checking}
Nei precedenti capitoli sono stati studiati alcuni modelli formali,
come le algebre di processo e le reti di Petri, che permettono di
rappresentare i sistemi reattivi, ovvero tutti quei sistemi
concorrenti, distribuiti e asincroni, che non obbediscono al paradigma
input-computazione-output, in quanto idealmente non hanno uno stato
di partenza e non terminano, ma continuano a offrire le loro funzionalità.

A partire da questi modelli, è possibile definire i relativi modelli di Kripke,
in modo da poter verificare alcune proprietà sul sistema e stabilirne
quindi la correttezza. Questa costruzione molto spesso, però, può portare
a un'esplosione combinatoria degli stati, aumentando notevolmente
i tempi di verifica e la quantità di memoria richiesta per rappresentarli.
Per questo motivo, usando i modelli "puri" visto fin'ora, possono essere
analizzati in un tempo ragionevole solo modelli con un numero di stati
che si aggira intorno al milione.

Nel corso degli anni, però, sono state sviluppate tecniche, come
gli algoritmi simbolici
\footnote{Gli algoritmi simbolici permettono di rappresentare lo spazio
degli stati del modello di Kripke usando delle funzioni, evitando quindi
di doverlo memorizzare esplicitamente e contenendo quindi la quantità
di memoria richiesta.},
il Bounded Model Checking
\footnote{Il Bounded Model Checking genera lo spazio degli stati fino a
una profondità prefissata, e verifica la formula solo sugli stati generati.
Il risultato di un Bounded Model Checking è quindi ternario: la formula
è verificata, la formula non è verificata, e non è possibile stabilire
se la formula è verificata.
Si può dimostrare, inoltre, che esiste una riduzione dal problema BMC a SAT:
BMC può quindi essere risolto usando algoritmi di SAT solving.}
e il model checking al volo
\footnote{Il model checking al volo prevede di tenere in memoria solo quegli
stati del modello di Kripke utili nella dimostrazione della formula temporale
considerata, evitando quindi di memorizzare tutti gli stati e riducendo
l'esplosione combinatoria.},
che permettono di memorizzare più efficacemente gli stati,
contenendo quindi l'esplosione combinatoria, e di velocizzare i tempi di verifica.
L'utilizzo di queste tecniche permette di trattare modelli di Kripke che possono
raggiungere un numero di stati elevatissimo, che si aggira intorno a $10^{120}$.

\section{Model checking e sviluppo software}
L'utilizzo del model checking nello sviluppo software si traduce principalmente
in quello che prende il nome di \textit{paradigma model based}: in questo paradigma
di programmazione, lo sviluppo è guidato da modelli formali del sistema
che si vuole implementare, su cui vengono verificate formule logiche
temporali, in modo da stabilirne la correttezza.
In particolare, questo paradigma prevede che, prima ancora di
di scrivere il codice, i requisiti vengano completamente modellati
con un modello di Kripke e controllati tramite model checking, in modo da
rilevare più errori possibili nell'implementazione e anticiparli
il più possibile nel processo di sviluppo. Solo dopo aver effettuato una
fase di model checking adeguata, allora il sistema potrà essere effettivamente
implementato tramite scrittura di codice.
Avendo però già descritto il sistema come un modello formale, molto spesso
la scrittura di codice a partire dal modello può essere effettuata da
traduttori automatici, contenendo il tempo richiesto per l'implementazione:
in questo modo, la maggior parte delle risorse può essere concentrata
sulla fase iniziale di modellazione e model checking, assicurando un'alta
qualità del software.

Un altro approccio che può essere utilizzato prende il nome di
\textit{Software Model Checking}, e consiste nel tradurre codice già scritto
in un modello di Kripke, in modo da poter verificarne alcune proprietà
tramite model checking. Questo approccio risulta più complesso, in quanto
già solo due variabili a 32 bit generano uno spazio degli stati con $2^{64}$
elementi. Sono stati comunque sviluppati tool che permettono di verificare
alcune proprietà sul codice, e in alcuni casi che accettano direttamente
in input il codice stesso, senza la necessità di convertirlo prima in un
modello di Kripke
\footnote{Esempi di software model checker sono SLAM (C),
CMBC (C++) e JavaPathFinder (Java)}.\\

\section{SPIN e NuSVM: tool per il model checking}

\section{SLAM: software model checker per il linguaggio C}

\section{Approfondimento: $\mu$-calcolo e fairness}
\subsection*{$\mu$-calcolo}
Il $\mu$-calcolo è un superset del $\text{CTL}^*$: ogni formula esprimibile
in LTL, CTL e $\text{CTL}^*$ può quindi essere riscritta in $\mu$-calcolo.

Il $\mu$-calcolo ha quindi la massima espressività, ma ha un'alta complessità
e le sue formule sono molto difficili da interpretare.

\subsection*{Sintassi $\mu$-calcolo}
Sia $AP$ l'insieme delle proposizioni atomiche, $p \in AP$ una proposizione
atomica, $A$ l'insieme finito delle azioni, e $V$ l'insieme
numerabile delle variabili.
L'insieme $WFF_{\mu}$ delle formule ben formate del $\mu$-calcolo è definito come:
\begin{itemize}
    \item $p \in WFF_{\mu}$,
    \item $\tn{true}, \tn{false} \in WFF_{\mu}$
    \item $\alpha \lor \beta \in WFF_{\mu}$ e $\lnot \alpha \in WFF_{\mu}$
    \item $[a]\phi \in WFF_{\mu}$
    \item $\ang{a}\phi \in WFF_{\mu}$
    \item $v Z . \phi \in WFF_{\mu}$
    \item $\mu Z . \phi \in WFF_{\mu}$
\end{itemize}


\subsection*{Semantica $\mu$-calcolo}
La semantica del $\mu$-calcolo è definita su modelli di Kripke attraverso
operatori di punto fisso.

\subsection*{Fairness}
Un evento è unfair se rimane sempre abilitato.
Si può quindi definire il concetto di fairness, che impedisce che
l'esecuzione della rete sia ingiusto e prediliga alcuni eventi rispetto
ad altri.

Esistono due tipi di fairness:
\begin{itemize}
    \item fairness debole: se un evento è abilitato, allora prima o poi
    diventerà disabilitato.
    \item fairness forte: se un evento è abilitato, allora prima o poi
    scatterà.
\end{itemize}
