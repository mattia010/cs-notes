\chapter{Logica di Hoare}
\section{Definizione della logica}
La logica di Hoare è una logica utilizzata per dimostrare la correttezza di
programmi sequenziali imperativi. Come tutte le logiche, la logica di Hoare
ha una propria sintassi, una propria semantica e un proprio apparato deduttivo,
che permette di ottenere nuove formule ben formate.
In particolare, la logica di Hoare si basa sul concetto di tripla, che definisce
quali proprietà debbano essere vere prima e dopo l'esecuzione del programma
considerato.

Sia $V$ l'insieme delle variabili utilizzate dal programma di cui si vuole
dimostrare la correttezza.
$\sigma: V \rightarrow \mathbb{Z}$ è la funzione che assegna
a ogni variabile un valore intero\footnote{Non si perde di generalità
considerando sole variabili intere.}, e di conseguenza è la funzione
che determina lo \textbf{stato della memoria} ad un certo istante di tempo.
Una formula della logica proposizionale è vera in uno stato della memoria
(definito da una certa funzione $\sigma$) sse la funzione $\sigma$
considerata assegna un valore alle variabili coinvolte nella formula tale
che essa risulti soddisfatta.

Come già detto, la logica di Hoare si basa sul concetto di tripla:
\begin{center}
    $\Hoaretriple{\alpha}{C}{\beta}$
\end{center}
dove:
\begin{itemize}
    \item $\alpha$, detta precondizione, è una WFF della logica proposizionale
    che deve essere vera sullo stato della memoria prima di eseguire $C$;
    \item $C$, detto comando, è il programma che deve essere eseguito;
    \item $\beta$, detta post-condizione, è una WFF della logica proposizionale
    che deve essere vera sullo stato della memoria dopo l'esecuzione di $C$.
\end{itemize}
Di conseguenza una tripla di questo tipo viene letta come:
\begin{center}
    \textit{Se $C$ viene eseguito a partire da uno stato della memoria in
    cui vale $\alpha$, e $C$ termina, allora nello stato della memoria
    finale raggiunto a seguito dell'esecuzione di $C$ vale $\beta$.}
\end{center}
Se si dimostra questa affermazione, allora il programma \tn{C} considerato
risulta corretto.


\subsection*{Sintassi}
La sintassi della logica di Hoare considera come formule valide tutte
le formule della logica proposizionale e tutte le triple della forma:
\[
    \Hoaretriple{p}{C}{q}
\]
dove $p$ è la precondizione del programma,
$C$ è un programma valido secondo la logica di Hoare, e $q$ è la postcondizione
del programma. $p$ e $q$ sono formule ben formate della logica proposizionale.

Per definire quali programmi sono effettivamente considerati validi, però,
è prima necessario definire l'insieme $E$ delle espressioni aritmetiche
e l'insieme $B$ delle formule booleane.

Grammatica delle espressioni aritmetiche:
\begin{itemize}
    \item $\forall v \in V$ (insieme delle variabili), $v \in E$;
    \item $\forall z \in \mathbb{Z}$, $z \in E$;
    \item $\forall e_1, e_2 \in E$, $(e_1+e_2), (e_1-e_2), (e_1*e_2), (e_1/e_2) \in E$.
\end{itemize}
Grammatica delle espressioni booleane:
\begin{itemize}
    \item $\textnormal{true}, \textnormal{false} \in B$;
    \item $\forall b_1, b_2 \in B$, $(b_1 \land b_2), (b_1 \lor b_2), \lnot b_1 \in B$;
    \item $\forall e_1, e_2 \in E$, $(e_1 < e_2), (e_1 = e_2), (e_1 \neq e_2), (e_1 \le e_2) \in B$.
\end{itemize}

Si può ora definire l'insieme dei programmi validi nella logica di Hoare.
Sia $C$ l'insieme dei programmi (detti comandi) validi della logica di Hoare,
allora la grammatica per $C$ è:
\begin{itemize}
    \item istruzione di skip $\in C$:
    \begin{center}
        skip
    \end{center}
    \item istruzione di assegnamento $\in C$:
    \begin{center}
        $v:=e \quad v \in V \land e \in E$
    \end{center}
    \item istruzione di scelta $\in C$:
    \begin{center}
        if $b$ then $c_1$ else $c_2$ endif $\quad b \in B \land c_1, c_2 \in C$
    \end{center}
    \item istruzione di iterazione $\in C$:
    \begin{center}
        while $b$ do $c$ endwhile $\quad b \in B \land c \in C$
    \end{center}
    \item istruzione di sequenza $\in C$:
    \begin{center}
        $c_1 ; c_2 \quad c_1, c_2 \in C$
    \end{center}
\end{itemize}

\subsection*{Apparato deduttivo}
L'apparato deduttivo contiene tutte le regole di inferenza, che permettono di
derivare nuove triple valide a partire da formule ben formate già dimostrate e di
dimostrare la correttezza di un programma.
Tutte le regole di inferenza sono corrette, ovvero se le triple
premesse sono dimostrate e si può derivare da esse una nuova tripla
tramite una regola, allora la tripla derivata è dimostrata.

\begin{itemize}
    \item regola della sequenza:
    \begin{center}
        \begin{prooftree}
            \AxiomC{$\Hoaretriple{p_1}{C}{p_2}$}
            \AxiomC{$\Hoaretriple{p_2}{D}{p_3}$}
            \BinaryInfC{$\Hoaretriple{p_1}{C;D}{p_3}$}
        \end{prooftree}
    \end{center}
    \item regola della scelta:
    \begin{center}
        \begin{prooftree}
            \AxiomC{$\Hoaretriple{p \land b}{C}{q}$}
            \AxiomC{$\Hoaretriple{p \land \tn{¬b}}{D}{q}$}
            \BinaryInfC{$\Hoaretriple{p}{\tn{if } b \tn{ then } C \tn{ else } D \tn{ endif}}{q}$}
        \end{prooftree}
    \end{center}
    \item regola di skip:
    \begin{center}
        \begin{prooftree}
            \AxiomC{}
            \UnaryInfC{$\Hoaretriple{p}{\tn{skip}}{p}$}
        \end{prooftree}
    \end{center}
    \item regola dell'implicazione (su pre-condizione):
    \begin{center}
        \begin{prooftree}
            \AxiomC{$p' \rightarrow p$}
            \AxiomC{$\Hoaretriple{p}{C}{q}$}
            \BinaryInfC{$\Hoaretriple{p'}{C}{q}$}
        \end{prooftree}
    \end{center}
    \item regola dell'implicazione (su post-condizione):
    \begin{center}
        \begin{prooftree}
            \AxiomC{$\Hoaretriple{p}{C}{q}$}
            \AxiomC{$q \rightarrow q'$}
            \BinaryInfC{$\Hoaretriple{p}{C}{q'}$}
        \end{prooftree}
    \end{center}
    \item regola dell'assegnamento:
    \begin{center}
        \begin{prooftree}
            \AxiomC{}
            \UnaryInfC{$\Hoaretriple{q[\sfrac{E}{x}]}{x:=E}{q}$}
        \end{prooftree}
    \end{center}
    dove $\cbra{q}$ è un WFF che può contenere la variabile $x$, mentre
    $\cbra{q[\sfrac{E}{x}]}$ è la formula $q$ in cui però ogni occorrenza
    di $x$ è stata sostituita con l'espressione $E$.
    \footnote{
        Esempio di derivazione della pre-condizione in cui l'espressione
        contiene a sua volta la variabile:\\
        $\Hoaretriple{q[\sfrac{i+1}{i}]}{i = i + 1}{s=x^i}$\\
        $q[\sfrac{i+1}{i}] = s=x^{i+1}$\\
        $\Hoaretriple{s=x^{i+1}}{i = i + 1}{s=x^i}$
    }.

    \item regola di iterazione (per sola correttezza parziale):
    \begin{center}
        \begin{prooftree}
            \AxiomC{$\Hoaretriple{\text{inv} \land \beta}{C}{\text{inv}}$}
            \UnaryInfC{$\Hoaretriple{\text{inv}}
                                    {\tn{while } \beta \tn{ do } C \tn{ endwhile}}
                                    {\text{inv} \land \tn{¬}B}$}
        \end{prooftree}
    \end{center}
    $\text{inv}$ è una WFF che risulta vera prima dell'esecuzione di $W$
    e a seguito di qualsiasi iterazione. Per questo motivo, prende il
    nome di \textbf{invariante di ciclo}.
\end{itemize}

\subsection*{Correttezza e completezza della logica di Hoare}
\marginnote{Un linguaggio è Turing-completo sse è dotato di un'istruzione
di assegnamento, un'istruzione di scelta, e un'istruzione iterativa.
La logica di Hoare presente tutti questi tipi di istruzioni, ed è
quindi Turing-completa.}
La logica di Hoare è corretta, ovvero ogni tripla derivabile a partire
da triple dimostrate è a sua volta una tripla dimostrata.
Essa non risulta però completa, ovvero se una tripla è dimostrabile, non
è detto che sia derivabile. Questo è dovuto al fatto che la logica di Hoare
è più potente dell'aritmetica di Peano, e per il teorema di Godel, una logica
di questo tipo non può essere contemporaneamente corretta e completa.
Se, però, tutte le formule vere dell'aritmetica di Peano vengono assunte
come assiomi della logica di Hoare, allora la logica di Hoare risulta
completa. Per questo si dice che la logica di Hoare è parzialmente completa.

\section{Dimostrazione di correttezza}
Dimostrare la correttezza di un programma (imperativo) usando la logica di
Hoare significa dimostrare semplicemente la validità di una tripla
$\Hoaretriple{p}{C}{q}$ usando l'apparato deduttivo della logica, dove
$C$ è il programma di cui si vuole dimostrare la correttezza.
In particolare, dimostrare la tripla $\Hoaretriple{p}{C}{q}$ significa
poter costruire una tripla valida $\Hoaretriple{p'}{C}{q}$, a partire dai soli comando e postcondizione ed usando
le regole di inferenza, in modo che $p' \rightarrow p$.

La dimostrazione di un programma inizia dalla sua ultima istruzione:
si cerca infatti di trovare una precondizione tale che la tripla
che coinvolge la sola ultima istruzione e la postcondizione data sia valida.
Una volta individuata questa precondizione, questo passaggio viene ripetuto
prendendo in considerazione la penultima istruzione e considerando come
sua postcondizione la precondizione appena trovata.
Questo passaggio viene effettuato per tutte le istruzioni del programma,
procedendo all'indietro, fino a quando non si ottiene una precondizione $p'$
per la prima istruzione del programma: se $p'$ implica la precondizione $p$
della tripla che si vuole dimostrare, allora la tripla sarà dimostrata.

Dimostrare la correttezza di programmi non iterativi risulta relativamente
semplice, e può essere fare in maniera completamente algoritmica.
Se invece il programma presenta almeno un'istruzione iterativa, allora
la dimostrazione si complica: per gestire questa istruzione, infatti,
devono essere individuati un invariante e un variante. Questa attività
è però un'attività "creativa", e sta quindi al dimostratore occuparsi di
trovare l'invariante e il variante più adatti alla dimostrazione.

La dimostrazione della correttezza di una tripla si divide in:
\begin{itemize}
    \item dimostrazione di \textbf{correttezza parziale}: viene dimostrata
    la derivabilità della tripla.\\
    Se si esegue il ciclo a partire da uno stato della memoria in cui vale
    la precondizione e si assume che il ciclo termini,
    allora nello stato di memoria finale vale la post-condizione.
    \footnote{La derivabilità implica la correttezza della tripla, in quanto
    la logica di Hoare è una logica corretta.}.
    \item dimostrazione di \textbf{correttezza totale}: viene dimostrata sia
    la derivabilità della tripla che la terminazione di eventuali istruzioni
    di iterazione.
\end{itemize}

\subsection*{Dimostrazione di correttezza di un'istruzione iterativa}
Per la dimostrazione della correttezza parziale di un programma
con almeno un'istruzione iterativa, bisogna individuare
un'invariante di ciclo, ovvero una WFF che è vera sia prima del ciclo
che dopo la sua esecuzione.
Un'istruzione iterativa può però ammettere più di un'invariante: bisogna
quindi individuare quella più utile per la dimostrazione che si sta effettuando.

Sia $W$ il ciclo definito come:
\[
    W \equiv \text{ while } \beta \text{ do } C \text{ endwhile}
\]
Si vuole trovare un'invariante inv tale che:
\begin{enumerate}
    \item la tripla $\Hoaretriple{\beta \land \text{ inv}}{C}{\text{inv}}$
    sia valida. Per considerarla valida, la tripla deve essere dimostrata.
    \item una volta dimostrata la tripla
    $\Hoaretriple{\beta \land \text{ inv}}{C}{\text{inv}}$, si può utilizzare
    la regola dell'iterazione per derivare una nuova tripla
    $\Hoaretriple{\text{inv}}{W}{\lnot \beta \land \text{ inv}}$.
    La postcondizione di questa nuova tripla, ovvero
    $\lnot \beta \land \text{ inv}$, deve implicare la postcondizione
    data per il ciclo che si sta considerando. Se ciò non accade, allora
    l'invariante individuata non è adatta.
\end{enumerate}
Se è stato possibile derivare una nuova tripla con la regola di
iterazione e la postcondizione ottenuta, ovvero $\lnot \beta \land \text{ inv}$,
implica la postcondizione data, allora inv è la precondizione del ciclo.
Si continua poi con la dimostrazione dell'intero programma e, se la precondizione
finale che si ottiene implica la precondizione della tripla che coinvolge
tutto il programma e che si vuole dimostrare, allora si è dimostrata la correttezza
parziale dell'intero programma.

La dimostrazione della sola correttezza parziale non implica la correttezza
di un programma con almeno un'istruzione iterativa: non è infatti garantito
che questi cicli, effettivamente, terminino. Si deve quindi procedere con
la dimostrazione di correttezza totale.
Per dimostrare che un ciclo termina è semplicemente necessario trovare
una espressione $E$, detta variante, tale che:
\begin{itemize}
    \item $\text{inv } \rightarrow E \ge$, ovvero l'invariante del ciclo considerato,
    precedentemente trovata nella dimostrazione di correttezza parziale,
    implica che il variante assume un valore maggiore o uguale a 0.
    \item $\Hoaretriple{\tn{inv} \land \beta \land E = k \land E > 0}{C}{\tn{inv} \land E < k}$
    \footnote{$k$ è una variabile di appoggio utilizzata per memorizzare
    il valore di $E$ prima dell'esecuzione di $C$, in modo da poterlo
    confrontare con il nuovo valore di $E$ al termine dell'esecuzione di $C$.}
    è una tripla derivabile, e quindi il valore assunto da $E$ diminuisce a
    seguito di ogni iterazione del ciclo, ovvero a seguito di ogni esecuzione
    di $C$.
\end{itemize}
Riuscire a trovare una variante per un'istruzione di iterazione implica la
correttezza totale dell'istruzione di iterazione, ovvero la sua terminazione.

\subsection{Esempi}
\begin{exmp}
    Dimostrare la correttezza della tripla:
    \begin{center}
        $\Hoaretriple{x>1}{y:=x+1}{y>1}$
    \end{center}
    A partire da $y:=x+1 \cbra{y>1}$, cerco di derivare una precondizione
    $p'$ tale per cui $x>1 \rightarrow p'$.
    Per la regola dell'assegnamento:
    \begin{center}
        $\vDash \Hoaretriple{x+1>1}{y:=x+1}{y>1}$
    \end{center}
    che si può riscrivere come:
    \begin{center}
        $\vDash \Hoaretriple{x>0}{y:=x+1}{y>1}$
    \end{center}
    Per la regola dell'implicazione (sulla pre-condizione):
    \begin{center}
        \begin{prooftree}
            \AxiomC{$x>1 \rightarrow x>0$}
            \AxiomC{$\Hoaretriple{x>0}{y:=x+1}{y>1}$}
            \BinaryInfC{$\Hoaretriple{x>1}{y:=x+1}{y>1}$}
        \end{prooftree}
    \end{center}
    Sono quindi riuscito a derivare la tripla che si voleva dimostrare.
    La tripla di partenza è quindi dimostrata.
\end{exmp}

\begin{exmp}
    Dimostrare la correttezza della tripla:
    \begin{center}
        $\Hoaretriple{s=x^i \land i < N}{i:=i+1; s:=s*x}{s=x^i \land i \le N}$
    \end{center}
    Il comando è composto da due istruzioni di assegnamento, e sarebbe
    quindi necessario applicare due volte la regola di assegnamento.
    Quando gli assegnamenti consecutivi sono indipendenti
    \footnote{Per assegnamenti indipendenti si intende che se è presente un
    assegnamento a una certa variabile, allora quella variabile non può
    comparire nelle espressioni aritmetiche degli altri assegnamenti.},
    però, è possibile effettuare contemporaneamente tutte le sostituzioni
    sulla post-condizione.
    \begin{center}
        $\vDash \Hoaretriple{s*x=x^{i+1} \land i+1 \le N}{i:=i+1; s:=s*x}{s=x^i \land i \le N}$
    \end{center}
    che si può riscrivere come:
    \begin{center}
        $\vDash \Hoaretriple{s=x^i \land i+1 \le N}{i:=i+1; s:=s*x}{s=x^i \land i \le N}$
    \end{center}
    Per la regola dell'implicazione (sulla pre-condizione):
    \begin{prooftree}
        \AxiomC{ $i < N \rightarrow i+1 \le N$ }
        \AxiomC{ $\Hoaretriple{\begin{aligned}
            &s*x=x^{i+1}\\
            &\land \; i+1 \le N
        \end{aligned}}
        {\begin{aligned}
            &i:=i+1;\\
            &s:=s*x
        \end{aligned}}
        {\begin{aligned}
            &s=x^i\\
            &\land i \le N
        \end{aligned}}$ }
        \BinaryInfC{ $\Hoaretriple{s=x^i \land \; i < N}
        {\begin{aligned}
            &i:=i+1; \\
            &s:=s*x
        \end{aligned}}
        {s=x^i \land \; i \le N}$ }
    \end{prooftree}
    Si è quindi riusciti a derivare la tripla di partenza. La tripla di
    partenza è quindi dimostrata.
\end{exmp}

\begin{exmp}
    Dimostrare la correttezza della tripla:
    \begin{center}
        $\Hoaretriple{\textnormal{T}}
        {\begin{aligned}
            & \textnormal{if } x<y\\
            & \textnormal{then } m:=y\\
            & \textnormal{else } m:=x\\
            & \textnormal{endif}
        \end{aligned}}
        {m \ge x \land m \ge y \land (m=x \lor m=y)}$
    \end{center}
    Usando la regola di scelta al contrario, la tripla può essere scomposta
    in due triple:
    \begin{center}
        $t_1 = \Hoaretriple{x<y}{m:=y}{m \ge x \land m \ge y \land (m=x \lor m=y)}$\\
        $t_2 = \Hoaretriple{x \ge y}{m:=x}{m \ge x \land m \ge y \land (m=x \lor m=y)}$
    \end{center}
    Bisogna quindi dimostrare la correttezza di entrambe le triple.
    Considerando $t_1$, per la regola dell'assegnamento:
    \begin{center}
        $\vDash \Hoaretriple{\begin{aligned}
            &y \ge x \land y \ge y\\
            &\land (y = x \lor y = y)
        \end{aligned}}
        {m:=y}
        {\begin{aligned}
            &m \ge x \land m \ge y\\
            &\land (m=x \lor m=y)
        \end{aligned}}$
    \end{center}
    che si può riscrivere come:
    \begin{center}
        $\vDash \Hoaretriple{y \ge x}{m:=y}
        {\begin{aligned}
            &m \ge x \land m \ge y\\
            &\land (m=x \lor m=y)
        \end{aligned}}$
    \end{center}
    Per la regola dell'implicazione (sulla pre-condizione):
    \begin{prooftree}
        \AxiomC{$x<y \rightarrow y \ge x$}
        \AxiomC{$\Hoaretriple{y \ge x}{m:=y}
        {\begin{aligned}
            &m \ge x \land m \ge y\\
            &\land (m=x \lor m=y)
        \end{aligned}}$}
        \BinaryInfC{$\Hoaretriple{x<y}{m:=y}
        {m \ge x \land m \ge y \land (m=x \lor m=y)}$}
    \end{prooftree}
    Si è quindi riusciti a derivare $t_1$, e quindi si è dimostrata la sua
    correttezza.\\
    Considerando $t_2$, per la regola dell'assegnamento:
    \begin{center}
        $\vDash \Hoaretriple{\begin{aligned}
            &x \ge x \land x \ge y\\
            &\land (x=x \lor x=y)
        \end{aligned}}
        {m:=x}
        {\begin{aligned}
            &m \ge x \land m \ge y\\
            &\land (m=x \lor m=y)
        \end{aligned}}$
    \end{center}
    che si può riscrivere come:
    \begin{center}
        $\vDash \Hoaretriple{x \ge y}{m:=x}{\begin{aligned}
            &m \ge x \land m \ge y\\
            &\land (m=x \lor m=y)
        \end{aligned}}$
    \end{center}
    Si è quindi riusciti a derivare $t_2$, e quindi si è dimostrata la sua
    correttezza.\\
    Avendo dimostrato la correttezza di entrambe le triple, allora anche la
    tripla di partenza è dimostrata.
\end{exmp}

Per dimostrare la correttezza di una tripla della forma $\{ p \} x:=E \{ q \}$
si deve partire considerando solo comando e la post-condizione.
Usando la regola dell'assegnamento, si deriva la pre-condizione
$q[\sfrac{E}{x}]$. Se $q[\sfrac{E}{x}] \rightarrow p$ (implicazione logica),
allora la tripla di partenza è dimostrata.

\section{Weakest precondition}
Siano $t(\sigma)$ la funzione che restituisce l'insieme delle formule valide
nello stato della memoria $\sigma$, e sia $m(p)$ la funzione che ritorna
l'insieme degli stati della memoria in cui vale $p$.

Per dimostrare la correttezza di una tripla, bisogna individuare una
precondizione che permetta di costruire una tripla valida
con il comando e la post-condizione dati, ovvero per cui se il comando
è eseguito da uno stato della memoria in cui vale la pre-condizione, allora
nello stato finale della memoria dopo l'esecuzione vale la post-condizione.
Una precondizione, però, non è altro che un vincolo sui possibili stati della memoria
che vengono accettati. Più sottoinsiemi di stati della memoria possono
portare alla costruzione di una tripla valida.

La weakest precondition è quindi la precondizione più generica, che impone
meno vincoli sullo stato iniziale del programma. Essa è la pre-condizione
di cardinalità massima, ovvero quella che comprende più stati della memoria
possibili che, a seguito dell'esecuzione di $C$, permettono di raggiungere
uno stato in cui vale la post-condizione.

Dato un comando $C$ e una post-condizione $q$, per trovare la precondizione
più debole $wp(C, q)$ per cui la tripla risulta valida possono essere
usate le seguenti regole:
\begin{itemize}
    \item assegnamento:
    \[
        wp(x:=E \; , \; q) = q[\sfrac{E}{x}]
    \]
    \item sequenza:
    \[
        wp(C_1;C_2 \; , \; q) = wp(C_1 \; , \; wp(C_2 \; , \; q))
    \]
    \item scelta:
    \[
        wp \bra{\begin{aligned}
            & \text{if } \beta\\
            & \text{then } C\\
            & \text{else } D\\
            & \text{endif}
        \end{aligned} \; , \; q} =
        (B \land wp(C \; , \; q)) \lor (\lnot B \land wp(D \; , \; q))
    \]
    \item iterazione:
    \[
        wp \bra{\begin{aligned}
            & \textnormal{while } \beta\\
            & \textnormal{do } C\\
            & \textnormal{endwhile}
        \end{aligned} \; , \; q} =
        (\lnot \beta \land q) \lor (\beta \land
        wp \bra{\begin{aligned}
            & C;\\
            & \text{while } \beta\\
            & \text{do } C\\
            & \text{endwhile}
        \end{aligned} \; , \; q})
    \]
    Ovviamente questa definizione ricorsiva non può essere tradotta in una procedura
    automatica. Risulta comunque essere corretta.
\end{itemize}

\section{Design by Contract}
Design by Contract è un paradigma di programmazione le cui radici affondano
nella verifica formale dei sistemi e nella logica di Hoare.
Questo paradigma, infatti, richiede che il programmatore specifichi, per
ogni componente software del sistema, interfacce formali, precise e verificabili:
queste interfacce si traducono di fatto con l'affiancamento, a ogni metodo/funzione,
di una precondizione, una postcondizione e una o più invarianti.
Queste specifiche prendono il nome di contratto.
Il Design by Contract chiama poi il metodo/funzione con il nome di
\textit{supplier}, mentre fa riferimento al chiamante con il nome di
\textit{client}.

Design by Contract assume inoltre che le funzioni non debbano verificare
se gli argomenti, ricevuti a seguito di una chiamata, rispettino
la precondizione: è infatti compito del chiamante assicurarsi che i valori
effettivi passati alla funzione rispettino il contratto associato alla
funzione stessa.
Al contrario, una funzione deve assicurare che la postcondizione sia
rispettata, e deve quindi effettuare adeguati controlli al termine della
sua esecuzione.\\
Nel caso in cui le funzioni sia particolarmente critiche, però,
è solitamente consigliato implementare controlli anche sulla precondizione.

\section{Logica di Hoare e la sua inadeguatezza per i sistemi reattivi}
La logica di Hoare è un formalismo utile per modellare sistemi (programmi)
sequenziali e dimostrarne la correttezza, secondo una certa specifica.
Si basa sul concetto di comando per il modello, e su
pre-condizione e post-condizione per esso. La pre-condizione deve
valere nello stato iniziale del sistema, mentre la post-condizione deve
valere nello stato finale.

Nei sistemi concorrenti
\footnote{Un sistema concorrente è un sistema composto da diverse parti che:
\begin{itemize}
    \item competono per l'accesso a risorse condivise
    \item cooperano per un obiettivo comune
    \item si coordinano
\end{itemize}}, però, l'esecuzione risulta essere non deterministica
e non composizionale: di conseguenza, la logica di Hoare non si
rivela adatta a descriverne il comportamento e verificarne le proprietà.
Inoltre, i sistemi concorrenti, molto spesso, sono sistemi reattivi,
ovvero sistemi che non hanno alcuno stato iniziale
\footnote{Nella pratica, anche i sistemi reattivi hanno uno stato di
partenza.} e alcuno stato finale, e il cui obiettivo è quello di mantenere
la sua esecuzione. Non seguendo il tradizionale paradigma
input-computazione-output, i sistemi reattivi non possono essere modellati
tramite logica di Hoare.\\
La logica di Hoare rimane comunque utile nel model checking di sistemi
concorrenti: essi sono infatti molto spesso formati da sottosistemi
sequenziali, la cui correttezza può essere verificata tramite la logica
di Hoare.

Nei prossimi capitoli verranno introdotti alcuni modelli formali
utili per la trattazione di sistemi concorrenti e
distribuiti. Verrà anche introdotta la tecnica del model checking, che
permette di stabilire se un sistema concorrente gode o meno di una
certa proprietà.
