\appendix
\chapter{Rappresentazione dei numeri}
Un numero può essere rappresentato in una base $b$ usando un sistema
di numerazione posizionale, in cui i valori delle cifre cambiano in base
alle posizioni in cui esse compaiono all'interno del numero.

\begin{rem}
    Un numero reale può ammettere più rappresentazioni in una base $b$, ma
    non è detto che tutte queste rappresentazioni siano finite.
\end{rem}

\begin{example}
    \[
        1 = 0,\overline{9}
    \]
    \begin{proof}
        $0,\overline{9}$ può essere scritto come $0,9 + 0,09 + 0,009 + \ldots$,
        ovvero come:
        \[
            \sum_{i=1}^{+\inf} 9 \cdot 10^{-i} =
            9 \cdot \sum_{i=1}^{+\inf} 10^{-i} =
            9 \cdot \sum_{i=0}^{+\inf} 10^{-i-1} =
            9 \cdot \sum_{i=1}^{+\inf} 10^{-i} 10^{-1} =
            \frac{9}{10} \cdot \sum_{i=1}^{+\inf} 10^{-i}
        \]
        Ma si può osservare che $\sum_{i=1}^{+\inf} 10^{-i}$ è una serie
        geometrica, e quindi:
        \[
            \frac{9}{10} \cdot \sum_{i=1}^{+\inf} 10^{-i} =
            \frac{9}{10} \cdot \frac{1}{1-\frac{1}{10}} =
            \frac{9}{10} \cdot \frac{10}{9} = 1
        \]
        Si è quindi dimostrato che $0,\overline{9} = 1$.
    \end{proof}
\end{example}

\begin{thm}
    Dato un numero razionale scritto nella forma $\frac{p}{q}$ e già
    ridotto ai minimi termini, ovvero la cui frazione non può essere
    ulteriormente semplificata, allora esso ammette rappresentazione
    finita in una base $b$ sse tutti i fattori primi di $q$ dividono $b$.
\end{thm}
\begin{example}
    $0,\overline{3}$ non ammette rappresentazione finita in base $10$.
    Esso può, infatti, essere scritto come $\frac{1}{3}$. I divisori primi
    del denominatore sono $\cbra{1, 3}$. $3$ non divide però $b=10$,
    e quindi non è ammessa rappresentazione finita.

    Se però si considera $b=3$, allora $0,\overline{3}$ ammette rappresentazione
    finita: tutti i divisori primi del denominatore, ovvero $\cbra{1, 3}$,
    dividono $b=3$.
\end{example}

\chapter{Aritmetica floating point}
L'aritmetica floating point è un'aritmetica che permette di operare su un
sottoinsieme dei numeri reali, i cui elementi possono essere rappresentati
nella forma
\[
    \pm M_b \cdot b^E \quad E \in \mathbb{Z}
\]
dove $M_b$, un numero intero in base $b$, prende il nome di mantissa,
$b$ è la base in cui si sta rappresentando il numero e $E$ è l'esponente.
Dando per scontato che si stia operando in un sistema di numerazione in
base $b$, $b$ può essere dato per scontato, e si può identificare univocamente
un numero in aritmetica floating point usando segno, mantissa e esponente.

Si consideri ora solo l'aritmetica floating point in base $2$, ovvero quella
utilizzata dai calcolatori moderni.
Il calcolatore rappresenta i numeri floating point, e
in particolare mantissa ed esponente, con un numero costante di
bit. Ha quindi senso dire che con un numero costante di bit è
possibile rappresentare tutti gli esponenti da un minimo esponente $L$
a un massimo esponente $U$. Lo stesso ragionamento può essere applicato
anche ai valori interi che la mantissa può assumere.

I calcolatori moderni usano lo standard IEEE 754, che definisce come
rappresentare i numeri floating point e definisce diversi gradi di precisione
che richiedono un numero diverso di bit. Uno di questi gradi di precisione
è il \verb|double|: un numero floating point a doppia precisione utilizza
$64$ bit, di cui $1$ viene usato per specificare il segno, $52$ per la mantissa
e $11$ per l'esponente ($1+52+11=64$).
Il valore che l'esponente può assumere è limitato tra $L=-1022$ e $U=1023$.
Si può osservare però che $11$ bit permettono di rappresentare i numeri con
segno compresi tra $-1023$ e $1024$, e quindi due valori sono mancanti.
Queste due rappresentazioni mancanti sono utilizzate per rappresentare
\verb|NaN| (Not a Number) e \verb|Inf| (Infinity).
Per quanto riguarda la mantissa, essa viene rappresentata come un intero
ma in realtà questo intero è implicitamente la parte decimale di un numero
decimale la cui parte intera è 0. Inoltre, la prima cifra decimale
del numero decimale rappresentato dalla mantissa è sempre diversa da $0$,
perchè altrimenti esso potrebbe riscritto come un numero con prima cifra
decimale diversa da $0$ modificando semplicemente l'esponente.
Questo bit sempre diverso da $0$ viene quindi sottointeso, e si può rappresentare
questo numero decimale usando un bit in meno. La mantissa utilizza
effettivamente 52 bit e, di conseguenza, essa permette di rappresentare
in realtà un numero binario decimale con 53 bit dopo la virgola.

\chapter{Algebra lineare}
