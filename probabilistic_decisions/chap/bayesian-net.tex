\chapter{Reti Bayesiane}
Le reti Bayesiane sono un modello grafico probabilistico che,
attraverso i concetti di indipendenza condizionata tra variabili, permette
di memorizzare e trattare efficientemente le distribuzioni di probabilità
congiunta che, se memorizzate esplicitamente tramite tabella delle
probabilità congiunte, richiederebbero una grandissima quantità di memoria,
oltre che un elevato tempo di calcolo.
\footnote{Nel caso di 100 variabili booleane, la tabella delle probabilità
congiunte presenta $2^100$ celle, ognuna contiene una probabilità congiunta
che deve essere calcolata. Eventuali aggiornamenti causati dall'osservazione
di una variabile, porterebbe al ricalcolo di moltissime celle.}.
Un altro svantaggio, parzialmente risolto nelle reti Bayesiana, è che le
probabilità congiunte richiedono moltissimi dati per essere stimate.

Formalmente, una rete Bayesiana è un grafo diretto aciclico (d'ora in poi
DAG), i cui nodi rappresentano le variabili aleatorie che si stanno considerando,
e gli archi le relazioni di dipendenza causale diretta tra esse.
Inoltre, ogni nodo è etichettato con una tabella di probabilità condizionata (CPT),
in cui è espressa la distribuzione di probabilità condizionata della variabile
associata al nodo condizionata dalla congiunzione di tutti i genitori,
ovvero le variabili associate ai nodi genitori del nodo considerato.
Una tabella delle probabilità congiunte "quantifica" l'impatto che i nodi
genitori hanno sul valore che la variabile associata al nodo considerato
può assumere.
Se un nodo non presenta nodi genitori, ovvero archi entranti, la sua tabella
delle probabilità congiunte conterrà semplicemente la distribuzione
a priori (non condizionata) della variabile associata al nodo.

La topologia della rete, ovvero i nodi presenti e gli archi tra essi, e le
tabelle di probabilità condizionata permettono di definire completamente
la distribuzione di probabilità congiunta di tutte le variabili aleatorie
che si stanno considerando.
Come la tabella delle probabilità congiunte, una rete Bayesiana permette di
fare inferenza, ovvero interrogare il modello in modo tale da calcolare una
probabilità desiderata: questo è possibile proprio grazie al fatto che
una rete Bayesiana permette di descrivere completamente, anche se non
in maniera esplicita, la distribuzione di probabilità congiunta su tutte le
variabile considerate.
Si può calcolare la probabilità di un evento atomico che coinvolge tutte
le variabili della rete come:
\[
    P(x_1, \ldots, x_n) = \Pi_{i=1}^{n} P(x_i | \text{parents}(x_i))
\]
dove con $x_i$ si intende che la variabile $X_i$ assume il valore $x_i$,
mentre $\text{parents}(x_i)$ (con la p minuscola) si indicano che
le variabili associate ai nodi genitori assumono i valori dettati
dall'evento atomico considerato.
In realtà, la probabilità può essere calcolata anche sulla congiunzione
di un sottoinsieme di variabili, anche se in questo caso il calcolo risulta
più complesso.

Sia $x_i$ la variabile considerata, e $\text{Parents}(x_i)$ l'insieme
delle variabili associate ai nodi genitori. Allora la tabella delle
probabilità condizionate associata al nodo di $x_i$ avrà un numero
di parametri, ovvero un numero di entrate/probabilità pari a
$\sum_{p \in \text{Parents}(x_i)} |p|$, dove $|p|$ è il numero di valori
che la variabile $p$ può assumere.

Per quanto riguarda la complessità di una rete Bayesiana, meno complessa
è la topologia della rete, minore è la complessità: si riducono infatti
sia il numero di parametri contenuti nelle tabelle di probabilità
condizionata, sia i nodi da aggiornare quando vengono modificate le distribuzioni
di probabilità associate a un sottoinsieme di variabili.
Per questo motivo, è importante definire la rete in modo che essa esprima
il dominio desiderato, mantenendo comunque una bassa complessità della topologia.
Esistono comunque tecniche che, data una rete Bayesiana, permettono di
semplificare la topologia, introducendo però errori quantificabili. In questo
caso, però, bisogna valutare i tradeoff tra complessità e qualità del modello
richiesta.

\section{Costruire una rete Bayesiana}
Per costruire una rete Bayesiana, è necessario innanzitutto aver studiato
il dominio che si sta considerando, ed aver raccolto dei dati che
permettano di stimare i parametri della rete (ed eventualmente anche
stabilire le relazione di dipendenza condizionale).

L'algoritmo per la sua costruzione si divide in cinque fasi:
\begin{enumerate}
    \item seleziono le variabili da modellare;
    \item scelgo l'ordinamento corretto delle variabili, in modo che
    se due variabili sono legate da una relazione di causa ed effetto,
    allora la variabile rappresentante l'effetto compare successivamente
    nell'ordinamento rispetto alla causa;
    \item inizializzo la rete;
\end{enumerate}
