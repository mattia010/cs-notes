\chapter{Metodi agili e DevOps}
I metodi agili sono tutti quei metodi di sviluppo che:
\begin{itemize}
    \item gestire meglio i cambiamenti di requisiti;
    \item si concentrano sulle persone anziché sulla produzione di artefatti. In processi di sviluppo classici, infatti, la documentazione molto spesso viene prodotta per poi non essere utilizzata;
    \item prevedono una comunicazione diretta e verbale all'interno del team, e anche tra team e stakeholder;
    \item prevedono un testing continuo dell'applicazione, in modo da individuare il più presto possibile eventuali errori.
\end{itemize}

La struttura di un generico processo agile prevede la presenza di diverse iterazioni, ognuna dedicata a un sottoinsieme di requisiti.\\
A inizio progetto, il team procede con uno studio parziale dei requisiti richiesti dagli stakeholder.
Vengono poi eseguite le iterazioni, ognuna con una durata specifica. Prima di ogni iterazione, il team sceglie un sottoinsieme di requisiti su cui si vuole concentra nella successiva iterazione.
Alla fine di un'iterazione, il team avrà prodotto una parte del software completo ma già funzionante e possibilmente testata e integrata.
Idealmente, il team non dovrebbe tornare a lavorare su parti del prodotto finale sviluppate in precedenti iterazioni, a meno di cambiamenti nei requisiti.

Alcuni esempi di metodi agili sono Scrum e Extreme Programming (XP).

\section{Scrum}
Scrum è un processo agile con iterazioni di durata compresa tra 1 e 4 settimane.

\begin{defn}
    Uno \textbf{Sprint} è un'iterazione del processo Scrum.
\end{defn}

\begin{defn}
    Un \textbf{Daily Scrum meeting} è un incontro che si tiene ogni giorno (solitamente la mattina) a cui partecipano tutto il team e lo Scrum Master.

    Durante il Daily Scrum meeting, il team discute brevemente sullo stato dell'iterazione e su cosa fare durante la giornata.
\end{defn}

\begin{defn}
    Il \textbf{product backlog} è l'insieme dei requisiti parziali definito all'inizio del progetto.
\end{defn}

\begin{defn}
    Lo \textbf{Sprint backlog} è il sottoinsieme dei requisiti scelti per una determinata iterazione.
\end{defn}

\begin{defn}
    Lo \textbf{Spring planning meeting} è un incontro che si tiene all'inizio di ogni iterazione, in cui vengono definiti la durata dell'interazione e lo Sprint backlog.
\end{defn}

\begin{rem}
    Alla fine di ogni iterazione, il team effettua un altro iincontro, in cui discute di eventuali criticità rilevate durante l'iterazione.
\end{rem}

Il processo Scrum prevede i seguenti ruoli:
\begin{itemize}
    \item team: si occupa dello sviluppo del software;
    \item Scrum master: controlla che il processo Scrum sia correttamente applicato;
    \item product owner: rappresentante di tutti gli stakeholder, a conoscenza di tutti i requisiti che il software deve rispettare.
\end{itemize}

\section{DevOps}
Il processo di sviluppo di un software è distribuito tra tre entità: stakeholder, team di sviluppo e team di operations. Le relazioni tra queste entità sono stakeholder-team di sviluppo e team di sviluppo-team di operations \footnote{Il team di operations si occupa di mettere in campo e rendere operativo il sistema sviluppato dal team di sviluppo. Effettuano inoltre il monitoring del sistema.}.\\
I metodi agili permettono di migliorare la relazione tra stakeholder e team di sviluppo ma, in alcuni casi, può essere utile anche migliorare la relazione tra team di sviluppo e team di operations, soprattutto in casi di frequenti release che modificano il sistema.

DevOps permette quindi di semplificare e facilitare la messa in campo delle release prodotte dal team di sviluppo.

DevOps si basa su una forte \textbf{automazione} dei processi e sulla loro \textbf{ripetibilità}.
Tutti i processi automatizzati implementati da DevOps riguardano:
\begin{itemize}
    \item Continous Development: 
    \item Continous Integration: processi che si occupano di integrare correttamente i cambiamenti nel sistema esistente.
    \item Continous Testing: processi che si occupano di verificare l'adeguatezza dei cambiamenti apportati.
    \item Continous Deployment: processi che si occupano di rendere operativo il sistema a cui sono state apportate modifiche.
    \item Continous Monitoring: processi che si occupano di monitorare il sistema e ottenere feedback.
\end{itemize}

I feedback in DevOps, quindi, non sono raccolti dagli utenti, ma da tool automatici, che forniscono aggiornamenti continui e dettagliati sullo stato del sistema.

I ruoli previsti da DevOps sono:
\begin{itemize}
    \item DevOps Evangelist: colui che conosce la pratica DevOps e controlla che sia applicata. Nella pratica, l'evangelist non è quasi mai presente ma vengono adottati gli strumenti tipici di DevOps;
    \item automation expert: colui che gestisce i processi di automazione, creandoli e modificandoli quando necessario;
    \item release manager;
    \item figura per la Quality Assurance;
    \item ingegnere della sicurezza.
\end{itemize}

DevOps segue 6 principi: la cosa più importante è che tutto deve essere automatizzato, testato e monitorato.

